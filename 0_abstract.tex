{
 \thispagestyle{empty}
 \large
 \noindent
 平成27年度 修士論文内容梗概
 \\
 \begin{center}
  \textbf{\LARGE フラグメント分割に基づく\\高速な化合物プレドッキング手法の開発}
 \end{center}
 \\
 \hfill 
 \begin{tabular}{llll}
 指導教員& \multicolumn{3}{l}{秋山 泰 教授} \\
  \multicolumn{4}{l}{計算工学専攻} \\
 14M38400\hspace{.5cm} & {柳澤 渓甫}
 \end{tabular}

\mbox{}\\
コンピュータによる予測を利用し、創薬コストの削減および創薬期間の短縮を行うバーチャルスクリーニングと
呼ばれる手法が近年広く用いられている。その中でも、タンパク質や化合物の立体構造情報を用いた
ドッキングシミュレーションを行う手法(Structure-Based Virtual Screening, SBVS)は
標的タンパク質に対する薬剤・阻害剤が知られていなくとも行うことのできる手法であり、
また既知の薬剤の情報を用いないために既知の薬剤と構造的に大きく異なる新たな化合物の発見につながるなど、
創薬の現場で大きく期待されている手法である。

しかし一方で、ドッキングシミュレーションは標的タンパク質と化合物とがどのような合体構造をとれば
最良のスコアを得られるか、という最適化問題を解くが、相対的な3次元位置、回転状態および、化合物内部の結合の
回転という多くの探索パラメータが存在するため計算コストが高く、データベースに存在する化合物をすべて評価するのが
困難な場合が多い。この問題を解決するため、化合物を何らかの手法を用いてフィルタリングし、個数が減少した
化合物サブセットについてドッキングシミュレーションを行う、というプロトコルが多々実践されている。
ここで、構造ベースの化合物選別手法の利点である既知の薬剤・阻害剤を用いないこと、構造的に新規な化合物を
発見することの2点を損なわずにフィルタリングを行うことが重要となるが、
従来の手法の多くはこの要件を達成することが出来ていないのが現状である。

そこで本研究では、化合物の部分構造に着目することで、ドッキングに基づきつつ高速にフィルタリングを行う手法
(プレドッキング)を提案する。この手法では化合物を内部に回転可能な結合を持たない部分構造(フラグメント)に
分割し、標的タンパク質とフラグメントとの間でドッキングシミュレーションを行う。
フラグメントへの分割によって化合物内部の回転を考慮せずに良くなるため、ドッキングシミュレーションの
最適化問題の探索パラメータを減少させることができ、結果として従来のドッキングベースのフィルタリング手法と比べて
平均して9倍程度の高速化を達成した。この高速化効率は化合物ライブラリ(データベース)のサイズが大きいほど
向上する傾向が示されている。
また、フラグメントのドッキングシミュレーション結果のスコアからフィルタリングに用いる化合物のスコアへの算出方法を
複数検討し、フィルタリング後の化合物ベースのドッキング計算も含めて評価した場合に
従来手法にくらべ予測精度・計算速度どちらも改善されるケースが多数存在することを示した。
\memo{(1050文字程度)}

\thispagestyle{empty}
\addtocounter{page}{-1}
}
