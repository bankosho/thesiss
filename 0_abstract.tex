{
 \thispagestyle{empty}
 \large
 \noindent
 平成25年度 学士論文内容梗概
 \\
 \begin{center}
  \textbf{\LARGE 半教師付き学習を用いた薬物クリアランス経路予測}
 \end{center}
 \\
 \hfill 
 \begin{tabular}{llll}
 指導教員& \multicolumn{3}{l}{秋山 泰 教授} \\
  \multicolumn{4}{l}{情報工学科} \\
 10\_24379\hspace{.5cm} & {柳澤 渓甫}
 \end{tabular}

\mbox{}\\
\small 
近年,新薬の開発にかかる費用は莫大であり,安全性や有効性を確かめるためにも10年以上の開発期間が必要となっている.そのため,新薬開発の後期において安全性などの理由で開発を中止するとなるとその損失は大きなものとなる.
このことから,合成前や臨床試験前の段階で薬物候補となる化合物の選定精度を向上させ,開発後期における中止を極力削減することが非常に重要である.

薬物候補化合物の選定はコンピュータを用いたシミュレーションによる手法が安全性・有効性どちらの観点からも研究されており,当研究室でも教師付き学習を用いて化合物の生化学的な性質を表す4種類の値からヒト体内で薬物が代謝・排出される経路(クリアランス経路)を予測するシステムの開発が行われてきたが,精度はいまだ十分ではない.

%薬物候補化合物の選定にはさまざまな方法が取り入れられているが,生体内などではなくコンピュータ上でのシミュレーションに基づく(\it{in silico}と呼ばれる)手法が注目されている.この手法は化合物の病気に対する有効性,副作用などの安全性,どちらにも応用され,盛んに研究が行われている.
%当研究室では教師付き学習を用いて化合物の生化学的な性質を表す値からヒト体内で薬物が代謝・排出される経路(クリアランス経路)を予測するシステムの開発を行ってきた.

本研究のデータとなる薬物のクリアランス経路を知るためにかかるコストは大きく,問題の複雑さに対してデータ数が少ない一方で,化合物そのものは低コスト・大量にデータを収集することができる.このように実際の分類結果が判明しているラベルありデータの取得が難しく,ラベルなしデータの取得が容易であるような予測問題には半教師付き学習が適していると言われている.半教師付き学習は,ラベルなしデータを用いることで教師付き学習よりも精度を向上させようとする機械学習手法である.

そこで本研究では,ラベルありデータ240個に対してラベルなしデータを最大800個追加して半教師付き学習を用いることでクリアランス経路予測システムの精度の改善を目指したが,精度の向上を得ることはできなかった.入力の化合物の生化学的な性質を表す値を貪欲法を用いて最大13個追加し,特徴空間を疎にした上で再度半教師付き学習を用いたが,やはり精度の向上は得られなかった.

一方,これまでの研究においてドッキング計算による結合自由エネルギー推定値を入力値としてシステムに追加することで予測精度が向上することがわかっていたため,より精密かつ薬学的知見に基づいたドッキング計算を行うことで得られた結合自由エネルギー推定値を利用し,CYP1A2およびCYP2D6のクリアランス経路予測の精度を向上させた.

\thispagestyle{empty}
\addtocounter{page}{-1}
}
