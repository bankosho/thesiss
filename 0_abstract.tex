{
 \thispagestyle{empty}
 \large
 \noindent
 平成27年度 修士論文内容梗概
 \\
 \begin{center}
  \textbf{\LARGE フラグメント分割に基づく\\高速な化合物プレドッキング手法の開発}
 \end{center}
 \\
 \hfill 
 \begin{tabular}{llll}
 指導教員& \multicolumn{3}{l}{秋山 泰 教授} \\
  \multicolumn{4}{l}{計算工学専攻} \\
 14M38400\hspace{.5cm} & {柳澤 渓甫}
 \end{tabular}

\mbox{}\\

創薬では、大量の候補化合物から薬になりうる化合物を選別する作業が日常的に行われている。
このうちコンピュータによる予測を利用してコスト削減および開発期間の短縮を目指す「バーチャルスクリーニング」と
呼ばれる手法が近年盛んに用いられている。中でも、タンパク質や化合物の立体構造情報を用いた
ドッキングシミュレーションを行う手法(Structure-Based Virtual Screening, SBVS)は、
標的タンパク質に対する既知の薬剤情報がなくても適用できる手法である。
既知の薬剤情報を用いないことから既知薬剤と構造的に大きく異なる新たな化合物の発見につながるなど、
SBVSは創薬の現場で大きく期待されている。

一方で、ドッキングシミュレーションは標的タンパク質と化合物との最適な複合体構造の探索を必要とするが、
相対的な3次元位置、回転および、化合物内部の結合の回転という多くの探索パラメータが存在するため計算コストが高く、
大量の化合物すべてを評価することは困難な場合が多い。
この問題を解決するため、化合物を何らかの方法でフィルタリングし、数を減らした化合物サブセットについて
ドッキングシミュレーションを行う、という段階的なプロトコルが多く実践されている。
しかし従来手法は、既知の薬剤情報を用いたフィルタリングや、構造類似度などを用いたフィルタリングを行っており、
SBVSと組み合わせた時にSBVSの利点が生かされないという問題が存在していた。

そこで本研究では、ドッキングシミュレーションに基づいた上で、化合物の部分構造を用いることでより高速にフィルタリングを行う
手法(プレドッキング)を提案する。提案手法では、まず全ての候補化合物をフラグメントと呼ばれる内部に回転可能な結合を持たない
部分構造に分割し、標的タンパク質とフラグメントとの間でドッキングシミュレーションを行う。このとき、フラグメントへの分割により回転自由度
がなくなるため、ドッキングシミュレーションの最適化問題の探索パラメータが減少し、計算の高速化が可能となる。
これにより、高速だが粗い探索を行うドッキングシミュレーションであるGlide HTVSモードに比べても、提案手法は一般的なベンチマークデータセット
(DUD-E)を用いたときに平均して約9倍の高速化を達成した。
フラグメントごとのドッキング結果は化合物間で再利用が可能であるため、高速化率は評価対象の化合物数が多いほど高くなる。

また、フラグメントのドッキングシミュレーション結果のスコアからフィルタリングに用いる化合物のスコアへの算出方法を
複数検討し、フィルタリング後の化合物ベースのドッキング計算も含めて評価した。その結果、提案手法は
従来手法に比べて精度が最大約25\%、速度が最大約40\%改善されるケースが存在することを示した。
\memo{(950-1000文字程度)}

\thispagestyle{empty}
\addtocounter{page}{-1}
}
