\chapter{立体構造に基づいた化合物の選別(Structure-based virtual screening, SBVS)}


\section{SBVSとは}
\begin{itemize}
\item 化合物およびタンパク質の立体構造情報を用いることで、大量の化合物から薬物のタネとなる化合物(リード化合物)を
	コンピュータを用いて選別すること
\item 一般的に、化合物-タンパク質ドッキングシミュレーションが行われる
\item LBVSに比べて、以下の点で優れている %フィルタリング手法の部分でLB的な手法を紹介するため、これは再度記載しておく
	\begin{itemize}
	\item 既知の阻害剤などを必要としない
	\item 多様な候補化合物を得ることができる
	\end{itemize}
\end{itemize}


\section{化合物-タンパク質ドッキングシミュレーション}
\subsection{ドッキングシミュレーションの要素}
探索アルゴリズムの問題とスコア関数の問題がある、ということを述べる。

\subsection{既存のドッキングシミュレーションツール}
Glide(SP, HTVS), eHiTS, Autodockについて、どのように問題解決しているかを簡潔に述べる。計算時間を示している比較論文があればよいが。


\section{ドッキングシミュレーションの問題点}
計算量が大きくて大量の化合物に対するドッキングは時間を要すること、
Glideなど有償ソフトはライセンス的な問題もあることを記述。


\section{化合物のフィルタリング手法}

\subsection{フィルタリング手法の必要性}
ドッキングシミュレーションの計算時間問題を解決するためにドッキングツールの高速化が様々行われているが、
根本的な解決になっていない。したがって、ドッキングツールのinputにする化合物数を減らすフィルタリングが必要になる、というストーリーを記述する。
\subsection{既存のフィルタリング手法}
実際にSBVSを行っている論文を例示し、どのようなフィルタリング手法が行われているかを示す
\subsection{既存手法の問題点}
以下の2点を示す。
\begin{itemize}
\item Glide HTVSをフィルタリングに転用する手法では高速化の度合いが不十分であり、
	仮想的に化合物空間を広げたようなデータセットに対するドッキングなどを行う条件下では計算時間的に現実的ではない
\item 化合物ベースの手法(含Pharmacophore)は構造ベースよりも一般に高速だが、既知阻害剤があることが条件になり、
	さらに得られる化合物は既知の化合物に似てしまうという問題がある(前述した通り~みたいな感じで)
\end{itemize}

%
%図\ref{fig:docking_freedom}を用いてドッキングの重さについても言及。
%\begin{figure}[htb]
% \begin{center}
%  \fig[width=0.6\hsize]{./fig/background/ドッキングの探索自由度.eps}
%  \caption{探索自由度について}
%  \label{fig:docking_freedom}
% \end{center}
%\end{figure}
%
%\section{化合物のフィルタリング}
%\comment{機械学習でフィルタリングしてからだと何故ダメなのか(新規性が云々)。LBDD vs SBDDの議論と混ざらないように注意。(大上先生)}
%\begin{itemize}
%\item Ligand-basedなフィルタリング\\
%	リピンスキー\r{ref}など、タンパク質の構造によらない手法について
%\item Structure-basedなフィルタリング\\
%	glide HTVSのようなドッキングに基づいた手法を説明する。
%\end{itemize}