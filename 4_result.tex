\chapter{実験}
ここで話すことは大きく二つ。

どこかで評価軸の説明をする
\begin{itemize}
\item それぞれの手法の単独性能について
\item それぞれの手法をfiltering手法として用いた場合について
\end{itemize}
\section{データセット}
DUD-E\r{ref}を利用した、だけではなく、inputとoutputを明確にする。
\section{計算環境}
TSUBAME Thinノード
\section{実験結果}
\subsection{ドッキングにかかる計算時間}
\subsection{予測精度}
ここでは単独手法としての精度を示す。EF1, 2, 5, 10\%, AUCを数値として出し、
ROC曲線はAppendixとして載せていることを記述
\subsection{filteringと通常のドッキングとを合わせた場合の計算時間および精度}
filtering時に何\%の化合物を残すか、などの議論と合わせて、精度と速度のバランスを示す。
