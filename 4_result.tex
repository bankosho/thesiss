\chapter{実験}
\comment{現状よりも詳しく記述すべき(大上先生)}

ここで話すことは大きく二つ。

どこかで評価軸の説明をする
\begin{itemize}
\item それぞれの手法の単独性能について
\item それぞれの手法をfiltering手法として用いた場合について
\end{itemize}


\section{データセット}
本実験では、データセットとしてDirectory of Useful Decoys (DUD-E)\citetodo{}を用いた。DUD-Eは102種類のターゲットについて、それぞれタンパク質・正例化合物・負例化合物を用意している。各ターゲットの詳細、および1つの化合物の平均フラグメント数については付録\ref{appendix:dude}に記載する。\memo{DUD-Eの一部の情報はココに記載する必要あり}


\section{予測精度の評価指標}
バーチャルスクリーニングでは一般的に正例と負例の比が非常に偏っている。
そのため単純なprecisionやrecallを用いた評価を行うことができず、以下の2種類の評価指標が多く用いられている。
本実験においても、この2種類を予測精度の評価指標として用いる。

\subsection{ROC-AUC}
Receiver Operating Characteristic(ROC)曲線の曲線下面積(Area Under the Curve, AUC)。
ROC曲線に必要なFP-RateとTP-Rateの数式を示し、算出例を示す。\todo{卒論から文章をひっぱってくる}

\subsection{EF (EnrichmentFactor)}
数式と算出例を示す。ROC-AUCと同じ例を用いた方が良い?


\section{計算環境}
TSUBAME Thinノード\todo{先輩の修論の記述をパク…参考にする}


\section{実験結果}


\subsection{ドッキングにかかる計算時間}
1化合物あたりの計算時間をCPU時間で示す。
フラグメントドッキングの場合も、フラグメントの個数でCPU時間を除算せずに化合物数で除算する。
\memo{ドッキングの計算時間のみを示してよいのか?フラグメント分割やスコア統合の計算時間をどう示すべきか(所要時間:後者$<<$前者$<<$ドッキング計算)}

\subsection{予測精度}
ここでは単独手法としての精度を示す。ROC曲線はAppendixとして載せていることを記述
\comment{ROC曲線も本文に示す。良い/悪いの評価をする際に必要になるため。(大上先生)}

\begin{table}[htb] \centering
	\caption{提案手法の予測精度}
	\label{tb:filtering_accuracy}
	\begin{tabular}{c|c|lllll|}
	\multirow{2}{*}{手法}				&フラグメント		&\multirow{2}{*}{ROC-AUC}	&\multicolumn{4}{c}{Enrichment Factor}	\\
									&ドッキング		&						&EF(1\%)	&EF(2\%)	&EF(5\%)	&EF(10\%)	\\ \hline
	\multirow{2}{*}{総和(score\_sum)}		&glide 通常モード	&						&		&		&		&			\\
									&glide 高速モード	&						&		&		&		&			\\
	\multirow{2}{*}{最良値(score\_max)}	&glide 通常モード	&						&		&		&		&			\\
									&glide 高速モード	&						&		&		&		&			\\
	\multirow{2}{*}{混合(maxsumBS)}		&glide 通常モード	&						&		&		&		&			\\
									&glide 高速モード	&						&		&		&		&			\\ \hline
	\multicolumn{2}{c|}{従来手法(glide 高速モード)}			&						&		&		&		&			\\ \hline
	\end{tabular}
\end{table}

\subsection{filteringと通常のドッキングとを合わせた場合の計算時間および精度}
filtering時に何\%の化合物を残すか、などの議論と合わせて、精度と速度のバランスを示す。
\comment{トレードオフを示すのについて、フィルタリングのパーセンテージをもっと振るべきでは?5, 10, 20, 30, 40, 50\%のように。(大上先生)}
\begin{figure}[htp]
 \begin{center}
  \fig[width=0.99\hsize]{./fig/result/性能vs時間.eps}
  \caption{計算時間と精度のトレードオフ}
  \label{fig:trade_off}
 \end{center}
\end{figure}
