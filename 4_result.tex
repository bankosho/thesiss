\chapter{実験}
ここで話すことは大きく二つ。

どこかで評価軸の説明をする
\begin{itemize}
\item それぞれの手法の単独性能について
\item それぞれの手法をfiltering手法として用いた場合について
\end{itemize}
\section{データセット}
DUD-E\r{ref}を利用した、だけではなく、inputとoutputを明確にする。
\begin{table}[htb] \centering
	\caption{DUD-E diverse subsetの詳細}
	\label{tb:dude_divset}
	\begin{tabular}{c|c|ll|ll|}
	\multirow{2}{*}{ターゲット名}	&\multirow{2}{*}{タンパク質名}	&\multicolumn{2}{c}{正例}	&\multicolumn{2}{c}{負例}	\\
							&							&化合物数	&平均分割数	&化合物数	&平均分割数	\\ \hline
	akt1						&							&			&			&			&			\\
	ampc					&							&			&			&			&			\\
	cp3a4					&							&			&			&			&			\\
	cxcr4					&							&			&			&			&			\\
	gcr						&							&			&			&			&			\\
	hivpr					&							&			&			&			&			\\
	hivrt						&							&			&			&			&			\\
	kif11						&							&			&			&			&			\\ \hline
	\end{tabular}
\end{table}
\section{計算環境}
TSUBAME Thinノード
\section{実験結果}
\subsection{ドッキングにかかる計算時間}
\subsection{予測精度}
ここでは単独手法としての精度を示す。ROC曲線はAppendixとして載せていることを記述

\begin{table}[htb] \centering
	\caption{フィルタリング手法の予測精度}
	\label{tb:filtering_accuracy}
	\begin{tabular}{c|c|lllll|}
	\multirow{2}{*}{手法}				&フラグメント		&\multirow{2}{*}{ROC-AUC}	&\multicolumn{4}{c}{Enrichment Factor}	\\
									&ドッキング		&						&EF(1\%)	&EF(2\%)	&EF(5\%)	&EF(10\%)	\\ \hline
	\multirow{2}{*}{総和(score\_sum)}		&glide 通常モード	&						&		&		&		&			\\
									&glide 高速モード	&						&		&		&		&			\\
	\multirow{2}{*}{最良値(score\_max)}	&glide 通常モード	&						&		&		&		&			\\
									&glide 高速モード	&						&		&		&		&			\\
	\multirow{2}{*}{線形和(maxsumBS)}	&glide 通常モード	&						&		&		&		&			\\
									&glide 高速モード	&						&		&		&		&			\\ \hline
	\multicolumn{2}{c|}{従来手法(glide 高速モード)}			&						&		&		&		&			\\ \hline
	\end{tabular}
\end{table}

\subsection{filteringと通常のドッキングとを合わせた場合の計算時間および精度}
filtering時に何\%の化合物を残すか、などの議論と合わせて、精度と速度のバランスを示す。

\begin{figure}[htp]
 \begin{center}
  \fig[width=0.99\hsize]{./fig/result/性能vs時間.eps}
  \caption{計算時間と精度のトレードオフ}
  \label{fig:trade_off}
 \end{center}
\end{figure}
