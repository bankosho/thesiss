\chapter{化合物の部分構造を利用したフィルタリング手法の開発}


\section{提案手法へのアイデア}
\subsection{ドッキング計算における探索自由度の削減}
\subsection{化合物のスコア算出}

\section{部分構造への分割}

\begin{itemize}
\item 分割手法に関しては、小峰のSIGBIOの論文を\r{ref}する。
\item ここの時点で小さい部分構造は除去する、ということを述べる?\r{要検討、score\_maxの評価をどうするかに依存。}
\item 分割について、実例を図で示す。
\item 分割することで発生する、共通部分の存在について述べる。
\end{itemize}


\section{部分構造単位のドッキング}
glide SPを用いることを記述。
\section{部分構造スコアの統合}
\subsection{部分構造スコアの総和}
score\_sumについて記述。
\subsection{部分構造スコアの最良値}
score\_maxについて記述。
\subsection{総和法と最良値法の線形和}
maxsumBSについて記述。