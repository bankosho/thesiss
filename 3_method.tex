\chapter{提案手法:化合物の部分構造を利用したフィルタリング手法の開発}
ここでは、従来手法であるGlide HTVSとは異なり、化合物を部分構造に分割することで高速にドッキングを完了させるフィルタリング手法の内容を説明する。

\section{提案手法の概説}
前章で述べた通り、ドッキングシミュレーションは時間を要し、その理由は探索空間の広さと化合物の多様性にある。
この節では、この2つの問題を解決するアイデア、および高速にフィルタリングを行うために追加する仮定を説明する。

\subsection{フィルタリングの要件}\label{subsec}
フィルタリングに求められる要件は2つ存在する。
\begin{itemize}
\item 高速に化合物を評価する\\
	フィルタリングを実用的に行うためには、フィルタリング後に行うドッキングシミュレーションよりも十分に高速である必要がある。
\item 予測の精度がある程度保持されている\\
	一般に計算速度と予測精度はトレードオフの関係にあるが、どれほど高速であってもある程度正例と負例が弁別できなければ
	フィルタリングとして機能しない。したがって、予測精度がある程度保持されていることもフィルタリングには求められる。
\end{itemize}
一方、フィルタリングはその後に通常のドッキングシミュレーションを行うことを前提とするため、
必ずしも「化合物がタンパク質のこの部分に結合する」というドッキングポーズを出力する必要はない。

\subsection{提案手法へのアイデア}\label{subsec:idea}
前節で示したフィルタリングの要件を満たすために、2つのアイデアを考案した。

\subsubsection{化合物を部分構造に分割し、部分構造のドッキングシミュレーションを行う}
ドッキングシミュレーションの探索空間のうち、並進運動と回転運動による探索空間の次元は合計6次元と固定されている。
一方、化合物の内部自由度は化合物の回転可能な結合の数によって変化し、DUD-Eデータセットに登録されている化合物の平均は
\r{x.xx}と無視できないほど大きな値である。
そこで、本提案手法では、小峰ら\citetodo{}による化合物の分割方法を用いて、化合物を内部自由度を考慮しなくて良い
「フラグメント」に分割、これらをドッキングすることで、必要最低限の探索空間でのドッキングシミュレーションを実現する。
%一般に、あるタンパク質へ結合する化合物と似た構造を持つ化合物はそのタンパク質に結合する確率が通常よりも高い。
%\citetodo{DUD-Eの論文?similarityとactivenessの関係が言えていればOK}
%したがって、部分構造レベルで

\subsubsection{フラグメントから化合物の構造を再構成せず、フラグメントの結合スコアから化合物のフィルタリングスコアを算出する}
化合物をフラグメントに分割した上でドッキングシミュレーションを行うと図\ref{fig:divided_fragment}\todo{図の作成}のように
フラグメントごとにタンパク質との結合予測構造が出力され、フラグメントの結合スコアが最も良いポーズを選択したとしても
繋がった一つの化合物としては有り得ない構造をとる場合がほとんどである。
しかし、矛盾のない化合物の構造をとるようなフラグメントの選択を行うのは$O(a^n) (nは化合物を構成するフラグメント数)$の計算量となり、
大きな計算コストを要してしまう。

一方、フィルタリングはその後に通常のドッキングシミュレーションを行うことを前提とするため、
必ずしも化合物とタンパク質との結合予測構造を出力する必要はない。
そこで、提案手法では構造の矛盾の考慮を行わず、得られたフラグメントの結合スコアのみに着目し、
フラグメントの結合スコアから化合物のフィルタリングスコアを算出するのに計算が$O(n)$で済むようなスコアの統合を行うことで、
高速な化合物の評価を達成する。


\begin{figure}[htb]
 \begin{center}
  \fig[width=0.4\hsize]{./fig/method/スコア統合イメージ.eps}
  \caption{スコア統合イメージ(暫定版)}
  \label{fig:integration_image}
 \end{center}
\end{figure}

\section{提案手法の詳細の説明}
前節で用いる2つのアイデアを示したが、それを用いてどのようにフィルタリングを実現しているのかをこの節で詳説する。

\subsection{提案手法のフローチャート}\label{subsec:flowchart}
提案手法は以下の手順で構成される。
\begin{enumerate}
\item 入力された化合物をフラグメントに分割する
\item ドッキングシミュレーションツールを用いてフラグメントの標的タンパク質への結合スコアを算出する
\item フラグメントの結合スコアから化合物のフィルタリングスコアを算出する
\item フィルタリングスコアの上位N\%をフィルタを通過した化合物として出力する
\end{enumerate}
フローチャートを図\ref{fig:flowchart}に示す。

\subsection{化合物の分割}
化合物の分割は小峰らによる手法\citetodo{}を用い、フラグメントを生成する。
実装にはC++を用い、ケモインフォマティクスツールであるOpenBabel\cite{OBoyle2011}, 並列化のためのOpenMPを利用している。
アルゴリズムを以下に示す。
\begin{enumerate} 
\item 
\end{enumerate}
この手法による分割例を図\ref{fig:ex_decomposition}に示す。
\begin{figure}[htp]
 \begin{center}
  \fig[width=0.6\hsize]{./fig/method/フラグメント分割例.eps}
  \caption{化合物のフラグメント分割例}
  \label{fig:ex_decomposition}
 \end{center}
\end{figure}
この化合物のフラグメントへの分割により、化合物の内部自由度を考慮することなくドッキングシミュレーションを行うことができる。

また、化合物は似た構造を持つことが非常に多く、

\begin{itemize}
\item 分割は基本的に小峰による手法を用いることを示す\citetodo{}
\item 重原子数が2以下のフラグメントは除外することを示す。
	\memo{除外する理由はどこで書くべき?精度評価まで行かないと理由がない。現在は考察で記述する予定}
\item 図\ref{fig:ex_decomposition}のような具体例を示し、分割がどのように行われるのか明確にする。
	必ず「内部自由度を考慮しないドッキングでOK」であることに言及する。
\item また、この分割によって多数の化合物で共通部分が発生することも示す。共通部分が発生することも図で示す\todo{図の作成}\\
	これについては、1000万化合物に対して行った実験(図\ref{fig:decomposition_amount})を説明、結果を示す。
	\begin{figure}[htp]
	 \begin{center}
	  \fig[width=0.6\hsize]{./fig/method/大量decomposition.eps}
	  \caption{大量の化合物を分割した場合の例}
	  \label{fig:decomposition_amount}
	 \end{center}
	\end{figure}

\end{itemize}

\subsection{部分構造単位でのドッキング}
\begin{itemize}
\item 以下の二通りのドッキングで実験を行うことを述べる。
	\begin{itemize}
	\item glide の通常ドッキングモード(SP) & 内部自由度を無視するオプション
	\item glide の高速ドッキングモード(HTVS)
	\end{itemize}
\item ドッキング結果として複数の構造が出力される場合があるが、そのときは最良のスコアを取得することを示す(図\ref{fig:fragment_result}のような)。
	\begin{figure}[htp]
	 \begin{center}
	  \fig[width=0.6\hsize]{./fig/method/フラグメントスコア算出方法.jpg}
	  \caption{部分構造スコアの取得}
	  \label{fig:fragment_result}
	 \end{center}
	\end{figure}
\end{itemize}
以下の二通りのドッキングを行う。

\subsection{部分構造のスコアの統合}
最初に、フラグメントごとのドッキング結果は構造を保持しないことを図で示し、したがってポーズを出力しないことを明記する。
\todo{図\ref{fig:fragment_docking_result}のようなものを示す。}(なぜか図\ref{fig:fragment_docking_result}が表示されないが、図\ref{fig:fragment_result}と同様に手書きなので無視。)
\begin{figure}[htp]
 \begin{center}
  \fig[width=0.6\hsize]{./fig/method/ドッキング結果構造.png}
  \caption{統合結果の構造}
  \label{fig:fragment_docking_result}
 \end{center}
\end{figure}

ここでは3種類のスコア統合手法について述べる。
\subsubsection{部分構造スコアの総和}
score\_sumについて記述。
\subsubsection{部分構造スコアの最良値}
score\_maxについて記述。
\subsubsection{総和法と最良値法の線形和}
maxsumBSについて記述。

