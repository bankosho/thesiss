\chapter{化合物の部分構造を利用したフィルタリング手法の開発}


\section{提案手法へのアイデア}
\subsection{ドッキング計算における探索自由度の削減}
\begin{itemize}
\item 化合物の内部自由度を図で示す。
\item 探索自由度の削減手法を示す。 図\ref{fig:docking_freedom}的な。
\begin{figure}[htb]
 \begin{center}
  \fig[width=0.6\hsize]{./fig/method/ドッキングの探索自由度.eps}
  \caption{探索自由度について}
  \label{fig:docking_freedom}
 \end{center}
\end{figure}

\end{itemize}

\subsection{化合物のスコア算出}

\begin{figure}[htb]
 \begin{center}
  \fig[width=0.4\hsize]{./fig/method/スコア統合イメージ.eps}
  \caption{スコア統合イメージ}
  \label{fig:integration_image}
 \end{center}
\end{figure}

\section{部分構造への分割}

\begin{itemize}
\item 分割手法に関しては、小峰のSIGBIOの論文を参照する。\citetodo{}
\item ここの時点で小さい部分構造は除去する、ということを述べる?\todo{要検討、score\_maxの評価をどうするかに依存。}\\
たぶんすることになると思う。その場合は分割の実例に除去の例を載せる。
\item 分割について、実例を図で示す。

\begin{figure}[htp]
 \begin{center}
  \fig[width=0.6\hsize]{./fig/method/フラグメント分割例.eps}
  \caption{フラグメント分割例}
  \label{fig:ex_decomposition}
 \end{center}
\end{figure}

\item 分割することで発生する、共通部分の存在について述べる。\todo{これも図で示す。}
\item 大量の化合物を分割した場合に図\ref{fig:decomposition_amount}のようになる、ということを示す。データセットがここで突然出てきてしまうが・・・。

\begin{figure}[htp]
 \begin{center}
  \fig[width=0.6\hsize]{./fig/method/大量decomposition.eps}
  \caption{大量の化合物を分割した場合の例}
  \label{fig:decomposition_amount}
 \end{center}
\end{figure}

\end{itemize}

\section{部分構造単位のドッキング}
glide 通常モード(SP)、およびglide 高速モード(HTVS)を用いることを記述。
\todo{部分構造スコアはベストをとることを図示する。}

\section{部分構造スコアの統合}
最初に、フラグメントごとのドッキング結果は構造を保持しないことを図で示す。結果の例を見せれば伝わるはず。
\subsection{部分構造スコアの総和}
score\_sumについて記述。
numHvyAtom $<=$ 2 のフラグメントを除去している。
\subsection{部分構造スコアの最良値}
score\_maxについて記述。
numHvyAtom $<=$ 2 のフラグメントを除去していない。\todo{除去しても結果はほとんど変わらないはず?}
\subsection{総和法と最良値法の線形和}
maxsumBSについて記述。
numHvyAtom $<=$ 2 のフラグメントを除去している。
