\chapter{序論}
\section{研究背景}

\begin{itemize}
\item computer-aided drug discoveryでは、SBDD, LBDD, CGBVSの3種類の薬剤候補化合物の選別手法が存在している
\item このうち、SBDDは演繹的な手法であり、タンパク質の構造が得られれば阻害剤が存在しなくとも薬剤開発が可能であり、非常に有用。\\
	もし阻害剤が知られているターゲットだとしても構造が既知の阻害剤とは大きく異なる薬剤候補を見つけられる。\\
	これはLBDDやCGBVSにはないメリット\todo{CGBVSにはこのメリットはないのか?どんなメリットが主張されているのか?調査が必要。}\\
	\memo{この時点でPharmacophoreの手法は除外している}
\item SBDDでは化合物-タンパク質ドッキングというシミュレーション手法を用いて化合物を評価する。
\item 様々な研究がすすんでおり、Glide, Autodock等といったさまざまなツールが開発されている。
\item その中でも、Glideというドッキングツールが良い精度を出すことが知られている\citetodo{比較論文ref}
\item このドッキングツールは、行う計算の内容の関係上、計算コストが高い。
\item ドッキング計算手法の高速化研究\citetodo{GPU実装, Autodock Vinaなど}は行われているが、不十分である。
\item そのため、ドッキングツールを用いて化合物として購入可能な数千万以上の化合物(ZINCの件数を利用)を一斉に評価することは難しい。
\item したがって、フィルタリングの必要性がある。
\item しかしフィルタリングは既知の化合物に基づいた手法が殆どであり、SBDDの長所である新規の構造を持つ薬剤候補の発見能力を奪うことになる。
\item Glideが高速ドッキングモードを提供しているが、数千万化合物の単位ではまだ計算コストが大きい。
\end{itemize}

\section{研究目的}
\begin{itemize}
\item 研究背景から、より高速に、新規の構造を持つ薬剤候補をフィルタリングする必要がある
\item そこで、ドッキングに基づいた、フィルタリングに特化した手法をこの研究では提案する
\end{itemize}

\section{本論文の構成}
2章では提案手法のベースとなるドッキングシミュレーションについて詳しく説明、3章で提案するフィルタリング手法の説明を行う。
4章で実験について述べ、5章ではこの実験の結果についての考察を加える。
最後に6章で結論および今後の展望について述べる。