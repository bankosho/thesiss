\chapter{序論}
\section{背景}
computer-aided drug discoveryについて、G. Sliwoski, et al., 2014\r{ref}の論文の章立てに触れながら説明。
SBDD, LBDD, ADMETの予測、という三種類の立場で計算手法が役立っているということを話せればOK。
ここにVSについての説明も入るだろう。

創薬にかかる時間やお金の具体数は記述しない予定

\section{研究目的}
SBDDでは計算時間がかかるということに言及。
docking計算手法の高速化が一般的に行われている一方、フィルタリングを行うということもされている。

フィルタリングに関しては基本的に構造を見ていない、ligand-basedな手法\r{ref}。glide HTVS\r{ref:HTVSって論文ある?}くらいしか構造を見ていない。
glide HTVSに関しても、速度を高めるために強い仮定を置いているため、精度が十分とはいえない。
したがって、より高速に、より精度よくフィルタリングを行うことが非常に大切となる。

\section{本論文の構成}
2章ではSBDDによる創薬について詳しく説明、3章で提案するフィルタリング手法の説明を行う。
4章で実験について述べ、5章ではこの実験の結果についての考察を加える。
最後に6章で結論および今後の展望について述べる。