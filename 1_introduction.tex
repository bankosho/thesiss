\chapter{序論}
\section{研究背景}\label{sec:background}
近年、創薬の初期段階において、コンピュータによる予測を通して大量の化合物から薬剤候補化合物を
選別するバーチャルスクリーニング(Virtual Screening, VS)と呼ばれる手法が用いられ、
創薬コストの削減および創薬にかかる時間の短縮が試みられている。
このコンピュータを用いた化合物の選別手法は大きく3つに分けられる。

\begin{enumerate}
\item タンパク質や化合物の立体構造を用いた手法(Structure-Based Virtual Screening, SBVS)
	\begin{itemize}
	\item タンパク質-化合物ドッキングシミュレーション\cite{Friesner2004, Zsoldos2007, Morris2009}
	\end{itemize}
\item 既知の薬剤・タンパク質の活動を阻害する化合物(阻害剤)の情報を用いた手法(Ligand-Based Virtual Screening, LBVS)
	\begin{itemize}
	\item 構造活性相関(Quantitative Structure-Activity Relationship, QSAR)を用いた手法\cite{Hansch1964}
	\item 機械学習による分類手法\cite{Ivanciuc2007}
	\item 化合物の官能基の性質を用いたファーマコフォアモデルに基づく化合物分類手法\cite{Wolber2008}
	\end{itemize}
\item タンパク質と薬剤との2部グラフなどのネットワークを構築し、類似度から予測を行う創薬手法 
	(Chemical Genomics-Based Virtual Screening, CGBVS)\cite{Brown2012}
\end{enumerate}

このうち、タンパク質-化合物ドッキングシミュレーションによるSBVSは物理的なエネルギーを計算する演繹的な手法であり、
既知の薬剤や阻害剤が存在しない創薬標的であってもタンパク質の構造が得られれば薬物候補化合物を選別することができる、
非常に有用な方法である。また、既知の薬剤や阻害剤から法則性を見つけ出すなど帰納的な手法であるLBVS等に比べて
既知の薬剤や阻害剤と大きく性質の異なる、「新規の構造を持った」薬剤候補化合物を発見する能力が高いことも
ドッキングシミュレーションによるSBVSのメリットである。

ドッキングシミュレーションはGlide\cite{Friesner2004}, eHiTS\cite{Zsoldos2007}, Autodock\cite{Morris2009}を始めとして
多様なツールが開発されており、その中でもGlideは予測精度が高く\cite{Kruger2010}、広く利用されている\cite{Yuriev2013}。

一方、ドッキングシミュレーションはタンパク質と化合物との複合体構造の予測を行うためには
化合物の回転や平行移動を行いながら探索を行う最適化問題が必要となり、
計算コストが非常に高いという問題点が存在する。これを解決するためにドッキング手法の高速化の研究\cite{Kannan2010, McIntosh-Smith2014, Trott2010}
が行われているが、速度の点で未だ不十分である。
例えば、購入可能な化合物の立体構造データベースを公開しているZINC
\cite{Irwin2005}に存在する22,724,825件の化合物を一斉にドッキングシミュレーションをしようとすると、現状では12コアの計算機で半年以上を要する。

%以上の理由から、SBVSをもちいた創薬研究ではドッキングシミュレーションを行う前に化合物を選別するフィルタリングが行われることが多い。
%しかし、表\ref{table:filtering_problems}に示すような様々な問題があり、手法が十分に洗練されているとは言い難いのが現状である。
%
%\begin{table}[htb] \centering
%	\caption{フィルタリング手法とその問題点}
%	\label{table:filtering_problems}
%	\begin{tabular}{p{2cm}p{3cm}p{6cm}}
%	\hline
%	\multicolumn{1}{c}{手法の種類}								&\multicolumn{1}{c}{手法利用例}										&問題点 \\ \hline
%	\multirow{2}{*}{既知の薬剤情報に基づいた手法}			&化合物のfingerprintを利用したフィルタリング\cite{Nilakantan1993}	&\multirow{2}{*}{SBVSの長所である「既知の薬剤や阻害剤と大きく性質の異なる薬剤候補化合物」をフィルタリングで落としてしまうことが多く、ドッキングシミュレーションと相性が悪い} \\
%																		&ファーマコフォアを利用したフィルタリング\cite{Parenti2003}			&	\\
%	\multirow{2}{*}{ドッキングシミュレーションに基づいた手法}	&Glideの簡易ドッキングモードである										&\multirow{2}{*}{数千万個の化合物の評価を行うには計算速度が不十分であり、商用ソフトであるためにTSUBAME 2.5などのスーパーコンピュータの大規模利用による高速化が不可能} \\ \hline
%	\end{tabular}
%\end{table}

以上の理由から、SBVSを用いた創薬研究ではドッキングシミュレーションを行う前に化合物を選別する
フィルタリングが行われることが多い\cite{Nilakantan1993, Parenti2003}。しかし、これらのフィルタリング手法の多くはLBVSのように、
既知の薬剤などの化合物情報を用いるものであり、前述したSBVSの長所である「既知の薬剤や阻害剤と大きく性質の異なる
薬剤候補化合物」をフィルタリングで落としてしまうことが多く、SBVSとは相性が悪いという問題がある。
また、Glideの簡易ドッキングモードであるHTVSモードを用いてフィルタリングを行うこともあり
\cite{Fujimoto2008}、この手法を用いればSBVSの長所を損なうことなくフィルタリングを行うことが
できるが、前述したような数千万単位の化合物数ではGlide HTVSモードですら計算量が膨大になってしまう。
また、Glideは計算に利用するコア数に応じてライセンスを購入しなければならない形式の商用ソフトであり、
TSUBAME2.5などのスーパーコンピュータの大規模利用による高速化を行うことができない。

\comment{このあたりの文章を図表にまとめられると良い}

\section{研究目的}
\ref{sec:background}節で示したように、SBVSにおけるフィルタリングは未だ研究が不十分であり、新規の構造を持つ化合物を残す
高速なフィルタリング手法を開発する必要がある。
本論文では、ドッキングに基づいた高速なフィルタリング手法を提案する。

\section{本論文の構成}
第2章では、ドッキングシミュレーションに基づいたSBVSについての説明を行い、同時に既存のフィルタリング手法について説明する。
第3章では提案手法について述べ、第4章でこの提案手法と簡易ドッキングであるGlide HTVSモードとの比較を行う。
また、第5章では第4章で行った実験の結果についての考察を加え、第6章で結論および今後の展望を述べる。