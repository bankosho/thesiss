\chapter{序論}
\section{研究背景}\label{sec:background}
近年、創薬の初期段階においてバーチャルスクリーニング (Virtual Screening, VS) と呼ばれる、コンピュータによる予測を用いて大量の化合物から
薬剤候補化合物を選別する手法を用いることで創薬コストの削減、および創薬にかかる時間の短縮が試みられている。
このコンピュータを用いた化合物の選別手法は大きく3つに分けられる。

\begin{itemize}
\item タンパク質や化合物の立体構造を用いた手法 (Structure-Based Virtual Screening, SBVS)
	\begin{itemize}
	\item タンパク質-化合物ドッキングシミュレーション\citetodo{Glide, eHiTS, Autodock}
	\end{itemize}
\item 既知の薬剤・タンパク質の活動を阻害する化合物(阻害剤)の情報を用いた手法(Ligand-Based Virtual Screening, LBVS)
	\begin{itemize}
	\item 構造活性相関 (Quantitative Structure-Activity Relationship, QSAR)を用いた手法\citetodo{}
	\item 機械学習による分類手法\citetodo{}
	\item 化合物の官能基の性質を用いたファーマコフォアモデルに基づく化合物分類手法\citetodo{}
	\end{itemize}
\item タンパク質と薬剤との2部グラフなどのネットワークを構築し、類似度から予測を行う創薬手法 
	(Chemical Genomics-Based Virtual Screening, CGBVS)\citetodo{Brown \& Okuno (2012)}
\end{itemize}

このうち、タンパク質-化合物ドッキングシミュレーションによるSBVSは物理的なエネルギーを計算する演繹的な手法であり、
既知の薬剤や阻害剤が存在しない創薬標的であってもタンパク質の構造得られれば薬物候補化合物を選別することができる、
非常に有用な方法である。また、既知の薬剤や阻害剤から法則性を見つけ出すなど帰納的な手法であるLB等に比べて
既知の薬剤や阻害剤と大きく性質の異なる、「新規の構造を持った」薬剤候補化合物を発見する能力が高いことも
ドッキングシミュレーションによるSBVSのメリットである。
ドッキングシミュレーションはGlide\citetodo{}, eHiTS\citetodo{}, Autodock\citetodo{}を始めとして多様なツールが開発されており、
その中でもGlideは予測精度が高く\citetodo{比較論文}、比較的広く利用されている。
しかし、ドッキングシミュレーションはタンパク質と化合物との結合構造という非常に複雑な探索空間の中での最適化問題を解くため、
計算コストが非常に高い。これを解決するためにドッキング手法の高速化の研究\citetodo{AutodockのGPU実装, BUDE, Autodock Vina}
が行われているが、速度的、もしくは精度的に未だ不十分であり購入可能な化合物の立体構造データベースを公開しているZINC\citetodo{}に
存在する\r{xxx件}の化合物を一斉にドッキングシミュレーションで予測することは困難というのが現状である。

以上の理由から、SBVSを用いた創薬研究ではドッキングシミュレーションを行う前に化合物を選別する、
フィルタリングが行われることが多い\citetodo{実例を示す}。しかし、このフィルタリング手法の多くはLBVSのように、
既知の薬剤などの化合物情報を用いるものであり、前述したSBVSの長所である「既知の薬剤や阻害剤と大きく性質の異なる
薬剤候補化合物」をフィルタリングで落としてしまうことが多く、ドッキングシミュレーションとは相性が悪い。
また、Glideの簡易ドッキングモードであるHTVSモードを用いてフィルタリングを行うこともあり
\citetodo{Glide HTVSをフィルタリング手法に用いている論文}、この手法を用いればSBVSの長所を損なうことなくフィルタリングを行うことが
できるが、前述したような数千万単位の化合物数ではGlide HTVSモードですら計算量が膨大になってしまう。
また、Glideは計算に利用するコア数に応じてライセンスを購入しなければならない形式の商用ソフトであり、
TSUBAME2.5などのスーパーコンピュータの大規模利用による高速化を行うことができない。

\section{研究目的}
\ref{sec:background}節で示したように、SBVSにおけるフィルタリングは未だ研究が不十分であり、高速に、新規の構造を持つ、
見込みのある化合物を残すフィルタリング手法を開発する必要がある。
本論文では、ドッキングに基づいた、フィルタリングに特化した手法を提案し、ドッキングに基づいたフィルタリングの既存手法である
Glide HTVSと比較、提案手法の有用性を述べる。

\section{本論文の構成}
2章では、ドッキングシミュレーションに基づいたSBVSについての説明を行い、同時に既存のフィルタリング手法について説明する。
3章では提案手法について述べ、4章でこの提案手法と既存手法であるGlide HTVSとの比較を行う。
また、5章では4章で行った実験の結果についての考察を加え、6章で結論および今後の展望を述べる。


% 1/2にコメント化
%\begin{itemize}
%\item computer-aided drug discoveryでは、SBDD, LBDD, CGBVSの3種類の薬剤候補化合物の選別手法が存在している
%\item このうち、SBDDは演繹的な手法であり、タンパク質の構造が得られれば阻害剤が存在しなくとも薬剤開発が可能であり、非常に有用。\\
%	もし阻害剤が知られているターゲットだとしても構造が既知の阻害剤とは大きく異なる薬剤候補を見つけられる。\\
%	これはLBDDやCGBVSにはないメリット\todo{CGBVSにはこのメリットはないのか?どんなメリットが主張されているのか?調査が必要。}\\
%	\memo{この時点でPharmacophoreの手法は除外している}
%\item SBDDでは化合物-タンパク質ドッキングというシミュレーション手法を用いて化合物を評価する。
%\item 様々な研究がすすんでおり、Glide, eHiTS, Autodock等といったさまざまなツールが開発されている。
%\item その中でも、Glideというドッキングツールが良い精度を出すことが知られている\citetodo{比較論文ref}
%\item このドッキングツールは、行う計算の内容の関係上、計算コストが高い。
%\item ドッキング計算手法の高速化研究\citetodo{GPU実装, Autodock Vinaなど}は行われているが、不十分である。
%\item そのため、ドッキングツールを用いて化合物として購入可能な数千万以上の化合物(ZINCの件数を利用)を一斉に評価することは難しい。
%\item したがって、フィルタリングの必要性がある。
%\item しかしフィルタリングは既知の化合物に基づいた手法が殆どであり、SBDDの長所である新規の構造を持つ薬剤候補の発見能力を奪うことになる。
%\item Glideが高速ドッキングモードを提供しているが、数千万化合物の単位ではまだ計算コストが大きい。
%\end{itemize}
%
%\section{研究目的}
%\begin{itemize}
%\item 研究背景から、より高速に、新規の構造を持つ薬剤候補をフィルタリングする必要がある
%\item そこで、ドッキングに基づいた、フィルタリングに特化した手法をこの研究では提案する
%\end{itemize}
%
%\section{本論文の構成}
%2章では提案手法のベースとなるドッキングシミュレーションについて詳しく説明、3章で提案するフィルタリング手法の説明を行う。
%4章で実験について述べ、5章ではこの実験の結果についての考察を加える。
%最後に6章で結論および今後の展望について述べる。