% 卒業論文 main.tex

\documentclass[11pt]{jreport}
\bibliographystyle{junsrt} % for bibtex
\bibdata{ref.bib}

\usepackage{pifont}
\usepackage{bachelor}
\usepackage{tabularx}
\usepackage{float}
\usepackage{tascmac}
\usepackage{amsmath,amssymb}
\usepackage{bm}
\usepackage{eclbkbox}
\usepackage{multirow}
\usepackage{lscape}
\usepackage{url}

\usepackage{overcite} %Use overcite package (T.Arikuma)
\renewcommand\citeform[1]{#1)} %Use overcite package (T.Arikuma)

%表の線太く
\newcolumntype{I}{!{\vrule width 0.8pt}}
\newcolumntype{J}{!{\vrule width 0.7pt}}

\def\hthick{\noalign{\hrule height 1pt}} %macro for table thick hline (T.Arikuma)
%\usepackage[dvips]{color,graphicx} % for eps files
\usepackage[dvipdfmx]{color,graphicx}%png-bbファイル対応
\usepackage{colortbl}
\usepackage{fancybox}
%\usepackage[dvips, dvipdfm]{color,graphicx} % for eps and png files
%\usepackage[dvipdfm]{graphicx}

%\usepackage{amssymb} % \mathfrak
%\usepackage{comment} % comment
\setlength{\oddsidemargin}{-15pt} %左側の余白
\setlength{\textwidth}{490pt}     % 本文テキスト全体の幅
\setlength{\topmargin}{0pt}       % 上側の余白
\setlength{\textheight}{650pt}    % 本文テキスト全体の高さ

%\usepackage{ascmac} 

%\pagestyle{empty}

\DeclareMathOperator*{\argmax}{argmax}
\DeclareMathOperator*{\argmin}{argmin}
%\newcolumntype{C}{>{\centering\arraybackslash}X}

%ハイフネーションをガンガン許す
\hyphenpenalty=0\relax
\exhyphenpenalty=0\relax

%%%独自%%%%%

\def\todo#1{\textcolor{red}{\bf @todo #1}}
\def\memo#1{\textcolor{blue}{{\small @memo #1}}}
\def\citetodo#1{\textcolor{green}{{\small @cite #1}}}
\def\comment#1{\textcolor{blue}{{@comment #1}}}

%\def\r#1{\textcolor{red}{\textbf{#1}}}
\def\b#1{\textbf{#1}}

\def\fig{\includegraphics}

%%%%%%    TEXT START    %%%%%%
\begin{document}

%============[ 表紙 ]=============================================================================

\年月{平成28年1月}
\題名A{フラグメント分割に基づく高速な}
\題名B{化合物プレドッキング手法の開発}
\指導教員名{秋山 泰}
\職名{教授}
\研究科{情報理工学研究科}
\専攻{計算工学専攻}{}
\学籍番号{14M38400}
\氏名{柳澤 渓甫}

\newpage
\clearpage

\maketitle
\clearpage
 \thispagestyle{empty}
 \\
\clearpage

%============[ 内容梗概 ]=============================================================================
{
 \thispagestyle{empty}
 \large
 \noindent
 平成27年度 修士論文内容梗概
 \\
 \begin{center}
  \textbf{\LARGE フラグメント分割に基づく\\高速な化合物プレドッキング手法の開発}
 \end{center}
 \\
 \hfill 
 \begin{tabular}{llll}
 指導教員& \multicolumn{3}{l}{秋山 泰 教授} \\
  \multicolumn{4}{l}{計算工学専攻} \\
 14M38400\hspace{.5cm} & {柳澤 渓甫}
 \end{tabular}

\mbox{}\\

創薬では、大量の候補化合物から薬になりうる化合物を選別する作業が日常的に行われている。
このうちコンピュータによる予測を利用してコスト削減および開発期間の短縮を目指す「バーチャルスクリーニング」と
呼ばれる手法が近年盛んに用いられている。中でも、タンパク質や化合物の立体構造情報を用いた
ドッキングシミュレーション(以下ドッキング)を行うSBVS(Structure-Based Virtual Screening)は
標的タンパク質に対する既知の薬剤情報がなくても適用できる手法であり、創薬の現場で大きく期待されている。

一方で、ドッキングは標的タンパク質と化合物との最適な複合体構造の探索を必要とするが、
相対的な3次元位置、回転および、化合物内部の結合の回転という多くの探索パラメータが存在するため計算コストが高く、
大量の化合物すべてを評価することは困難な場合が多い。
この問題を解決するため、化合物を何らかの方法でフィルタリングし、数を減らした化合物サブセットについて
ドッキングを行う、という段階的な手法が多く実践されている。
しかし従来のフィルタリング手法は既知の薬剤情報を用いるため、
SBVSと組み合わせた時にSBVSの利点が生かされないという問題が存在していた。

そこで本研究では、ドッキングに基づいた上で、化合物の部分構造を用いることでより高速にフィルタリングを行う
手法(プレドッキング)を提案する。提案手法では、まず全ての候補化合物をフラグメントと呼ばれる内部に回転可能な結合を持たない
部分構造に分割し、標的タンパク質とフラグメントとの間でドッキングを行う。このとき、フラグメントへの分割により回転自由度
がなくなるため、ドッキングの最適化問題の探索パラメータが減少し、計算の高速化が可能となる。
さらに、フラグメントのスコアから化合物のスコアを高速に計算する方法を検討することで、
高速だが粗い探索を行うドッキングであるGlide HTVSモードに比べ、提案手法は一般的なベンチマークデータセット
(DUD-E)を用いたときに精度は劣るものの平均して約9倍の高速化を達成した。
フラグメントごとのドッキング結果は化合物間で再利用が可能であるため、高速化率は評価対象の化合物数が多いほど高くなる。

また、現実の用途に基づいてフィルタリング後にドッキングを行った場合の速度や精度を評価することは重要であるので、
ドッキング計算も含めたフィルタリング手法の評価を行った。その結果、提案手法はGlide HTVSモードに比べて
精度が最大約25\%、速度が最大約40\%向上させるケースが存在することが示された。

\thispagestyle{empty}
\addtocounter{page}{-1}
}

%\clearpage
\newpage

%============[ 目次 ]=============================================================================
\pagenumbering{roman}
\tableofcontents
\listoffigures
\listoftables
\pagebreak

%============[ 本文 ]==============================================================================

\pagenumbering{arabic}

\chapter{序論}
\section{研究背景}\label{sec:background}
近年,創薬の初期段階において,コンピュータによる予測を通して大量の化合物から薬剤候補化合物を
選別するバーチャルスクリーニング(Virtual Screening, VS)と呼ばれる手法が用いられ,
創薬コストの削減および創薬にかかる時間の短縮が試みられている.
このコンピュータを用いた化合物の選別手法は大きく3つに分けられる.

\begin{enumerate}
\item タンパク質や化合物の立体構造を用いた手法(Structure-Based Virtual Screening, SBVS)
	\begin{itemize}
	\item タンパク質-化合物ドッキングシミュレーション\cite{Friesner2004, Zsoldos2007, Morris2009}
	\end{itemize}
\item 既知の薬剤・タンパク質の活動を阻害する化合物(阻害剤)の情報を用いた手法(Ligand-Based Virtual Screening, LBVS)
	\begin{itemize}
	\item 構造活性相関(Quantitative Structure-Activity Relationship, QSAR)を用いた手法\cite{Hansch1964}
	\item 機械学習による分類手法\cite{Ivanciuc2007}
	\item 化合物の官能基の性質を用いたファーマコフォアモデルに基づく化合物分類手法\cite{Wolber2008}
	\end{itemize}
\item タンパク質と薬剤との2部グラフなどのネットワークを構築し,類似度から予測を行う創薬手法 
	(Chemical Genomics-Based Virtual Screening, CGBVS)\cite{Brown2012}
\end{enumerate}

このうち,タンパク質-化合物ドッキングシミュレーションによるSBVSは物理的なエネルギーを計算する演繹的な手法であり,
既知の薬剤や阻害剤が存在しない創薬標的であってもタンパク質の構造が得られれば薬物候補化合物を選別することができる,
非常に有用な方法である.また,既知の薬剤や阻害剤から法則性を見つけ出すなど帰納的な手法であるLBVS等に比べて
既知の薬剤や阻害剤と大きく性質の異なる,「新規の構造を持った」薬剤候補化合物を発見する能力が高いことも
ドッキングシミュレーションによるSBVSのメリットである.

ドッキングシミュレーションはGlide\cite{Friesner2004}, eHiTS\cite{Zsoldos2007}, Autodock\cite{Morris2009}を始めとして
多様なツールが開発されており,その中でもGlideは予測精度が高く\cite{Kruger2010},広く利用されている\cite{Yuriev2013}.

一方,ドッキングシミュレーションはタンパク質と化合物との複合体構造の予測を行うためには
化合物の回転や平行移動を行いながら探索を行う最適化問題が必要となり,
計算コストが非常に高いという問題点が存在する.これを解決するためにドッキング手法の高速化の研究\cite{Kannan2010, McIntosh-Smith2014, Trott2010}
が行われているが,速度の点で未だ不十分である.
例えば,購入可能な化合物の立体構造データベースを公開しているZINC
\cite{Irwin2005}に存在する22,724,825件の化合物を一斉にドッキングシミュレーションをしようとすると,現状では12コアの計算機で半年以上を要する.

%以上の理由から,SBVSをもちいた創薬研究ではドッキングシミュレーションを行う前に化合物を選別するフィルタリングが行われることが多い.
%しかし,表\ref{table:filtering_problems}に示すような様々な問題があり,手法が十分に洗練されているとは言い難いのが現状である.
%
%\begin{table}[htb] \centering
%	\caption{フィルタリング手法とその問題点}
%	\label{table:filtering_problems}
%	\begin{tabular}{p{2cm}p{3cm}p{6cm}}
%	\hline
%	\multicolumn{1}{c}{手法の種類}								&\multicolumn{1}{c}{手法利用例}										&問題点 \\ \hline
%	\multirow{2}{*}{既知の薬剤情報に基づいた手法}			&化合物のfingerprintを利用したフィルタリング\cite{Nilakantan1993}	&\multirow{2}{*}{SBVSの長所である「既知の薬剤や阻害剤と大きく性質の異なる薬剤候補化合物」をフィルタリングで落としてしまうことが多く,ドッキングシミュレーションと相性が悪い} \\
%																		&ファーマコフォアを利用したフィルタリング\cite{Parenti2003}			&	\\
%	\multirow{2}{*}{ドッキングシミュレーションに基づいた手法}	&Glideの簡易ドッキングモードである										&\multirow{2}{*}{数千万個の化合物の評価を行うには計算速度が不十分であり,商用ソフトであるためにTSUBAME 2.5などのスーパーコンピュータの大規模利用による高速化が不可能} \\ \hline
%	\end{tabular}
%\end{table}

以上の理由から,SBVSを用いた創薬研究ではドッキングシミュレーションを行う前に化合物を選別する
フィルタリングが行われることが多い\cite{Nilakantan1993, Parenti2003}.しかし,これらのフィルタリング手法の多くはLBVSのように,
既知の薬剤などの化合物情報を用いるものであり,前述したSBVSの長所である「既知の薬剤や阻害剤と大きく性質の異なる
薬剤候補化合物」をフィルタリングで落としてしまうことが多く,SBVSとは相性が悪いという問題がある.
また,Glideの簡易ドッキングモードであるHTVSモードを用いてフィルタリングを行うこともあり
\cite{Fujimoto2008},この手法を用いればSBVSの長所を損なうことなくフィルタリングを行うことが
できるが,前述したような数千万単位の化合物数ではGlide HTVSモードですら計算量が膨大になってしまう.
また,Glideは計算に利用するコア数に応じてライセンスを購入しなければならない形式の商用ソフトであり,
TSUBAME2.5などのスーパーコンピュータの大規模利用による高速化を行うことができない.

\comment{このあたりの文章を図表にまとめられると良い}

\section{研究目的}
\ref{sec:background}節で示したように,SBVSにおけるフィルタリングは未だ研究が不十分であり,新規の構造を持つ化合物を残す
高速なフィルタリング手法を開発する必要がある.
本論文では,ドッキングに基づいた高速なフィルタリング手法を提案する.

\section{本論文の構成}
第2章では,ドッキングシミュレーションに基づいたSBVSについての説明を行い,同時に既存のフィルタリング手法について説明する.
第3章では提案手法について述べ,第4章でこの提案手法と簡易ドッキングであるGlide HTVSモードとの比較を行う.
また,第5章では第4章で行った実験の結果についての考察を加え,第6章で結論および今後の展望を述べる.
\chapter{ドッキングシミュレーションによる薬物候補化合物の選別}
この章ではドッキングシミュレーションに基づく化合物の選別手法を説明し、既存の化合物フィルタリング手法を紹介する。

\section{SBVS(Structure-Based Virtual Screening)とは}
バーチャルスクリーニング(Virtual Screening, VS)とは、コンピュータを用い、データベースに存在する化合物について、
創薬標的となっているタンパク質の活性部位への結合のしやすさを仮想的に(Virtual)評価、選別(Screening)することを指す。
化合物の評価・選別を創薬標的のタンパク質や化合物の立体構造に基づいて行う手法のことをSBVSと呼ぶ。
このSBVSは、化合物の評価・選別を既知の創薬標的タンパク質へ結合する化合物(リガンド, ligand)を用いて行う
LBVS(Ligand-Based Virtual Screening)と比べて
\begin{itemize}
\item 既知のリガンド情報を必要とせず
\item 既知のリガンドにとらわれない、多様な薬剤候補化合物を得ることができる
\end{itemize}
という長所を持っている。

\section{化合物-タンパク質ドッキングシミュレーション}
SBVSにおける化合物の評価には化合物-タンパク質ドッキングシミュレーションが一般に用いられる。
ドッキングシミュレーションは、1つのタンパク質の立体構造と1つの化合物の立体構造を入力として、化合物がタンパク質中でどのような構造をとると
エネルギー的に最も安定であるかという最適化問題を解き、最安定であると考えられる化合物の構造とその時のスコアを
出力する(図\ref{fig:docking})。
\begin{figure}[tb]
 \begin{center}
  \fig[width=0.9\hsize]{./fig/background/docking_image.eps}
  \caption{ドッキングシミュレーションのイメージ}
  \label{fig:docking}
 \end{center}
\end{figure}
この得られたスコアを直接、あるいは何らかの形で変換を行った評価値を用いて複数の化合物の選別を行う。

この化合物-タンパク質ドッキングシミュレーションを行うツールは
Glide\cite{Friesner2004}, eHiTS\cite{Zsoldos2007}などの有償ソフトウェア、AutoDock\cite{Morris2009}などのオープンソースウェアを
始めとして、有償無償問わず様々開発されている。主なドッキングシミュレーションソフトウェアを表\ref{table:docking_tools_eg}にまとめた。

\begin{table}[htb] \centering
	\caption{主なドッキングシミュレーションソフトウェア}
	\label{table:docking_tools_eg}
	\begin{tabular}{llcl}
	\hline
	\multicolumn{1}{c}{ソフトウェア名}	&\multicolumn{1}{c}{論文}			&\multicolumn{1}{c}{有償/無償}	&\multicolumn{1}{c}{ホームページ}	\\ \hline
	AutoDock					&Morris et al., 1998\cite{Morris1998}												&無償									&\url{http://autodock.scripps.edu/}	\\ 
	AutoDock Vina			&																											&無償									&\url{http://autodock.scripps.edu/}	\\ 
	DOCK							&\cite{Ewing2001}																				&無償									&\url{http://dock.compbio.ucsf.edu/}	\\
	eHiTS							&Zsoldos et al., 2007\cite{Zsoldos2007}											&有償									&\url{http://www.simbiosys.ca/ehits/}	\\
	FlexX							&																											&有償									&\url{http://www.biosolveit.de/flexx}	\\
	Glide							&Friesner et al.\cite{Friesner2004}													&有償									&\url{http://www.schrodinger.com/Glide/}	\\
	GOLD							&\cite{Jones1997}																				&有償									&\url{http://www.ccdc.cam.ac.uk/}	\\
	ICM								&																											&有償									&\url{http://www.molsoft.com/docking.html}	\\ \hline
	\end{tabular}
\end{table}


\subsection{ドッキングシミュレーションの要素}\label{subsec:docking_elements}
SBVSの薬物候補化合物の選別はドッキングシミュレーションによって得られたスコアを基に行われるため算出されるスコアは重要となるが、
後述するように探索空間が非常に広く、さらに最適化を行うべきスコア値も一般的に探索空間内で単調ではないため、
厳密な最適スコアを求めることは事実上不可能である。そのため、ドッキングシミュレーションにおいては
\begin{itemize}
\item 非常に広い探索空間からなる最適化問題で良い準最適解を効率良く見つける探索アルゴリズム
\item 適度に高速に計算でき、タンパク質-化合物の結合構造の良し悪しを適切に見積もるスコア関数
\end{itemize}
の2つが非常に重要であり、これらは1982年に最初のドッキングシミュレーションツールであるDOCK\cite{Kuntz1982}が開発されてより、
様々なグループによって研究が進められている。

\subsubsection{探索空間}
ドッキングシミュレーションでは、タンパク質の位置を固定として、化合物がタンパク質とどのような構造をとると良いかを探索する。\comment{図がほしい}
この際、探索しなければならない空間は化合物の並進運動および回転運動の6次元に加え、化合物の内部に回転可能な結合を持つため
化合物の内部自由度を考慮しなければならない(図\ref{fig:docking_freedom})。
この内部自由度はZINC Drug Database に登録されている2924個の薬剤化合物で平均4.61であり、これが計算量に大きな影響を及ぼす。

\begin{figure}[tb]
 \begin{center}
  \fig[width=0.9\hsize]{./fig/background/internal_freedom.eps}
  \caption{化合物の内部自由度}
  \label{fig:docking_freedom}
 \end{center}
\end{figure}


\subsubsection{探索アルゴリズム}
前述のように探索空間の広さのために大域最適解を求めることは困難であるため、より良い局所最適解を求めるための
工夫がツール毎になされている。
\begin{itemize}
\item Glide\\
	段階的な全探索を行うことで局所最適解を得る。具体的には、最初の段階では化合物を球体に近似しての位置が良いかどうかの
	見積もりから始め、徐々に化合物の近似を厳密なものにしていく。それぞれの段階で上位の位置・構造のみを残し次の段階へ進めることで、
	全探索の空間を現実的な量に制限し、探索を完了させる(図\ref{fig:glide_flowchart})。
\item eHiTS\\
	化合物を部分構造に分割し、部分構造にとって良い構造をそれぞれ多数記録し、ノードにする。
	その後、2つの部分構造が構造を構成するのに適度な距離、適度な向きになっているノード間にエッジを張り、
	作成されたグラフに関して最大クリーク問題を解くことで適切な構造を得る(図\ref{fig:eHiTS_clique})。
\item AutoDock\\
	並進運動位置、回転運動位置、化合物の内部回転角を用いた遺伝的アルゴリズム(Genetic algorithm, GA)で
	より良い局所最適解を得る(図\ref{fig:AutoDock_gene})。
\end{itemize}

\begin{figure}[htb]
 \begin{center}
  \fig[width=0.4\hsize]{./fig/background/glide.png}
  \caption{Glideのワークフロー\cite{Friesner2004}}
  \label{fig:glide_flowchart}
 \end{center}
\end{figure}
\begin{figure}[htb]
 \begin{center}
  \fig[width=0.6\hsize]{./fig/background/eHiTS_clique.eps}
  \caption{eHiTSのクリーク探索}
  \label{fig:eHiTS_clique}
 \end{center}
\end{figure}
\begin{figure}[htb]
 \begin{center}
  \fig[width=0.4\hsize]{./fig/background/AutoDock_gene.eps}
  \caption{AutoDockのGAで用いる変数群}
  \label{fig:AutoDock_gene}
 \end{center}
\end{figure}


\subsubsection{スコア関数}
探索アルゴリズムがどれほど良く、大域最適なスコアを得たとしても、そのスコアがタンパク質と化合物との物理的な結合エネルギーとの相関が
なければ意味がない。しかし、結合エネルギーを厳密に計算するには量子化学計算が必要となり、実用的な時間では計算が完了しないので、
近似計算が必要となる。したがって、スコア関数に関しても様々な提案がなされている。
\begin{enumerate}
\item 物理化学的スコア関数(Force field)
	静電相互作用力やファンデルワールス力など、原子と原子との間に働く物理化学的な力を元としたスコア関数であり、
	考慮する物理的な項や、原子に対するパラメータが異なるなどによって多数のスコア関数が提案されている
	\cite{Morris1998, Ewing2001, Jones1997}。
\item 経験的スコア関数(Empirical)
	水素結合など物理化学的な要素に基づきつつも、実験から得られたタンパク質と化合物との結合エネルギーを再現できるように
	関数の係数ではなく関数そのものをパラメータ化することによって作成されたスコア関数\cite{Wang2002, Gehlhaar1995, Eldridge1997}。
\item 統計的スコア関数(Knowledge-based)
	タンパク質と化合物との複合体構造はProtein Data Bank(PDB)\cite{Berman2000}に多数登録されている。
	この複合体構造におけるそれぞれの原子種間の距離や角度の確率密度を計算し、それをエネルギーに変換することで作成されたスコア関数
	\cite{Muegge2006, Xue2010, Gohlke2000, Huang2010}。この手法では、どの複合体構造のセットを用いるかによってスコア関数が変化するため、一部のターゲットに特化した
	スコア関数なども作成されている\cite{Seifert2009}。
\end{enumerate}

それぞれの具体的なスコア関数名、もしくはドッキングシミュレーションソフトウェア名を表\ref{table:scoring_function}に示す。
\todo{table:scoring\_function, Cheng2012の表をこちらに作成する}
\begin{figure}[h]
 \begin{center}
  \fig[width=0.8\hsize]{./fig/background/scoring_functions.png}
  \caption{作成する表(Cheng et al., 2012より)}
  \label{fig:scoring_function}
 \end{center}
\end{figure}

\subsection{ドッキングシミュレーションの問題点}\label{subsec:docking_problem}
\ref{subsec:docking_elements}節に述べたように、ドッキングシミュレーションツールはそれぞれ高速化のための工夫を凝らしているが、
それでも不十分であるのが現状である。
例えば、1コアを用いて1つの化合物を評価するのにGlideで0.2-2.4分程度\cite{Friesner2004}、eHiTSは最速で数秒\cite{Zsoldos2007}を要すると
述べられている。この速度で1,000万化合物を選別しようとすると10秒で1つの化合物を評価できたとしても1,200 CPU daysもの時間を必要とする。
このような場合に一般的に用いられる手段である大規模計算化に関しても、GlideやeHiTSはライセンス式の有償ソフトウェアであるために、
大量のライセンスを購入する必要があり現実的ではない。
一方、AutoDockはライセンスが必要なく大規模並列計算が可能であるが、Glideと比べて250倍程度も遅いという
報告がなされている\cite{Tuccinardi2010}。AutoDockはオープンソースウェアであるため、GPU実装による高速化も提案されているが、
遺伝的アルゴリズムやスコア関数の計算が最大50倍程度高速になる程度であり\cite{Kannan2010}、Glideに及ばない。

\begin{table}[htb] \centering
	\caption{ドッキングシミュレーションソフトウェアの計算速度}
	\label{table:docking_tools}
	\begin{tabular}{c|cl}
	\hline
	ソフトウェア名					&有償/無償				&\multicolumn{1}{c}{1 CPUでの計算速度} 					\\ \hline
	Glide							&有償						&10-150 sec/compound\cite{Friesner2004}				\\
	eHiTS							&有償						&$<$10 sec/compound(最速)\cite{Zsoldos2007}			\\
	AutoDock					&無償						&Glideの約250倍の計算時間\cite{Tuccinardi2010}		\\ \hline
%	\multirow{2}{*}{AutoDock}	&\multirow{2}{*}{無償}	&500 sec/compound\cite{Trott2010} ,						\\ 	
%									&							&Glideの約250倍の時間を要する\cite{Tuccinardi2010}	\\ \hline	
	\end{tabular}
\end{table}

\section{化合物のフィルタリング}
ドッキングシミュレーションは大きな計算量を必要とするために、1,000万個もの化合物から薬物候補化合物を選別しようとすることが非常に難しい
ことを\ref{subsec:docking_problem}節で述べた。このため、ドッキングシミュレーションを高速化するのではなく、
ドッキングシミュレーションの入力とする化合物の数をあらかじめ減ずることで総計算時間を削減するという戦略が創薬研究では良く用いられる。

\subsection{既存のフィルタリング手法}\label{subsec:existing_filtering}
既存のフィルタリング手法は大きく分けて1. 化合物の物理的特徴に基づくフィルタリング、2. 化合物の構造に基づくフィルタリング、3. ドッキングベースのフィルタリングの3種類が存在している。
\begin{enumerate}
\item \b{化合物の物理的特徴に基づくフィルタリング}\\
	分子量や水溶性か油溶性かを示す分配係数(LogP)など、
	化合物の物理的な特徴を示す値は体内での吸収などの上で重要な値である。
	したがって、化合物の物理的特徴を用いたフィルタリングが提案されている。
	\begin{itemize}
	\item \b{リピンスキーの法則\cite{Lipinski1997}}\\
		経口薬として優れた薬物の物理的特徴を4つの法則にまとめたもの
	\item \b{Quantitative Estimate of Druglikeness(QED)\cite{Bickerton2012}}\\
		既知薬剤の物理的な値からヒストグラムを作成し、化合物の薬物らしさ(Druglikeness)のスコアを付ける手法
	\end{itemize}
\item \b{化合物の構造に基づくフィルタリング}\\
	タンパク質の活動の阻害はタンパク質や化合物1分子単位の非常に微視的なメカニズムによって発生しており、
	したがって化合物の分子構造はタンパク質との複合体を形成する上で非常に重要な情報である。
	特に、構造が似ている化合物同士は同じタンパク質との複合体を形成することが多く、
	この性質を利用したフィルタリング手法が複数提案されている。
	\begin{itemize}
	\item \b{フィンガープリント(fingerprint)を用いたフィルタリング\cite{Nilakantan1993}}\\
		化合物の分子構造式を数百~数千のあらかじめ定めた局所構造が存在するか否かのバイナリである
		フィンガープリントに落とし、これを用いて既知の薬剤やタンパク質の阻害剤にどれほど近いかを判定する手法
	\item \b{ファーマコフォア(pharmacophore)を用いたフィルタリング\cite{Parenti2003}}\\
		分子の構造式のみではなく、化合物の立体構造も用いて化合物の類似性を評価する手法
	\end{itemize}
\item \b{ドッキングシミュレーションベースのフィルタリング}\\
	\ref{subsec:docking_problem}節で述べた通り、ドッキングシミュレーションは一般的に計算コストが高くフィルタリングには適していないが、
	一部の
	Glideには化合物の構造について強い仮定を置くことで計算を簡易化し、通常ドッキングモード(SPモード)の
	10倍程度の速度\cite{GlideHomePage}で計算を完了させる高速ドッキングモード(HTVSモード)が存在する。
	このモードをフィルタリングとして利用し、フィルタリング後の化合物群に対してSPモードによる
	ドッキングシミュレーションを行うという手法が用いられることがある\cite{Fujimoto2008}。
\end{enumerate}
\memo{これも文章ではなく表にまとめられないか?}

\subsection{既存手法の問題点}
\ref{subsec:existing_filtering}節で述べたように、既存のフィルタリング手法は多く存在するものの、
以下の2点からこれらの手法は改善する余地が残されている。
\begin{itemize}
\item 化合物の物理的特徴や化合物の構造に基づくフィルタリング手法は帰納的な手法であり、
	標的タンパク質を狙った既知の薬剤や阻害剤が必須となる。さらに既知の薬剤や阻害剤を利用できたとしても、
	化合物の類似性を利用する手法であるために既知の化合物に似てしまうという問題がある。	
\item Glideの高速ドッキングモードはドッキングシミュレーションとしては高速であるが、それでも1化合物1秒程度を要する。
	1,000万件の化合物のフィルタリングを行う場合1 CPU利用で4か月程度の期間を要してしまうため、この速度は十分とは言えない。
\end{itemize}

\chapter{提案手法:化合物の部分構造を利用したフィルタリング手法の開発}
ここでは、従来手法であるGlide HTVSとは異なり、化合物を部分構造に分割することで高速にドッキングを完了させるフィルタリング手法の内容を説明する。

\section{提案手法の概説}
前章で述べた通り、ドッキングシミュレーションは時間を要し、その理由は探索空間の広さと化合物の多様性にある。
この節では、この2つの問題を解決するアイデア、および高速にフィルタリングを行うために追加する仮定を説明する。

\subsection{フィルタリングの要件}\label{subsec}
フィルタリングに求められる要件は2つ存在する。
\begin{itemize}
\item 高速に化合物を評価する\\
	フィルタリングを実用的に行うためには、フィルタリング後に行うドッキングシミュレーションよりも十分に高速である必要がある。
\item 予測の精度がある程度保持されている\\
	一般に計算速度と予測精度はトレードオフの関係にあるが、どれほど高速であってもある程度正例と負例が弁別できなければ
	フィルタリングとして機能しない。したがって、予測精度がある程度保持されていることもフィルタリングには求められる。
\end{itemize}
一方、フィルタリングはその後に通常のドッキングシミュレーションを行うことを前提とするため、
必ずしも「化合物がタンパク質のこの部分に結合する」というドッキングポーズを出力する必要はない。

\subsection{提案手法へのアイデア}\label{subsec:idea}
前節で示したフィルタリングの要件を満たすために、2つのアイデアを考案した。

\subsubsection{化合物を部分構造に分割し、部分構造のドッキングシミュレーションを行う}
\ref{subsec:docking_elements}節で述べたように、化合物の内部自由度が及ぼす計算量への影響は大きい。
そこで、本提案手法では、小峰ら\citetodo{}による化合物の分割方法を用いて、化合物を内部自由度を考慮しなくて良い
「フラグメント」に分割、これらをドッキングすることで、必要最低限の探索空間でのドッキングシミュレーションを実現する。

\subsubsection{フラグメントから化合物の構造を再構成せず、フラグメントの結合スコアから化合物のフィルタリングスコアを算出する}
化合物をフラグメントに分割した上でドッキングシミュレーションを行うと図\ref{fig:divided_fragment}\todo{図の作成}のように
フラグメントごとにタンパク質との結合予測構造が出力され、フラグメントの結合スコアが最も良いポーズを選択したとしても
繋がった一つの化合物としては有り得ない構造をとる場合がほとんどである。
しかし、矛盾のない化合物の構造をとるようなフラグメントの選択を行うのは$O(a^n) (nは化合物を構成するフラグメント数)$の計算量となり、
大きな計算コストを要してしまう。

一方、フィルタリングはその後に通常のドッキングシミュレーションを行うことを前提とするため、
必ずしも化合物とタンパク質との結合予測構造を出力する必要はない。
そこで、提案手法では構造の矛盾の考慮を行わず、得られたフラグメントの結合スコアのみに着目し、
フラグメントの結合スコアから化合物のフィルタリングスコアを算出するのに計算が$O(n)$で済むようなスコアの統合を行うことで、
高速な化合物の評価を達成する。

\section{提案手法の詳細の説明}
前節で用いる2つのアイデアを示したが、それを用いてどのようにフィルタリングを実現しているのかをこの節で詳説する。

\subsection{提案手法のフローチャート}\label{subsec:flowchart}
提案手法は以下の手順で構成される。
\begin{enumerate}
\item 入力された化合物をフラグメントに分割する
\item ドッキングシミュレーションツールを用いてフラグメントの標的タンパク質への結合スコアを算出する
\item フラグメントの結合スコアから化合物のフィルタリングスコアを算出する
\item フィルタリングスコアの上位N\%をフィルタを通過した化合物として出力する
\end{enumerate}
ワークフローを図\ref{fig:workflow}に示す。

\begin{figure}[p]
 \begin{center}
  \fig[width=0.6\hsize]{./fig/method/proposal_workflow.eps}
  \caption{提案手法の手順}
  \label{fig:workflow}
 \end{center}
\end{figure}

\subsection{化合物のフラグメントへの分割}\label{subsec:decomposition}
化合物の分割は小峰らによる手法\cite{Shunta2015}を用い、内部自由度を持たない部分構造であるフラグメントを生成する。
実装にはC++を用い、ケモインフォマティクスツールであるOpenBabel\cite{OBoyle2011}およびOpenMP、Boostを利用している。
フラグメント分割のアルゴリズムを以下に示し、このアルゴリズムによるフラグメント分割の進行を図\ref{fig:decomposition}に示す。
\begin{enumerate} 
\item 元の分子のうち、重原子(水素以外の原子)のみに着目し、原子一つひとつをフラグメントとする。(図\ref{fig:decomposition} 左から2番目)
\item 回転可能な単結合以外の結合の両端の2原子を同一フラグメントとする。
\item 環構造を構成している原子を同一フラグメントとする。(図\ref{fig:decomposition} 左から3番目)
\item 回転可能な単結合を構成する原子ペアのうち、片方にそれ以上原子がつながっていない場合には同一フラグメントとする。
	これは、片方にそれ以上の原子がつながっていない場合、回転可能な単結合を回転させてもその原子がその場で回転するだけとなり、
	化合物の原子の位置関係には影響を与えないためである。
\item 2つの単結合の切断により孤立してしまう原子は、切断された先に存在する2つのフラグメントのどちらかに併合する。
	なお、3つ以上の単結合の切断により孤立してしまう原子に関してはこの操作を行わない。(図\ref{fig:decomposition} 左から4番目)
\item 全ての水素原子について、その原子が結合している重原子の属するフラグメントに含める。
\end{enumerate}

\begin{figure}[htp]
 \begin{center}
  \fig[width=0.99\hsize]{./fig/method/decomposition.png}
  \caption{化合物のフラグメント分割アルゴリズム\cite{Shunta2015}}
  \label{fig:decomposition}
 \end{center}
\end{figure}

この化合物のフラグメントへの分割により、内部自由度を考慮することなくドッキングシミュレーションを行うことができる。
また、複数の化合物間で部分構造に共通性が見られることが非常に多く、本研究で用いている分割手法によって得られるものの中にも
多数の共通フラグメントが発生する。例えば、ZINCの"drugs now"データセットに含まれている10,639,555化合物を順次フラグメント分割した
場合のフラグメントの種類数をプロットすると、図\ref{fig:decomposition_amount}のようになり、わずか20万フラグメントによって1,000万化合物が
構成されていることが分かる。
また、プロットの曲線具合からもわかるように、フラグメント分割を行いドッキングを行うという手法は、化合物数が多ければ多いほど化合物単位で
ドッキングする手法に比べて優位になる。

\begin{figure}[htp]
 \begin{center}
  \fig[width=0.6\hsize]{./fig/method/decomposition_zinc.eps}
  \caption{ZINC "drugs now" 10,639,555化合物を分割した例}
  \label{fig:decomposition_amount}
 \end{center}
\end{figure}

\subsection{フラグメント単位でのドッキングシミュレーション}
次に、分割されたフラグメントについて、標的タンパク質との結合スコアを求めるためにドッキングシミュレーションを行う。
本研究では、有償ソフトであるglide\citetodo{}を用いる。glideには高速(HTVS)モード、通常(SP)モード、精密(XP)モードの3種類のモードが
存在するが、本研究ではSPモードとHTVSモードを利用した場合の評価を行う。
SPモードはデフォルト設定では内部自由度を考慮したドッキングを行ってしまうため、内部自由度を無視するオプションを追加している。
また、一般的に1つのタンパク質と1つの化合物とのドッキング結果では複数のタンパク質-フラグメント結合予測構造および
結合スコアが出力されるが、この後の化合物のフィルタリングスコアの算出ではこのうち最良の結合スコアを利用する(図\ref{fig:fragment_result})。

\begin{figure}[htp]
 \begin{center}
  \fig[width=0.6\hsize]{./fig/method/フラグメントスコア算出方法.jpg}
  \caption{フラグメントの結合スコアの取得}
  \label{fig:fragment_result}
 \end{center}
\end{figure}

\subsection{化合物のフィルタリングスコアの算出}
フラグメント単位でのドッキングシミュレーションによって、フラグメントの結合構造およびその結合スコアを得た。続いて、このフラグメント結合スコアから
化合物のフィルタリングに用いるスコアを算出する。本研究では、3種類のスコアの算出方法の実験を行った。
なお、重原子数が2以下の小さなフラグメントの結合スコアはフィルタリングスコア算出から除外している。

\subsubsection{総和法(score\_sum)}
フラグメント結合スコアの総和をとり、それを化合物のフィルタリングスコアとする。
構成する全てのフラグメントがタンパク質と良い結合構造を取れるような化合物が薬物候補化合物として適している、として評価を高くする手法
である。フラグメント群は化合物に存在する結合という束縛条件を一部緩和したものであるため、一般にこの手法によって得られた
化合物フィルタリングスコアは化合物そのものの結合スコアよりも高くなる。

\subsubsection{最良値法(score\_max)}
フラグメント結合スコアの最良値をとり、それを化合物のフィルタリングスコアとする。
構成するフラグメントの内、1つでもタンパク質と非常に良い結合構造をとれるような化合物が薬物候補化合物として適している、として
評価を高くする手法である。フラグメント1つの結合スコアが化合物のフィルタリングスコアとなること、ドッキングシミュレーションを行う分子のサイズと
結合スコアには正の相関がある\cite{Verdonk2004}ことから、総和法とは異なりこの手法によって得られた化合物フィルタリングスコアは
化合物そのものの結合スコアよりも低くなる。

\subsubsection{総和法と最良値法の値の線形和(maxsumBS)}
これまでに示した総和法と最良値法はフラグメント結合スコアの全て、もしくはただ一つを見る手法であり両極端であるため、
これらを統合して用いることで、より良い指標となるのではないかと考えた。
しかし、総和法の値域が最良値法の値域よりも大きいために単純和では総和法の影響を大きく受けてしまう。
そこで、二つの手法を適当なバランスで組み合わせるために、フィルタリングを行いたい化合物の総和法によるスコア、最良値法によるスコアを
それぞれ平均0、分散1にし(すなわちzスコア化し)、変換後のスコアを足し合わせることでバランスよくスコアを統合することを試みた(図\ref{fig:maxsumBS})。
\begin{figure}[htp]
 \begin{center}
  \fig[width=0.99\hsize]{./fig/method/maxsumBS.eps}
  \caption{maxsumBSの算出}
  \label{fig:maxsumBS}
 \end{center}
\end{figure}
なお、総和法によるスコアと最良値法によるスコアのバランスをとったスコア、という意味でmaxsumBS(max-sum Balanced Score)として以下記述する。
\chapter{実験}
ここでは,提案手法と既存手法との比較実験を行い,提案手法の長所を示す.


\section{データセット}
本実験では,データセットとしてDirectory of Useful Decoys, Enhanced (DUD-E)\cite{Mysinger2012}を用いた.
DUD-Eはカリフォルニア大学サンフランシスコ校のMysingerらによって作成されたドッキングシミュレーション手法を評価するためのデータセットである.
DUD-Eには102種類の標的タンパク質(ターゲット)が登録されており,それぞれに対してタンパク質構造・正例化合物・負例化合物を用意している.
表\ref{table:dude}にターゲットごとの化合物数,正例と負例の比率の最小値,最大値,平均値を示す.
各ターゲットの詳細については付録\ref{appendix:dude}に記載する.なお,DUD-Eのターゲットのうちfgfr1およびfa10は
記載されている負例数とデータセットに実際に含まれている負例数が大きく異なっているが,そのまま扱うこととする.

\begin{table}[htb] \centering
	\caption{DUD-Eのターゲットの化合物数}
	\label {table:dude}
	\begin{tabular}{c|cc}
	\hline
				&総化合物数	&正/負例の比率	\\ \hline
	最大値	&52,022 (fnta)	&1:104 (fnta)			\\
	平均値	&13,881			&1:60					\\
	中央値	&9,297			&1:59					\\
	最小値	&472 (fgfr1)		&1:2.4 (fgfr1)			\\ \hline
	\end{tabular}
\end{table}

	
\section{予測精度の評価指標}
バーチャルスクリーニングでは,計算機による選別を通過して実際に活性実験が行われる化合物数が
元の化合物と比べてきわめて少ない状況を想定することが多い.
また,化合物データベースの中で実際に標的タンパク質を阻害する化合物は約1,000個に1個であると言われており,
したがって正例と負例の比率が大きく偏っていると言える.
そのためこの分野における予測精度の評価指標は以下の2種類が多く用いられている.

\begin{itemize}
\item \b{ROC-AUC}\\
	Receiver Operating Characteristic(ROC)曲線は,正例/負例の予測の閾値を変化させながら,縦軸にTrue Positive(TP)率,横軸にFalse Positive(FP)率をとった曲線である.
	TP率とはデータセット中の正例の中で正しく正例と判別されたものの割合であり,FP率とはデータセット中の負例の中で誤って正例と判別されたものの割合である.
	TP率,FP率はそれぞれ以下の式で求められる.
	\begin{eqnarray}
	{\rm TP率}	&=&	\frac{\rm \#TP}{\rm \#TP+\#FN} \\
	{\rm FP率}	&=&	\frac{\rm \#FP}{\rm \#FP+\#TN}
	\end{eqnarray}
	この方法によって描かれたROC曲線の曲線下面積(Area Under the Curve, AUC)を用いた評価指標がROC-AUCである.
	具体例を図\ref{fig:roc_example}に示す.
\item \b{Enrichment Factor}\\
	Enrichment Factor(EF)とは,予測結果の上位のみを取り出したときに,元々のデータセットからどれだけ正例が「濃縮されたか」を表す指標である.具体例を図\ref{fig:ef_example}に示す.
	上位どのくらいを取り出すかによって値が異なり,上位x\%取り出したときの集合の正例率を正例率(x\%),EFをEF (x\%)と表記することにすると,これらは以下の式で求められる.
	\begin{eqnarray}
	{\rm 正例率(x\%)}	&=& \frac{\rm 正例数(x\%)}{\rm 正例数(x\%)+負例数(x\%)} \\
	{\rm EF(x\%)} 	&=& \frac{\rm 正例率(x\%)}{\rm 正例率(100\%)}
	\end{eqnarray}
\end{itemize}

\begin{figure}[p]
 \begin{center}
  \fig[width=0.99\hsize]{./fig/result/roc計算例.eps}
  \caption{ROC-AUC計算例}
  \label{fig:roc_example}
 \end{center}
\end{figure}
\begin{figure}[htp]
 \begin{center}
  \fig[width=0.99\hsize]{./fig/result/ef計算例.eps}
  \caption{EF計算例}
  \label{fig:ef_example}
 \end{center}
\end{figure}

本研究においては,ROC-AUC,EF(1\%),EF(2\%),EF(5\%),EF(10\%)の5つの指標を用いて手法の評価を行う.


\section{計算環境}
本研究では,東京工業大学のスーパーコンピュータであるTSUBAME 2.5のThinノードを利用した.
利用した計算環境を表\ref{table:computer_node}に示す.

\begin{table}[htb] \centering
	\caption{利用した計算環境}
	\label{table:computer_node}
	\begin{tabular}{cc}
	\hline
	CPU		&Intel Xeon X5670, 2.93 GHz (6 cores) $\times2$ \\
	Memory	& 54 GB RAM \\ \hline 
	\end{tabular}
	\vspace{-2cm}
\end{table}

\section{比較対象}
本提案手法はドッキングに基づくフィルタリング手法であるため,同様の用途に用いられることのあるGlide HTVS(高速)モードを比較対象として用いる.
また,フィルタリングとしての性能を評価するために,Glide SP(通常)モードによる化合物ドッキングシミュレーションと組み合わせた評価も行うため,
計算時間などの評価に関してはGlide SPモードも比較対象とする.

\section{評価実験}

\subsection{フラグメント分割}\label{subsec:result_decomposition}
まず,今回用いる複数のターゲットについて,フラグメント分割を行うことでドッキングの必要数をどの程度減らせるのかを示す.
それぞれターゲットにフラグメント分割を適用した場合における化合物数とフラグメント種類数の推移は図\ref{fig:dude_decomposition}の通り
となり,DUD-Eターゲット全体で平均するとフラグメント種類数は化合物数の約4分の1に抑えられている(表\ref{table:dude_decomposition}).
化合物数が多いほど,化合物数に対するフラグメント種類数が抑えられる傾向にあることも確認された.

\begin{figure}[bhtp]
 \begin{center}
  \fig[width=0.75\hsize]{./fig/result/ターゲットに対するフラグメント数.eps}
	\vspace{-0.5cm}
  \caption{DUD-Eターゲットにおける化合物数とフラグメント種類数の関係}
  \label{fig:dude_decomposition}
 \end{center}
\end{figure}
\begin{table}[htb] \centering
	\caption{フラグメント分割を行った時のフラグメント1種類あたりの化合物数}
	\label{table:dude_decomposition}
	\begin{tabular}{c|rr}
	\hline
								&\multirow{2}{*}{ターゲット数}	&フラグメント1種類			\\
								&						&あたりの平均化合物数		\\ 
	\hline
	化合物数1万未満のDUD-Eターゲット		&53			&3.17							\\
	化合物数1万以上のDUD-Eターゲット		&49			&4.91							\\ 
	\hline
	全DUD-Eターゲット						&102		&4.00							\\ 
	\hline
	\end{tabular}
\end{table}


\subsection{ドッキング速度の評価}\label{subsec:single_calc_time}
つづいてフィルタリング手法の計算速度を評価する.ここでは以下に示す4種類の手法の計算時間を比較する.
\begin{enumerate}
\item Glide SPモードを利用した通常のドッキングシミュレーション 「Glide SP」
\item Glide HTVSモードを利用した簡易なドッキングシミュレーション 「Glide HTVS」
\item フラグメントのドッキングシミュレーションにGlide SPモードを用いた提案手法 「提案手法(SP)」
\item フラグメントのドッキングシミュレーションにGlide HTVSモードを用いた提案手法 「提案手法(HTVS)」
\end{enumerate}

\ref{subsec:result_decomposition}節で述べたように,1つのターゲットに含まれる化合物数
が多ければ多いほどフラグメント数は相対的に少なくなり提案手法の計算コストの削減が増幅される.
そこでDUD-E 102ターゲット全てでの所要計算時間の平均以外に,総化合物数が最小であるターゲット fgfr1,
総化合物数が平均値に近いターゲット adrb2,総化合物数が最大であるターゲット fntaの3種類について独立して結果を示す.

結果は表\ref{table:calc_time}の通りであり,Glide HTVSモードと比較すると提案手法(SP)は平均約9倍,
提案手法(HTVS)は平均約15倍の速度向上を達成している.

\begin{table}[htb] \centering
	\caption{ドッキング計算時間の比較(括弧内はGlide HTVSとの速度比)}
	\label{table:calc_time}
	\begin{tabular}{cc|rr|rrrr}
	\hline
	\multirow{3}{*}{\shortstack{問題\\サイズ}}	&\multirow{3}{*}{ターゲット名}	&\multirow{3}{*}{化合物数}	&フラグ	&\multicolumn{4}{c}{計算時間 [CPU sec.]}											\\
														&									&									&メント	&\mrow{2}{Glide SP}	&\mrow{2}{Glide HTVS}	&提案手法		&提案手法	\\
														&									&									&種類数	&							&					&(SP) 			&(HTVS)				\\ 
		\hline
	小													&fgfr1								&472								&166		&3,523					&566 (x1.0)		&164 (x3.5)		&140 (x4.0)				\\
	中													&adrb2							&15,224							&2,779	&338,511					&17,043 (x1.0)	&1,481 (x11.5)	&899 (x19.0)				\\
	大													&fnta								&52,022							&7,767	&1,770,967				&98,665 (x1.0)	&4,149 (x24.0)	&2,549 (x38.7)			\\ 
		\hline
	\multicolumn{2}{c|}{全ての平均}													&13,881							&3,231	&236,156					&14,813 (x1.0)	&1,673 (x8.9)	&987 (x15.0)				\\ 
		\hline
	\end{tabular}
\end{table}

\subsection{予測精度の評価}\label{subsec:single_accuracy}
次に,提案手法の予測精度の評価を行う.提案手法は2つのドッキングモード(SPモードおよびHTVSモード),
3つのフィルタリングスコア算出方法が存在するため合計6通りを示す.

\begin{table}[htb] \centering
	\caption{提案手法の予測精度}
	\label{table:single_accuracy}
	\begin{tabular}{l|l|rrrrr}
	\hline
	\multicolumn{1}{c|}{\mrow{2}{ドッキング計算}}	&\multicolumn{1}{c|}{化合物スコア}		&\multirow{2}{*}{ROC-AUC}	&\multicolumn{4}{c}{Enrichment Factor}	\\
															&\multicolumn{1}{c|}{算出方法}			&						&EF(1\%)	&EF(2\%)	&EF(5\%)	&EF(10\%)	\\ 
	\hline
															&総和(score\_sum)					&0.624					&5.08	&4.14	&3.02	&2.34		\\
	提案手法(SP)										&最良値(score\_max)					&0.637					&6.78	&\b{5.65}	&3.81	&2.60		\\
															&線形和(maxsumBS)					&\b{0.679}				&6.03	&5.03	&\b{3.96}	&\b{3.00}		\\
															&総和(score\_sum)					&0.618					&4.84	&3.97	&2.99	&2.29		\\
	提案手法(HTVS)									&最良値(score\_max)					&0.627					&\b{6.94}	&5.55	&3.32	&2.55		\\
															&線形和(maxsumBS)					&0.665					&5.98	&4.84	&3.58	&2.82		\\ 
	\hline
	\multicolumn{2}{c|}{簡易ドッキングシミュレーション}										&\mrow{2}{0.705}			&\mrow{2}{16.67}	&\mrow{2}{11.18}	&\mrow{2}{6.38}	&\mrow{2}{4.11}		\\ 
	\multicolumn{2}{c|}{(Glide HTVSモード)}											&&&&& \\
	\hline
	\end{tabular}
\end{table}

結果は表\ref{table:single_accuracy}の通りである.なお,各手法を用いた場合のターゲットごとのROC曲線は付録\ref{appendix:roc}に記載している.
この結果から,単体での予測精度に関しては,どの評価指標においても既存手法であるGlide HTVSモードが高速性を重視した本研究の提案手法よりも
勝っていることが分かる.

また,提案手法間の比較を行うことで以下のことが言える.
\begin{itemize}
\item ドッキング計算について,化合物フィルタリングスコアの算出方法に関わらず,提案手法(SP)は提案手法(HTVS)と比べて
	ほぼすべての評価指標で予測精度が良くなる.
\item ROC-AUCはmaxsumBSが他の2つの提案手法に比べて良い結果が出ているが,EF (1\%)やEF (2\%)に関してはscore\_maxが
	maxsumBSを上回っている.
\end{itemize}
\ref{subsec:single_calc_time}で述べたように提案手法(SP)の速度は9倍程度,従来手法に比べて高速である.
提案手法(HTVS)は従来手法に比べて15倍程度高速であり,計算時間の短縮が特に重要な場合には考慮に値するが,
本研究では高速化を達成した中での精度を重視し,フラグメントのドッキングシミュレーションにはGlide SPモードを用いることとする.

\newpage

\subsection{フィルタリング手法としての性能評価実験}
\ref{subsec:single_calc_time}節および\ref{subsec:single_accuracy}節では,フィルタリング手法を単体で用いた場合の性能を評価し,
速度では提案手法が勝っているものの,精度ではGlide HTVSモードに後塵を拝する結果となった.
しかし,本研究で提案した手法はフィルタリングを想定したものであり,その次に行われる
通常のドッキングシミュレーション手法と組み合わせた場合の速度や精度の評価はより重要となる.


この節では通常のドッキングシミュレーションであるGlide SPモードとの組み合わせを通した評価を行う.
組み合わせを通した評価は
\begin{enumerate}
\item フィルタリング手法で2\%, 5\%, 10\%まで化合物を削減(以下,この割合を「通過率」と示す)
\item 残った化合物を通常のドッキングシミュレーション(Glide SPモード)で再計算
\item 再計算の結果の上位1\%および上位2\%の濃縮率(EF (1\%), EF (2\%))を評価
\end{enumerate}
という手順を用いる.
なおフィルタリング手法を用いて2\%まで削減した場合,EF (2\%)はフィルタリング手法単体の性能と変わらなくなるため,
「\textendash」と表記する.

\begin{figure}[htp]
 \begin{center}
  \fig[width=0.99\hsize]{./fig/result/filtering_image.pdf}
  \caption{EF(1\%),EF(2\%)算出までの流れ}
  \label{fig:filtering_image}
 \end{center}
\end{figure}

\newpage

\subsubsection{提案手法間の精度比較}\label{subsubsec:filtering_proposal}
まず提案手法間の精度比較を行い,化合物フィルタリングスコアの算出方法を検討した.
結果は表\ref{table:filtering_proposal}のようになり,多くの場合
フィルタリングスコア算出方法はmaxsumBSを用いるのが最適であることが分かった.
以降ではmaxsumBSを用いることとする.

\begin{table}[htb] \centering
	\caption{フィルタリング手法としての提案手法間の精度評価}
	\label{table:filtering_proposal}
	\begin{tabular}{lc|rr|r}
	\hline
	\multicolumn{2}{c|}{フィルタリング}	&\multirow{2}{*}{EF(1\%)}	&\multirow{2}{*}{EF(2\%)}	&合計計算時間	\\
	\multicolumn{1}{c}{手法}	&通過率	&								&									&[CPU sec.]		\\ \hline
	総和(score\_sum)			&			&6.84							&\textendash					&					\\
	最良値(score\_max)		&2\%		&\textbf{9.09}				&\textendash					&6,396			\\
	線形和(maxsumBS)		&			&8.75							&\textendash					&					\\ \hline
	総和(score\_sum)			&			&9.61							&5.92								&					\\
	最良値(score\_max)		&5\%		&10.93						&7.49								&13,481			\\
	線形和(maxsumBS)		&			&\textbf{12.92}				&\textbf{7.99}					&					\\ \hline
	総和(score\_sum)			&			&12.41						&7.67								&					\\
	最良値(score\_max)		&10\%		&11.85						&8.24								&25,289			\\
	線形和(maxsumBS)		&			&\textbf{15.45}				&\textbf{10.00}					&					\\ \hline
	\end{tabular}
\end{table}


\subsubsection{予測精度の従来手法との比較}\label{subsubsec:filtering_comparison}
続いて,提案手法と従来手法との比較を行った.

\begin{table}[htb] \centering
	\caption{フィルタリング手法としての提案手法と従来手法の比較}
	\label{table:filtering_proposal_Glide}
	\begin{tabular}{lc|rr|rr}
	\hline
	\multicolumn{2}{c|}{フィルタリング}					&\multirow{2}{*}{EF(1\%)}	&\multirow{2}{*}{EF(2\%)}	&合計計算時間	&\multicolumn{1}{c}{提案手法の}	\\
	\multicolumn{1}{c}{手法}		&通過率				&						&						&[CPU sec.]		&\multicolumn{1}{c}{高速化率}	\\ \hline
	提案手法(maxsumBS)		&\multirow{2}{*}{2\%}	&8.75					&\textendash				&6,396			&\mrow{2}{\b{x 3.05}}\\
	従来手法(Glide HTVSモード)	&					&17.85					&\textendash				&19,536		&\\ \hline
	提案手法(maxsumBS)		&\multirow{2}{*}{5\%}	&12.92					&7.99					&13,481			&\mrow{2}{\b{x 1.97}}\\
	従来手法(Glide HTVSモード)	&					&18.97					&12.50					&26,621		&\\ \hline
	提案手法(maxsumBS)		&\multirow{2}{*}{10\%}	&15.46					&10.00					&25,289			&\mrow{2}{\b{x 1.52}}\\
	従来手法(Glide HTVSモード)	&					&19.60					&12.92					&38,429		&\\ \hline
	\multicolumn{2}{l|}{通常ドッキング(Glide SPモード)}	&21.54					&14.68					&236,156& \multicolumn{1}{c}{\textendash}	\\ \hline
	\end{tabular}
\end{table}

表\ref{table:filtering_proposal_Glide}の結果より,以下のことが言える.
\begin{itemize}
\item \ref{subsec:single_accuracy}節で示した単体での性能評価と同様に,Glide HTVSモードが
	提案手法よりも精度が良くなっている.
\item 一方,計算速度について,フィルタリングで元の化合物群の2\%を通過させる場合,提案手法と通常ドッキング計算の
	合計必要時間が従来用いられていたGlide HTVSモードよりも少なくなっており,これまででは達成できなかった速度での
	化合物の選別が可能になっていることがこの結果からわかる.この利点はフィルタを通過させる化合物の割合を高めるほど薄れて行く.
	これは,通常のドッキングシミュレーションの計算時間が支配的となり,
	提案しているフィルタリング手法の計算時間の面での利点が失われてしまうためである.
\end{itemize}
\chapter{考察}

\section{総和法におけるフラグメント数に対するペナルティ}\label{sec:discussion_penalty}
もし,フラグメントの結合スコアを単純に全て加算し,それを化合物のフィルタリングスコアとすると,図\ref{fig:no_omit_score_graph}のように
化合物の総原子数が同じであっても分割数が多いほどフィルタリングスコアが向上してしまう.これは計算手法によって発生してしまった誤った傾向である.
この分割数と総和法のスコアとの相関は最適化問題の条件緩和と考えることで説明できる.すなわち,本来化合物には原子間の結合距離という拘束条件が存在している.
フラグメント分割によって切断された原子間の結合は距離を考えずにスコア付けして良いので,分割は原子間の結合という拘束条件を
1つずつ緩和することに対応する.このため,フラグメント分割がされればされるほどスコアが良くなってしまうのである.
\begin{figure}[b]
 \begin{center}
  \fig[width=0.56\hsize]{./fig/discussion/no_omit_score.eps}
  \caption{ターゲットfntaの全ての化合物のうち重原子数32の化合物の単純加算スコア}
  \label{fig:no_omit_score_graph}
 \end{center}
\end{figure}

このような現象を改善するための手法として,以下2つの実験を行った.
\begin{enumerate}
\item \b{小さなフラグメントの無視}\\ 
	重原子(水素以外の原子)の個数に閾値を設け,その閾値を超えているフラグメントの結合スコアのみを
	総和に用いる.分割が多ければ多いほど小さなフラグメントが発生するため,
	小さなフラグメントの結合スコアを無視することで事実上のフラグメント数に対するペナルティとなる.
\item \b{フラグメント数に対する線形ペナルティ}\\
	全てのフラグメントの結合スコアを加算した後,化合物が持つフラグメントの個数に応じたペナルティを付与する.
	図\ref{fig:no_omit_score_graph}を見ると,フィルタリングスコアの平均とフラグメント数との関係は線形に近く,
	フラグメント数に対して線形なペナルティを課すことでフラグメント数に依存しないフィルタリングスコアとなることが想定される.
\end{enumerate}

この2つの手法を個別に利用した場合の総和法(score\_sum)の精度は表\ref{table:omit}および表\ref{table:penalty}のようになり,
重原子数3以下のフラグメントの結合スコアを無視することが最も精度を高めている.

\begin{table}[h] \centering
	\caption{小さなフラグメントを無視することによるscore\_sumの精度の変化}
	\label{table:omit}
	\begin{tabular}{c|rrrrr}
	\hline
	\multirow{2}{*}{無視するフラグメントのサイズ}	&\multirow{2}{*}{ROC-AUC}	&\multicolumn{4}{c}{Enrichment Factor}				\\
										&						&EF(1\%)		&EF(2\%)		&EF(5\%)		&EF(10\%)	\\ \hline
	全てのフラグメントを利用					&0.545					&3.46		&2.76		&2.01		&1.63		\\
	重原子数1							&0.557					&2.38		&2.16		&1.85		&1.66		\\
	重原子数2以下						&0.624					&5.08		&4.14		&3.02		&2.34		\\
	重原子数3以下						&{\bf 0.634}				&{\bf 5.75}	&{\bf 4.34}	&{\bf 3.03}	&{\bf 2.49}	\\
	重原子数4以下						&0.620					&4.27		&3.43		&2.79		&2.32		\\
	重原子数5以下						&0.614					&4.43		&3.68		&2.75		&2.13		\\
	重原子数6以下						&0.537					&2.20		&1.86		&1.53		&1.43		\\ \hline
	\end{tabular}
\end{table}
\begin{table}[h] \centering
	\caption{フラグメント数に対する線形ペナルティによるscore\_sumの精度の変化}
	\label{table:penalty}
	\begin{tabular}{c|rrrrr}
	\hline
	フラグメント1つあたりの	&\multirow{2}{*}{ROC-AUC}	&\multicolumn{4}{c}{Enrichment Factor}				\\
	ペナルティ$c$			&						&EF(1\%)		&EF(2\%)		&EF(5\%)		&EF(10\%)	\\ \hline
	ペナルティなし			&0.545					&3.46		&2.76		&2.01		&1.63		\\
	$c=1$				&0.559					&3.81		&2.98		&2.18		&1.77		\\
	$c=2$				&0.586					&4.70		&3.65		&2.66		&2.08		\\
	$c=3$				&{\bf 0.622}				&{\bf 5.03}	&{\bf 4.03}	&{\bf 2.86}	&{\bf 2.32}	\\
	$c=4$				&0.588					&3.80		&3.19		&2.51		&2.14		\\
	$c=5$				&0.549					&3.57		&2.96		&2.20		&1.78		\\
	$c=6$				&0.530					&3.30		&2.53		&1.91		&1.57		\\
	$c=7$				&0.520					&3.06		&2.29		&1.72		&1.46		\\ \hline
	\end{tabular}
\end{table}

\newpage

一方,同様に総和法のペナルティを変化させながら総和法と最良値法の線形和(maxsumBS)の精度について実験を行うと,
重原子数2以下のフラグメントの結合スコアを無視した総和法を用いた場合に最良のROC-AUCとなった(表\ref{table:omit_maxsumBS},
表\ref{table:penalty_maxsumBS}).
maxsumBSの精度はscore\_sumよりも良いことから,本研究の提案手法では重原子数2以下のフラグメントの結合スコアを無視した
総和法を利用する.
\begin{table}[hb] \centering
	\caption{小さなフラグメントを無視することによるmaxsumBSの精度の変化}
	\label{table:omit_maxsumBS}
	\begin{tabular}{c|rrrrr}
	\hline
	\multirow{2}{*}{無視するフラグメントのサイズ}	&\multirow{2}{*}{ROC-AUC}	&\multicolumn{4}{c}{Enrichment Factor}				\\
										&						&EF(1\%)		&EF(2\%)		&EF(5\%)		&EF(10\%)	\\ \hline
	全てのフラグメントを利用					&0.652					&5.35		&4.56		&3.32		&2.60		\\
	重原子数1							&0.652					&4.67		&4.18		&3.25		&2.56		\\
	重原子数2以下						&\b{0.679}				&\b{6.03}		&\b{5.03}		&\b{3.96}		&\b{3.00}		\\
	重原子数3以下						&0.672					&5.57		&4.79		&3.78		&2.85		\\
	重原子数4以下						&0.653					&4.89		&4.32		&3.46		&2.67		\\
	重原子数5以下						&0.643					&4.95		&4.28		&3.29		&2.55		\\
	重原子数6以下						&0.566					&2.76		&2.46		&2.03		&1.79		\\ \hline
	\end{tabular}
\end{table}
\begin{table}[hb] \centering
	\caption{フラグメント数に対する線形ペナルティによるmaxsumBSの精度の変化}
	\label{table:penalty_maxsumBS}
	\begin{tabular}{c|rrrrr}
	\hline
	フラグメント1つあたりの	&\multirow{2}{*}{ROC-AUC}	&\multicolumn{4}{c}{Enrichment Factor}				\\
	ペナルティ$c$			&						&EF(1\%)		&EF(2\%)		&EF(5\%)		&EF(10\%)	\\ \hline
	ペナルティなし			&0.652					&5.35		&4.56		&3.32		&2.60		\\
	$c=1$				&0.657					&5.67		&4.78		&3.40		&2.67		\\
	$c=2$				&\b{0.665}				&6.40		&\b{5.02}		&3.59		&2.80		\\
	$c=3$				&\b{0.665}				&5.97		&4.84		&\b{3.78}		&\b{2.88}		\\
	$c=4$				&0.630					&6.16		&4.69		&3.41		&2.66		\\
	$c=5$				&0.609					&\b{6.49}		&4.71		&3.17		&2.45		\\
	$c=6$				&0.600					&\b{6.49}		&4.69		&3.11		&2.36		\\
	$c=7$				&0.591					&6.43		&4.62		&3.06		&2.32		\\ \hline
	\end{tabular}
\end{table}

\newpage

\section{提案手法が得意とするケースの調査}
\ref{subsec:single_accuracy}節の実験結果より,提案手法は簡易なドッキングシミュレーションであるGlide HTVSモード
と比べて精度が低調に終わることが判明している.
しかし,本研究で用いた102ターゲット中46ターゲットに関しては提案手法が従来手法であるGlide HTVSモードよりも精度が良く,
ROC-AUCで0.2以上上回っているケースも表\ref{table:target_accuracy_good}に示す通り3例存在している.

どのような場合において提案手法が有用であるかを調べるため,この3つのターゲットについて
化合物の持つフラグメント数の平均,小さなフラグメントを削減した後のフラグメント数の平均,
sitemap\cite{Halgren2009}によって計算された各タンパク質の結合部位のサイズを求めた.
その結果,結合部位のサイズやデータセット全体を通してのフラグメント数などに傾向は見受けられなかったが,
\begin{itemize}
\item 化合物の持つフラグメント数の平均
\item 重原子数が2以下のフラグメント数の平均
\end{itemize}
どちらも正例より負例が上回っているということが判明した(表\ref{table:good_property}).

\begin{table}[hb] \centering
	\caption{提案手法が上手く行ったケース}
	\label{table:target_accuracy_good}
	提案手法(maxsumBS)が従来手法(Glide HTVSモード)よりもROC-AUCで\\
	0.2以上上回ったケースについて,ROC-AUCの差の降順で示している.
	\begin{tabular}{c|r|rr}
	\hline
	\multirow{2}{*}{ターゲット名}	&\multirow{2}{*}{ROC-AUC差}	&\multicolumn{2}{c}{ROC-AUC}	\\
							&							&従来手法	&提案手法		\\ \hline
	mcr					&0.319						&0.466		&{\bf 0.785}		\\
	akt1					&0.285						&0.539		&{\bf 0.824}		\\
	gcr						&0.252						&0.528		&{\bf 0.780}		\\ \hline
	\end{tabular}
\end{table}
\begin{table}[htb] \centering
	\caption{提案手法が得意なターゲットの性質}
	\label{table:good_property}
	\begin{tabular}{c|rrrrrrr}
	\hline
	\multirow{2}{*}{ターゲット}	&\multicolumn{3}{c}{\multirow{2}{*}{フラグメント数の平均}}	&\multicolumn{3}{c}{重原子数2以下の}		&\multirow{2}{*}{結合部位のサイズ [$\mathrm{\r{A}}^3$]}	\\
							&		&		&								&\multicolumn{3}{c}{フラグメント数の平均}	&										\\
							&全体	&正例	&負例							&全体	&正例		&負例			&										\\ \hline
	akt1						&7.98	&6.94	&8.00							&4.64	&3.08		&4.66			&637									\\
	gcr						&5.93	&5.33	&5.94							&2.68	&2.00		&2.69			&471									\\
	mcr						&5.90	&5.43	&5.91							&2.67	&2.06		&2.68			&179									\\ \hline
	全102ターゲット平均		&7.22	&7.43	&7.21							&3.83	&3.78		&3.83			&437									\\ \hline
	\end{tabular}
\end{table}

\newpage

例えば,ホルモテロール(図\ref{fig:drugs}-(a))とカンデサルタン(図\ref{fig:drugs}-(b))はそれぞれ薬剤として
認められている化合物だが,重原子数が2以下になるフラグメントの量がかなり異なる.
このような場合,
前者(a)のような重原子数2以下のフラグメントが多く発生してしまう化合物が結合するタンパク質を
対象としたフィルタリングには従来通り簡易ドッキングシミュレーションを用い,
後者(b)のような重原子数2以下のフラグメントがあまり発生しない化合物が結合するタンパク質を
対象としたフィルタリングには提案手法を用いることで
より高い精度でフィルタリングを行うことができると推定される.

一方この性質は\ref{sec:discussion_penalty}節で示したペナルティの影響を受けていると考えられるため,
化合物の構造の有利不利が発生しないようなスコア計算手法およびペナルティの考案を引き続き行う必要がある.

\begin{figure}[t]
 \begin{center}
  \fig[width=0.85\hsize]{./fig/discussion/drugs_decomposition.eps}
 \end{center}
  \caption{薬剤化合物の例:(a)ホルモテロール, (b)カンデサルタン}
  \label{fig:drugs}
\end{figure}


\section{提案手法の利用例}
バーチャルスクリーニングでは数百万化合物から数百化合物程度を選別することが多く,
上位0.01\%など,非常に小さな比率におけるEnrichment Factorの計算などが本来必要となる.
また,計算時間に関しても数百万化合物を用いた場合に何日を要するのか,という評価が必要である.

しかし本研究で評価に利用したデータセットであるDUD-Eは表\ref{table:dude}で示したように472化合物しか存在しないターゲットも存在しており,このようなターゲットは上位0.01\%を計算することは不可能である.
そこで,ここではDUD-Eのターゲットのうち総化合物数が10,000以上である49ターゲットを用いることで,フィルタリングにおける化合物の通過率がより少ない場合や,
EF (0.1\%)などの小さな割合におけるEFの評価(表\ref{table:usecase_accuracy})を用い,実際のバーチャルスクリーニングでの提案手法の利用例を示す.
計算速度に関しては線形に計算量が増大すると仮定することで数百万化合物を評価した場合の計算時間を見積もる.

なお,総化合物数が10,000以上である49ターゲットの平均化合物数は22,259, 平均フラグメント種類数は4,588である.

\begin{enumerate}
\item \b{超高速な化合物全体の評価}\\
	表\ref{table:usecase_accuracy}によると,提案手法で0.5\%の化合物をフィルタリングし,それらを通常のドッキングシミュレーションで再評価
	することでGlide HTVSモードを用いる場合の約8分の1の計算時間で評価を完了させることができる.例えば1,000万化合物を評価する場合,
	今回のケースの450倍程度の化合物数となるので,Glide HTVSモードは1 CPU換算で4か月程度を要してしまう.
	一方,提案手法と通常ドッキングであるGlide SPモードの組み合わせでは1 CPUでも半月程度で済む計算となる.この差は大きく,
	提案手法は有用であると言える.
\begin{table}[htbp] \centering
	\caption{化合物全体を評価するのに要する時間の比較}
	\label{calc_speed_ultrafast}
	\begin{tabular}{l|rr}
	\hline
												&合計計算時間	&1,000万化合物評価の		\\ 
												&[CPU sec.]		&推定時間 [CPU days]		\\ \hline
	提案手法で0.5\%フィルタリング	&3,280				&17.1								\\
	Glide HTVSモード単独性能		&23,552				&122.5								\\ \hline
	\end{tabular}
\end{table}
\item \b{従来手法以下の所要時間の中での予測精度の向上}\\
	表\ref{table:usecase_accuracy}に示されている通り,提案手法と従来手法とで単純に比較を行うと精度は従来手法に分がある.
	しかしいくつかのケースについては,化合物ライブラリのサイズを変えることで同程度の所要時間の中で精度を高めることができる.
	例えば,100万化合物をGlide HTVSモードを用いて10\%にフィルタリングし,Glide SPでリランキングした場合,上位1万化合物の濃縮率(EF 1\%に相当する)は16.89,この時の推定必要計算時間は32.0 CPU daysとなる.
	一方,1,000万化合物を提案手法で1\%にフィルタリングし,Glide SPでリランキングした場合,上位1万化合物の濃縮率(EF 0.1\%に相当する)は20.26,この時の推定必要計算時間は26.9 CPU daysとなり,
	速度を向上させつつ,予測精度を高めることができる.このようなケースは複数存在しており(表\ref{table:win_proposal_case}),これらの場合においては提案手法を利用すべきであると言える.
\end{enumerate}


\begin{landscape}
\begin{table}[p] \centering
	\caption{総化合物数が1万以上存在するDUD-Eのターゲットに対する評価実験}
	\label{table:usecase_accuracy}
	\begin{tabular}{lc|rrrrr|r}
	\hline
	\multicolumn{2}{c|}{フィルタリング}					&\multirow{2}{*}{EF(0.1\%)}	&\multirow{2}{*}{EF(0.2\%)}	&\multirow{2}{*}{EF(0.5\%)}	&\multirow{2}{*}{EF(1\%)}	&\multirow{2}{*}{EF(2\%)}	&合計計算時間	\\
	\multicolumn{1}{c}{手法}	&通過率					&						&						&						&						&						&[CPU sec.]		\\ \hline
	提案手法(maxsumBS)		&\multirow{2}{*}{0.5\%}	&14.33					&9.39					&\textendash				&\textendash				&\textendash				&3,280			\\
	従来手法(Glide HTVSモード)	&					&35.16					&29.97					&\textendash				&\textendash				&\textendash				&25,452			\\
	提案手法(maxsumBS)		&\multirow{2}{*}{1\%}	&20.26					&13.91					&7.14					&\textendash				&\textendash				&5,180			\\
	従来手法(Glide HTVSモード)	&					&35.36					&31.17					&22.19					&\textendash				&\textendash				&27,352			\\
	提案手法(maxsumBS)		&\multirow{2}{*}{2\%}	&24.87					&19.14					&10.57					&6.32					&\textendash				&8,979			\\
	従来手法(Glide HTVSモード)	&					&35.54					&31.69					&23.10					&15.50					&\textendash				&31,151			\\
	提案手法(maxsumBS)		&\multirow{2}{*}{5\%}	&29.24					&25.04					&16.80					&10.51					&6.29					&20,378			\\
	従来手法(Glide HTVSモード)	&					&35.54					&31.40					&23.56					&16.28					&10.59					&42,550			\\
	提案手法(maxsumBS)		&\multirow{2}{*}{10\%}	&31.94					&27.48					&19.68					&13.58					&8.36					&39,377			\\
	従来手法(Glide HTVSモード)	&					&35.70					&31.78					&23.80					&16.89					&11.07					&61,549			\\ \hline
	\multicolumn{2}{l|}{通常ドッキング(Glide SPモード)}	&35.98					&32.82					&25.57					&18.96					&12.82					&379,965			\\ \hline
	\end{tabular}
\end{table}
\end{landscape}

\begin{table}[p] \centering
	\caption{提案手法が従来手法に速度・精度ともに勝る例}
	\label{table:win_proposal_case}
	\begin{tabular}{l|rrr}
	\hline
												&\multirow{2}{*}{化合物数}	&上位1万化合物の		&推定計算時間	\\
												&														&濃縮率(EF)				&[CPU days]		\\ \hline
	提案手法で1\%フィルタリング		&1,000万											&20.26 (EF 0.1\%)		&26.9				\\
	Glide HTVSモードで10\%フィルタリング	&100万												&16.89 (EF 1\%)			&32.0				\\ \hline
	提案手法で1\%フィルタリング		&500万												&13.91 (EF 0.2\%)		&13.5				\\
	Glide HTVSモードで10\%フィルタリング	&50万												&11.07 (EF 2\%)			&16.0				\\ \hline
	提案手法で2\%フィルタリング		&1,000万											&24.87 (EF 0.1\%)		&46.7				\\
	Glide HTVSモードで10\%フィルタリング	&200万												&23.80 (EF 0.5\%)		&64.0				\\ \hline
	提案手法で5\%フィルタリング		&200万												&16.80 (EF 0.5\%)		&21.2				\\
	Glide HTVSモードで5\%フィルタリング		&100万												&16.28 (EF 1\%)			&22.1				\\ \hline
	

	\end{tabular}
\end{table}
	

\chapter{結論}
\section{本研究の結論}\label{sec:conclusion}
本研究では,ドッキングに基づいた高速なフィルタリング手法を提案した.
提案手法をDUD-Eの全ターゲットである102種のデータセットを用いて評価すると,
予測精度は簡易ドッキングシミュレーションであるGlide HTVSモードに比べ劣っているが
計算速度は既存手法では実現不可能なほど高速であることが示された.
\begin{table}[h] \centering
	\caption{提案手法の性能}
	\label{table:conclusion_1}
	\begin{tabular}{l|rrrrrr}
	\hline
	\multicolumn{1}{c|}{\multirow{2}{*}{手法}}	&\multirow{2}{*}{ROC-AUC}	&\multicolumn{4}{c}{Enrichment Factor}	&平均計算時間	\\
										&						&EF(1\%)	&EF(2\%)	&EF(5\%)	&EF(10\%)	&[CPU sec.]		\\ \hline
	\b{提案手法(maxsumBS, }	&&&&&& \\
	\b{フラグメントドッキング:}				&0.679					&6.03	&5.03	&3.96	&3.00		&\b{1,673}		\\
	\b{Glide SPモード)}			&&&&&& \\ \hline
	従来手法(簡易ドッキング	&&&&&& \\
	シミュレーション	Glide 		&0.705&16.67&11.18&6.38&4.11&14,813 \\ 
	HTVSモード 通常利用)		&&&&&& \\
	\hline
	\end{tabular}
\end{table}

また,フィルタリング後に行う通常のドッキングシミュレーションと組み合わせた場合の速度・精度の評価を行い,
提案手法をフィルタリング手法として用いるべきユースケースを示した.
\begin{table}[b] \centering
	\caption{通常ドッキング(Glide SP)と組み合わせた速度・精度評価}
	\label{table:conclusion_2}
	\begin{tabular}{l|rrr}
	\hline
												&\multirow{2}{*}{化合物数}	&上位1万化合物の		&推定計算時間	\\
												&														&濃縮率(EF)				&[CPU days]		\\ \hline
	提案手法で1\%フィルタリング		&1,000万											&\b{20.26 (EF 0.1\%)}		&\b{26.9}				\\
	従来手法で10\%フィルタリング	&100万												&16.89 (EF 1\%)			&32.0				\\ \hline
	\end{tabular}
	\vspace{-0.5cm}
\end{table}

\newpage

\section{今後の課題}
本研究の今後の課題として,以下の事項が考えられる.
\begin{enumerate}
\item 速度をなるべく維持しつつの精度の向上\\
	提案手法は高速な計算を可能にしている一方,精度は簡易ドッキングシミュレーションに劣っており,改善の余地がある.
	改善の方策として,以下が考えられる.
	\begin{itemize}
	\item フラグメントをドッキングする際のスコア関数の改善
	\item 通常ドッキングシミュレーションの化合物の結合スコアへの,フィルタリングスコアのフィッティング
	\item 総和法(score\_sum)のペナルティのターゲットごとの調整
	\item 化合物のスコア算出時の非現実的なフラグメント配置に対するペナルティの付与
	\end{itemize}
\item 数百万~数千万化合物程度の,より現実のバーチャルスクリーニングに即した化合物データセットを用いた速度評価\\
	本研究では一般的なベンチマークデータセットであるDUD-Eを用いた評価を行ったが,これは1つのターゲットに対して最大でも約50,000個
	しか化合物が登録されていない.
	現実のバーチャルスクリーニングで一般的に行われている数百万個の化合物を用いた評価を行うことで,提案手法の有用性をより明示的に
	示すことができると考えられる.
\item 提案手法と従来手法の2段階フィルタリングを行った場合の性能・速度評価\\
	本研究の結果,提案手法は簡易ドッキングシミュレーションのGlide HTVSモードよりも高速であるが精度は劣っていた.
	通常のドッキングシミュレーションに対するフィルタリングのように,簡易ドッキングシミュレーションの前に提案手法を用いることで
	予測精度を保ちつつ計算量を削減することができると考えられる.
\item Glide以外のツールを利用した場合の提案手法の評価\\
	本研究ではフラグメントのドッキングシミュレーションにはGlide SPモードないしはGlide HTVSモードを利用したが,
	ここで用いるソフトウェアは自由に選択することができる.他のソフトウェアを使った場合の精度や速度の評価を行うことで,
	手法の汎用性を確かめることは重要である.
\end{enumerate}


%==============[ 謝辞 ]================================================================================
\newpage
\begin{syaji}
%\chapter{謝辞}
本研究を進めるにあたり、貴重な時間を割いてご指導を賜り、本論文をまとめる際においても細やかなご助言をいただきました秋山 泰教授に深く感謝申し上げます。

また、研究内容の方向性のディスカッションや発表資料のとりまとめなど、多くの事柄に対して丁寧なご指導をいただきました
石田 貴士准教授、ならびに大上 雅史助教に感謝の意を表します。

さらに、本研究を進めるにあたり秋山研究室・関嶋研究室・石田研究室合同ゼミを通して物理化学的な背景を含めた細かく的確なアドバイスを頂いた関嶋 政和准教授に御礼申し上げます。

最後に、本研究を行うにあたり秋山研究室・関嶋研究室・石田研究室の皆様には多大なるご協力を賜りました。暖かく、時には厳しいご指摘を通してご支援いただきましたことを心より感謝いたします。


\end{syaji}

\clearpage


%==============[ 参考文献 ]============================================================================
\bibliography{ref.bib}
%\begin{thebibliography}{}
%\small
%\bibitem{hoge} ほげほげ
%\end{thebibliography}


%==============[ 付録 ]================================================================================
\newpage
\appendix
\appendix
\chapter{DUD-Eの詳細}\label{appendix:dude}
\begin{table}[htb] \centering
	\caption{DUD-Eの詳細}
	\label{tb:dude_description}
	\begin{tabular}{c|c|c|rr|rr|}
	\multirow{2}{*}{ターゲット名}	&\multirow{2}{*}{PDBID}	&\multirow{2}{*}{タンパク質詳細}	&\multicolumn{2}{c|}{正例}	&\multicolumn{2}{c|}{負例}	\\
							&					&							&化合物数	&平均分割数	&化合物数	&平均分割数	\\ \hline
	aa2ar					&3EML				&Adenosine A2a receptor		&nnn		&nnn		&nnn		&nnn		\\
	kif11						&					&							&			&			&			&			\\ \hline
	\end{tabular}
\end{table}

\chapter{ROC曲線}\label{appendix:roc}
DUD-Eの102ターゲットそれぞれについて、7通りの手法のROC曲線を記載する。
図\ref{fig:roc:1}のようなものがターゲットを変えながら102個並ぶ。
2×3=6が1ページで、それが18ページ続く形になることを想像している。

\begin{figure}[tb]
 \begin{minipage}{0.5\hsize}
  \begin{center}
   \fig[width=0.9\hsize]{./fig/appendix/aa2ar_gscore_ROC.pdf}
  \end{center}
 \end{minipage}
 \begin{minipage}{0.5\hsize}
  \begin{center}
   \fig[width=0.9\hsize]{./fig/appendix/abl1_gscore_ROC.pdf}
  \end{center}
 \end{minipage}
 \begin{minipage}{0.5\hsize}
  \begin{center}
   \fig[width=0.9\hsize]{./fig/appendix/ace_gscore_ROC.pdf}
  \end{center}
 \end{minipage}
 \begin{minipage}{0.5\hsize}
  \begin{center}
   \fig[width=0.9\hsize]{./fig/appendix/aces_gscore_ROC.pdf}
  \end{center}
 \end{minipage}
 \begin{minipage}{0.5\hsize}
  \begin{center}
   \fig[width=0.9\hsize]{./fig/appendix/ada_gscore_ROC.pdf}
  \end{center}
 \end{minipage}
 \begin{minipage}{0.5\hsize}
  \begin{center}
   \fig[width=0.9\hsize]{./fig/appendix/ada17_gscore_ROC.pdf}
  \end{center}
 \end{minipage}
  \caption{ROC曲線例}
  \label{fig:roc:1}
\end{figure}

\clearpage
\end{document}
