\chapter{考察}

\section{score\_sum手法におけるフラグメント数に対するペナルティ}
今回、提案手法では重原子(水素以外の原子)の個数が2つ以下であるフラグメントはフィルタリングスコア算出から除外した。
これは、フラグメントの結合スコアの和を用いるscore\_sum手法において、図\ref{fig:no_omit_score_graph}のように、
化合物に属するフラグメントが多ければ多いほど同じ原子数であってもフィルタリングスコアが向上してしまうという現象が得られたためである。
\begin{figure}[bt]
 \begin{center}
  \fig[width=0.7\hsize]{./fig/discussion/no_omit_score.eps}
  \caption{ターゲットfntaの全てのligand, decoyのうち重原子数32の化合物のフラグメントスコアの単純加算}
  \label{fig:no_omit_score_graph}
 \end{center}
\end{figure}
この現象は最適化問題の条件緩和と考えることで説明できる。すなわち、本来化合物には原子間の結合距離という拘束条件が存在している。
フラグメント分割によって切断された原子間の結合は距離を考えずにスコア付けして良いので、分割は原子間の結合という拘束条件を
一つずつ緩和することに対応する。このため、フラグメント分割がされればされるほどスコアが良くなってしまうのである。

この現象を改善するため、本研究では小さなフラグメントの無視、およびフラグメント数に対して線形なペナルティの付加を実験した。
\begin{description}
\item[小さなフラグメントの無視] 提案手法として用いている手法である。分割が多ければ多いほど小さなフラグメントが発生するため、
	小さなフラグメントの結合スコアを無視することで事実上のフラグメント数に対するペナルティとなっている。
\item[フラグメント数に対する線形ペナルティ]
	図\ref{fig:no_omit_score_graph}を見ると、フィルタリングスコアの平均とフラグメント数との関係は線形に近い。このことから、
	フラグメント数に対して線形なペナルティを課すことでフラグメント数に依存しないフィルタリングスコアとなることが想定される。
\end{description}
この2つのペナルティ手法を個別に利用した場合の予測精度は表\ref{table:omit}および表\ref{table:penalty}のようになり、
総合すると本研究で提案している重原子数2以下のフラグメントの結合スコアを無視することが最も精度を高めている。
\todo{CAUTION! 重原子数3以下のフラグメントの結合スコアを無視することが最も精度を高めているので、本論でもそのように修正!!}

\begin{table}[htb] \centering
	\caption{小さなフラグメントを無視することによる精度の変化}
	\label{table:omit}
	\begin{tabular}{c|rrrrr|}
	\multirow{2}{*}{無視するフラグメントのサイズ}	&\multirow{2}{*}{ROC-AUC}	&\multicolumn{4}{c|}{Enrichment Factor}				\\
										&						&EF(1\%)		&EF(2\%)		&EF(5\%)		&EF(10\%)	\\ \hline
	全てのフラグメントを利用					&0.545					&3.46		&2.76		&2.01		&1.63		\\
	重原子数1							&0.557					&2.38		&2.16		&1.85		&1.66		\\
	重原子数2以下						&0.624					&5.08		&4.14		&3.02		&2.34		\\
	重原子数3以下						&0.634					&5.75		&4.34		&3.03		&2.49		\\
	重原子数4以下						&0.620					&4.27		&3.43		&2.79		&2.32		\\
	重原子数5以下						&0.614					&4.43		&3.68		&2.75		&2.13		\\
	重原子数6以下						&0.537					&2.20		&1.86		&1.53		&1.43		\\ \hline
	\end{tabular}
\end{table}
\begin{table}[htb] \centering
	\caption{フラグメント数に対する線形ペナルティによる精度の変化}
	\label{table:penalty}
	\begin{tabular}{c|rrrrr|}
	フラグメント1つあたりの	&\multirow{2}{*}{ROC-AUC}	&\multicolumn{4}{c|}{Enrichment Factor}				\\
	ペナルティ$c$			&						&EF(1\%)		&EF(2\%)		&EF(5\%)		&EF(10\%)	\\ \hline
	ペナルティなし			&0.545					&3.46		&2.76		&2.01		&1.63		\\
	$c=1$				&0.559					&3.81		&2.98		&2.18		&1.77		\\
	$c=2$				&0.586					&4.70		&3.65		&2.66		&2.08		\\
	$c=3$				&0.622					&5.03		&4.03		&2.86		&2.32		\\
	$c=4$				&0.588					&3.80		&3.19		&2.51		&2.14		\\
	$c=5$				&0.549					&3.57		&2.96		&2.20		&1.78		\\ \hline
	\end{tabular}
\end{table}


\todo{score\_sum手法における評価を実施。何もしなかった場合と、ペナルティの重さを変えながら実験を繰り返す。無視するサイズを1~4、
	線形ペナルティは1~5の1刻みの合計10通りの結果を示す。}

\section{提案手法の得手・不得手の調査}
\ref{subsec:single_accuracy}節の実験結果より、提案手法は従来手法に比べて平均的に見れば精度が低調に終わることが判明した。
しかし、一部のターゲットに関しては提案手法が従来手法である glide HTVSモードに優っており(表\ref{table:target_accuracy_good})、
この理由がわかればどのようなケースにおいて提案手法によるフィルタリングを用いるべきかを明示的にすることができる。
同様に提案手法が従来手法より明らかに悪いケース(表\ref{table:target_accuracy_bad})に関して
原因が判明すれば、今後の提案手法の改善につながる。

\begin{table}[htb] \centering
	\caption{提案手法が上手く行ったケース}
	\label{table:target_accuracy_good}
	提案手法(score\_sum、score\_max、maxsumBSのいずれか)が従来手法(glide HTVSモード)\\
	よりもROC-AUCで0.2以上上回ったケースについて、ROC-AUCの差の降順で示している。
	\begin{tabular}{lc|r|rr|}
	\multirow{2}{*}{提案手法の種類}	&\multirow{2}{*}{ターゲット名}	&\multirow{2}{*}{ROC-AUC差}	&\multicolumn{2}{c|}{ROC-AUC}	\\
								&						&							&従来手法	&提案手法		\\ \hline
	線形和(maxsumBS)			&mcr					&0.319						&0.466		&{\bf 0.785}		\\
	線形和(maxsumBS)			&akt1					&0.285						&0.539		&{\bf 0.824}		\\
	最良値(score\_max)			&kith					&0.272						&0.615		&{\bf 0.887}		\\
	最良値(score\_max)			&akt1					&0.265						&0.539		&{\bf 0.804}		\\
	最良値(score\_max)			&mcr					&0.257						&0.466		&{\bf 0.723}		\\
	線形和(maxsumBS)			&gcr						&0.252						&0.528		&{\bf 0.780}		\\
	最良値(score\_max)			&gcr						&0.242						&0.528		&{\bf 0.770}		\\
	総和(score\_sum)				&ital						&0.212						&0.529		&{\bf 0.741}		\\
	総和(score\_sum)				&akt1					&0.209						&0.539		&{\bf 0.748}		\\ \hline
	\end{tabular}
\end{table}
\begin{table}[htb] \centering
	\caption{提案手法が上手く行かないケース}
	\label{table:target_accuracy_bad}
	提案手法(score\_sum、score\_max、maxsumBSのいずれか)が従来手法(glide HTVSモード)\\
	よりもROC-AUCで-0.45以上下回ったケースについて、ROC-AUCの差の昇順で示している。
	\begin{tabular}{lc|r|rr|}
	\multirow{2}{*}{提案手法の種類}	&\multirow{2}{*}{ターゲット名}	&\multirow{2}{*}{ROC-AUC差}	&\multicolumn{2}{c|}{ROC-AUC}	\\
								&						&							&従来手法	&提案手法		\\ \hline
	最良値(score\_max)			&lkha4					&-0.572						&{\bf 0.880}	&0.308			\\
	最良値(score\_max)			&xiap					&-0.506						&{\bf 0.802}	&0.296			\\
	総和(score\_sum)				&def						&-0.495						&{\bf 0.733}	&0.238			\\
	総和(score\_sum)				&wee1					&-0.487						&{\bf 0.933}	&0.446			\\
	総和(score\_sum)				&hs90a					&-0.464						&{\bf 0.761}	&0.297			\\
	線形和(maxsumBS)			&hs90a					&-0.460						&{\bf 0.761}	&0.301			\\
	線形和(maxsumBS)			&xiap					&-0.460						&{\bf 0.802}	&0.342			\\ \hline
	\end{tabular}
\end{table}

\subsection{提案手法が得意なターゲット}
表\ref{target_accuracy_good}より、提案手法が従来手法であるglide HTVSモードに明らかに優っているケースは
akt1, gcr, ital, kith, mcr各ターゲットである。これらのターゲットについて、正例化合物、および負例化合物の分割数、

\todo{akt1, gcr, ital, kith, mcrの5種類について、ROC曲線を示し、フラグメント数などからの説明を試みる。}
\subsection{提案手法が不得意なターゲット}
\todo{def, hs90a, lkha4, wee1, xiapの5種類について、ROC曲線を示し、フラグメント数などからの説明を試みる。
タンパク質のポケットが大きいと苦手とか、そういう可能性はあるのかな・・?}

\memo{wee1, xiapは非常に柔らかい化合物がpositiveである模様。典型的なligandを表示して、cutした場合の例も示す。それによって、現在の手法ではスコアがどうしても低くなってしまうことを述べる。}


\section{提案手法のユースケース}
\ref{subsubsec:filtering_comparison}節の結果より、提案手法は精度より速度を重視したいケースにおいて有用であることは先に述べた。
ここではDUD-Eのターゲットの内、総化合物数が25,000以上である20ターゲットを用いて、フィルタリングにおける化合物の通過率がより少ない場合や、
さらに小さな割合におけるEFの評価を行い、大規模化合物ライブラリを利用する場合のユースケースを示す。
\todo{score\_sum, score\_max, maxsumBS, glide HTVSを用いたフィルタリングで0.5, 1, 2, 5\%に化合物を削減した場合について、
計算時間およびEF0.1, 0.2, 0.5, 1, 2\%を求め、結果を示し、この結果をもとに提案手法をどのように用いるのが有意義か、というユースケースを示す。}



%filtering時に何\%の化合物を残すか、などの議論と合わせて、精度と速度のバランスを示す。
%\comment{トレードオフを示すのについて、フィルタリングのパーセンテージをもっと振るべきでは?5, 10, 20, 30, 40, 50\%のように。(大上先生)}
%\begin{figure}[htp]
% \begin{center}
%  \fig[width=0.99\hsize]{./fig/result/性能vs時間.eps}
%  \caption{計算時間と精度のトレードオフ}
%  \label{fig:trade_off}
% \end{center}
%\end{figure}

%\todo{以下はメモ。どうまとめるか検討。}
%フラグメント数と統合スコアの関係
%
%\begin{itemize} 
%\item 提案手法では重原子数が2以下の小さなフラグメントを無視した
%\item これは、フラグメント分割数とスコアとの相関が強く出てしまうためである。
%\item 
%\end{itemize}
%
%
%ターゲットのactive ligandについて、何らかの考察を加えることで提案手法が向くタイプのターゲット、向かないタイプのターゲットを評価する。
%\begin{itemize}
%\item 演繹的な推論
%	\begin{itemize}
%	\item 総和法はフラグメント数と相関があるはず
%	\item 最大値法は化合物フラグメントの最大サイズに相関するはず
%	\end{itemize}
%\item 帰納的な推論
%	\begin{itemize}
%	\item glide SPでのスコア-化合物原子数の散布図と
%		glide HTVSでのスコア-化合物原子数の散布図、
%		提案手法でのスコア-化合物原子数の散布図の比較を元に何か言えないか調べてみる
%	\item glide HTVSと提案手法の精度比較をターゲット別に行い、どういうターゲットでは勝てるかどうかの議論
%	\end{itemize}
%\end{itemize}
