\chapter{考察}

\section{提案手法の得手・不得手}
ターゲットのactive ligandについて、何らかの考察を加えることで提案手法が向くタイプのターゲット、向かないタイプのターゲットを評価する。
\begin{itemize}
\item 演繹的な推論
	\begin{itemize}
	\item 総和法はフラグメント数と相関があるはず
	\item 最大値法は化合物フラグメントの最大サイズに相関するはず
	\end{itemize}
\item 帰納的な推論
	\begin{itemize}
	\item glide SPでのスコア(z-scored)-化合物原子数の散布図と
		glide HTVSでのスコア(z-scored)-化合物原子数の散布図、
		提案手法でのスコア(z-scored)-化合物原子数の散布図の比較を元に何か言えないか調べてみる
	\item glide HTVSと提案手法の精度比較をターゲット別に行い、どういうターゲットでは勝てるかどうかの議論
	\end{itemize}
\end{itemize}

\section{フラグメント数と統合スコアの関係}

\begin{itemize} 
\item 提案手法では重原子数が2以下の小さなフラグメントを無視した
\item これは、フラグメント分割数とスコアとの相関が強く出てしまうためである。
\item 
\end{itemize}


