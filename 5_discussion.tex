\chapter{考察}

\section{提案手法の得手・不得手}
ターゲットのactive ligandについて、何らかの考察を加えることで提案手法が向くタイプのターゲット、向かないタイプのターゲットを評価する。
\begin{itemize}
\item 演繹的な推論
	\begin{itemize}
	\item 総和法はフラグメント数と相関があるはず
	\item 最大値法は化合物フラグメントの最大サイズに相関するはず
	\end{itemize}
\item 帰納的な推論
	\begin{itemize}
	\item glide SPでのスコア(z-scored)-化合物原子数の散布図と
		glide HTVSでのスコア(z-scored)-化合物原子数の散布図、
		提案手法でのスコア(z-scored)-化合物原子数の散布図の比較を元に何か言えないか調べてみる
	\item glide HTVSと提案手法の精度比較をターゲット別に行い、どういうターゲットでは勝てるかどうかの議論
	\end{itemize}
\end{itemize}

\section{フラグメント数と統合スコアの関係}
\begin{itemize} 
\item 提案手法では重原子数が2以下の小さなフラグメントを無視した
\item 分割数とスコアとの相関が強く出てしまうため。
\item 
\end{itemize}


\section{フラグメントスコア計算式の改善}
\memo{以下はpending。DUD-Eすべてのターゲットを用いたらhbondやevdwのスコアの良さはなくなり、スコアの修正をする理由がなくなっている。}
\subsection{各種エネルギースコアの寄与率の評価}
hbond, evdw, ecoulなど、それぞれ別々に用いた場合の評価がどうなっているのか。

ターゲットの正解化合物の傾向との関係が見つけられればうれしいが。
\subsection{Glide GScoreのフラグメント向け修正}
フラグメントドッキングにおいてはgscoreをそのまま使うのではなく、少しパラメータを調整した方が良い。

公正な評価のために、通常のドッキングに関しても同様のパラメータチューニングをかける。

\begin{itemize}
\item 大きなDBに適用した場合の話をする(? その場合は実験の段において「一致率」による評価を行う)
\end{itemize}
