\chapter{考察}

\section{提案手法の得手・不得手}
ターゲットのactive ligandについて、何らかの考察を加えることで提案手法が向くタイプのターゲット、向かないタイプのターゲットを評価する。

\section{フラグメントスコア計算式の改善}
\subsection{各種エネルギースコアの寄与率の評価}
hbond, evdw, ecoulなど、それぞれ別々に用いた場合の評価がどうなっているのか。

ターゲットの正解化合物の傾向との関係が見つけられればうれしいが。
\subsection{Glide GScoreのフラグメント向け修正}
フラグメントドッキングにおいてはgscoreをそのまま使うのではなく、少しパラメータを調整した方が良い。

公正な評価のために、通常のドッキングに関しても同様のパラメータチューニングをかける。

\begin{itemize}
\item 大きなDBに適用した場合の話をする(? その場合は実験の段において「一致率」による評価を行う)
\end{itemize}
