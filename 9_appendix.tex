\appendix
\chapter{DUD-Eの詳細}\label{appendix:dude}
DUD-E(A Database of Useful Decoys: Enhanced)はMysingerらによって作成されたドッキングシミュレーションツールを評価するための
データセットである\cite{Mysinger2012}。ターゲットは多様性を考慮して102種類が選択されており、
それぞれに対して
\begin{itemize}
\item 代表タンパク質構造
\item 正例となる既知の薬剤・阻害剤
\item 負例となるデコイ(実験は行われていないが、既知の薬剤・阻害剤と構造が似ていないためターゲットを阻害しないと
	考えられる化合物)および実験によってターゲットを阻害しないことが知られている化合物
\end{itemize}
が用意されている。
\begin{table}[htb] \centering
	\caption{DUD-Eの詳細}
	\label{tb:dude_description}
	\begin{tabular}{c|c|p{5cm}|rr|rr}
	\hline
	\multirow{2}{*}{ターゲット名}	&\multirow{2}{*}{PDBID}	&\multicolumn{1}{c|}{\multirow{2}{*}{タンパク質詳細}}	&\multicolumn{2}{c|}{化合物数}	&\multicolumn{2}{c}{平均フラグメント数}		\\
							&					&											&正例	&負例				&正例	&負例						\\ \hline
	aa2ar					&3EML				&Adenosine A2a receptor						&482	&31,498				&6.26	&7.04						\\
	abl1						&2HZI				&Tyrosine-protein kinase ABL					&182	&10,746				&7.08	&7.33						\\ \hline
	\end{tabular}
\end{table}
\todo{以下作成中。スクリプト書いて完成させねば}

\chapter{各手法を単体で用いた場合のROC曲線}\label{appendix:roc}
総和法(score\_sum)、最良値法(score\_max)、総和法と最良値法の値の線形和(maxsumBS)の3つの提案手法
およびglide HTVSモードをそれぞれ単体で用いた場合のROC曲線を示す。
3つの提案手法についてはフラグメントの結合スコア算出をglide SPモード/glide HTVSモードで行った場合がそれぞれ示されている。
例えば、「score\_max\_SP」とはフラグメントの結合スコアをglide SPモードで求め、そのフラグメントスコアの最良値をとった場合の精度が
ROC曲線で示されている。
\memo{
DUD-Eの102ターゲットそれぞれについて、7通りの手法のROC曲線を記載する。
図\ref{fig:roc:1}のようなものがターゲットを変えながら102個並ぶ。
2×3=6が1ページで、それが18ページ続く形になることを想像している。
}
\begin{figure}[tb]
 \begin{minipage}{0.5\hsize}
  \begin{center}
   \fig[width=0.9\hsize]{./fig/appendix/aa2ar_gscore_ROC.pdf}
  \end{center}
 \end{minipage}
 \begin{minipage}{0.5\hsize}
  \begin{center}
   \fig[width=0.9\hsize]{./fig/appendix/abl1_gscore_ROC.pdf}
  \end{center}
 \end{minipage}
 \begin{minipage}{0.5\hsize}
  \begin{center}
   \fig[width=0.9\hsize]{./fig/appendix/ace_gscore_ROC.pdf}
  \end{center}
 \end{minipage}
 \begin{minipage}{0.5\hsize}
  \begin{center}
   \fig[width=0.9\hsize]{./fig/appendix/aces_gscore_ROC.pdf}
  \end{center}
 \end{minipage}
 \begin{minipage}{0.5\hsize}
  \begin{center}
   \fig[width=0.9\hsize]{./fig/appendix/ada_gscore_ROC.pdf}
  \end{center}
 \end{minipage}
 \begin{minipage}{0.5\hsize}
  \begin{center}
   \fig[width=0.9\hsize]{./fig/appendix/ada17_gscore_ROC.pdf}
  \end{center}
 \end{minipage}
  \caption{ROC曲線例}
  \label{fig:roc:1}
\end{figure}
