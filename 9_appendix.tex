\appendix
\chapter{DUD-Eの詳細}\label{appendix:dude}
DUD-E(A Database of Useful Decoys: Enhanced)はMysingerらによって作成されたドッキングシミュレーションツールを評価するための
データセットである\cite{Mysinger2012}.ターゲットは多様性を考慮して102種類が選択されており,
それぞれに対して
\begin{itemize}
\item 代表タンパク質構造
\item 正例となる既知の薬剤・阻害剤
\item 負例となるデコイ(実験は行われていないが,既知の薬剤・阻害剤と構造が似ていないためターゲットを阻害しないと
	考えられる化合物)および実験によってターゲットを阻害しないことが知られている化合物
\end{itemize}
が用意されている.

\newpage

\begin{table}[t] \centering
	\caption{DUD-Eの詳細(1)}
	\label{tb:dude_description:1}
	\begin{tabular}{c|c|p{6cm}|rr|rr}
	\hline
	\multirow{2}{*}{ターゲット}	&\multirow{2}{*}{PDBID}	&\multicolumn{1}{c|}{\multirow{2}{*}{タンパク質詳細}}	&\multicolumn{2}{c|}{化合物数}	&\multicolumn{2}{c}{平均フラグメント数}		\\
							&					&											&正例	&負例				&正例	&負例						\\ \hline
aa2ar&3EML&Adenosine A2a receptor&482&31,498&6.26&7.04 \\
abl1&2HZI&Tyrosine-protein kinase ABL&182&10,746&7.08&7.33 \\
ace&3BKL&Angiotensin-converting enzyme&282&16,860&9.90&7.52 \\
aces&1E66&Acetylcholinesterase&453&26,233&9.48&8.07 \\
ada&2E1W&Adenosine deaminase&93&5,449&8.16&8.01 \\
ada17&2OI0&ADAM17&532&35,809&8.81&8.32 \\
adrb1&2VT4&Beta-1 adrenergic receptor&247&15,842&10.05&9.62 \\
adrb2&3NY8&Beta-2 adrenergic receptor&231&14,993&10.35&9.72 \\
akt1&3CQW&Serine / threonine-protein kinase AKT&293&16,426&6.94&8.00 \\
akt2&3D0E&Serine / threonine-protein kinase AKT2&117&6,893&6.61&7.39 \\
aldr&2HV5&Aldose reductase&159&8,995&4.81&5.34 \\
ampc&1L2S&Beta-lactamase&48&2,832&4.94&4.95 \\
andr&2AM9&Androgen Receptor&269&14,343&3.91&4.67 \\
aofb&1S3B&Monoamine oxidase B&122&6,900&4.37&4.39 \\
bace1&3L5D&Beta-secretase 1&283&18,080&10.72&8.79 \\
braf&3D4Q&Serine / threonine-protein kinase B-raf&152&9,942&7.27&7.24 \\
cah2&1BCD&Carbonic anhydrase II&492&31,133&7.14&7.10 \\
casp3&2CNK&Caspase-3&199&10,692&10.47&8.72 \\
cdk2&1H00&Cyclin-dependent kinase 2&474&27,830&6.25&6.70 \\
comt&3BWM&Catechol O-methyltransferase&41&3,848&4.85&5.15 \\
cp2c9&1R9O&Cytochrome P450 2C9&120&7,446&7.18&6.78 \\
cp3a4&3NXU&Cytochrome P450 3A4&170&11,796&7.71&7.42 \\
csf1r&3KRJ&Macrophage colony stimulating factor receptor&166&12,144&6.86&6.92 \\
cxcr4&3ODU&C-X-C chemokine receptor type 4&40&3,406&6.75&7.30 \\
def&1LRU&Peptide deformylase&102&5,696&9.78&7.77 \\
dhi1&3FRJ&11-beta-hydroxysteroid dehydrogenase 1&330&19,340&5.89&5.54 \\ 
dpp4&2I78&Dipeptidyl peptidase IV&533&40,916&6.52&6.44 \\
\hline
	\end{tabular}
\end{table}
\begin{table}[t] \centering
	\caption{DUD-Eの詳細(2)}
	\label{tb:dude_description:2}
	\begin{tabular}{c|c|p{6cm}|rr|rr}
	\hline
	\multirow{2}{*}{ターゲット}	&\multirow{2}{*}{PDBID}	&\multicolumn{1}{c|}{\multirow{2}{*}{タンパク質詳細}}	&\multicolumn{2}{c|}{化合物数}	&\multicolumn{2}{c}{平均フラグメント数}		\\
							&					&											&正例	&負例				&正例	&負例						\\ \hline
drd3&3PBL&Dopamine D3 receptor&480&34,022&7.58&7.30 \\
dyr&3NXO&Dihydrofolate reductase&231&17,170&6.68&7.10 \\
egfr&2RGP&Epidermal growth factor receptor erbB1&542&35,020&7.27&7.74 \\
esr1&1SJ0&Estrogen receptor alpha&383&20,663&5.55&6.72 \\
esr2&2FSZ&Estrogen receptor beta&367&20,182&5.21&6.56 \\
fa10&3KL6&Coagulation factor X&537&20,023&9.01&8.18 \\
fa7&1W7X&Coagulation factor VII&114&6,245&10.29&8.26 \\
fabp4&2NNQ&Fatty acid binding protein adipocyte&47&2,749&6.68&6.53 \\
fak1&3BZ3&Focal adhesion kinase 1&100&5,350&8.44&8.12 \\
fgfr1&3C4F&Fibroblast growth factor receptor 1&139&333&7.33&7.55 \\
fkb1a&1J4H&FK506-binding protein 1A&111&5,800&9.75&8.42 \\
fnta&3E37&Protein farnesyltransferase / geranylgeranyltransferase type I alpha subunit&592&51,430&8.20&7.65 \\
fpps&1ZW5&Farnesyl diphosphate synthase&85&8,822&7.08&7.01 \\
gcr&3BQD&Glucocorticoid receptor&258&14,987&5.33&5.94 \\
glcm&2V3F&Beta-glucocerebrosidase&54&3,799&8.57&7.93 \\
gria2&3KGC&Glutamate receptor ionotropic, AMPA 2&158&11,832&6.47&6.52 \\
grik1&1VSO&Glutamate receptor ionotropic kainate 1&101&6,547&5.83&6.32 \\
hdac2&3MAX&Histone deacetylase 2&185&10,299&10.02&8.29 \\
hdac8&3F07&Histone deacetylase 8&170&10,448&9.52&7.95 \\
hivint&3NF7&Human immunodeficiency virus type 1 integrase&100&6,644&6.42&6.35 \\
hivpr&1XL2&Human immunodeficiency virus type 1 protease&536&35,688&11.16&8.75 \\
hivrt&3LAN&Human immunodeficiency virus type 1 reverse transcriptase&338&18,879&4.98&5.65 \\
hmdh&3CCW&HMG-CoA reductase&170&8,743&9.79&8.56 \\
\hline
	\end{tabular}
\end{table}
\begin{table}[t] \centering
	\caption{DUD-Eの詳細(3)}
	\label{tb:dude_description:3}
	\begin{tabular}{c|c|p{6cm}|rr|rr}
	\hline
	\multirow{2}{*}{ターゲット}	&\multirow{2}{*}{PDBID}	&\multicolumn{1}{c|}{\multirow{2}{*}{タンパク質詳細}}	&\multicolumn{2}{c|}{化合物数}	&\multicolumn{2}{c}{平均フラグメント数}		\\
							&					&											&正例	&負例				&正例	&負例						\\ \hline
hs90a&1UYG&Heat shock protein HSP 90-alpha&88&4,848&5.56&7.20 \\
hxk4&3F9M&Hexokinase type IV&92&4,696&6.78&6.97 \\
igf1r&2OJ9&Insulin-like growth factor I receptor&148&9,291&7.95&8.27 \\
inha&2H7L&Enoyl-[acyl-carrier-protein] reductase&43&2,300&7.14&6.17 \\
ital&2ICA&Leukocyte adhesion glycoprotein LFA-1 alpha&138&8,487&7.72&7.62 \\
jak2&3LPB&Tyrosine-protein kinase JAK2&107&6,495&6.23&6.63 \\
kif11&3CJO&Kinesin-like protein 1&116&6,848&6.78&6.18 \\
kit&3G0E&Stem cell growth factor receptor&166&10,447&7.80&7.57 \\
kith&2B8T&Thymidine kinase&57&2,850&6.70&7.60 \\
kpcb&2I0E&Protein kinase C beta&135&8,692&5.67&6.85 \\
lck&2OF2&Tyrosine-protein kinase LCK&420&27,374&7.20&7.32 \\
lkha4&3CHP&Leukotriene A4 hydrolase&171&9,448&7.70&7.59 \\
mapk2&3M2W&MAP kinase-activated protein kinase 2&101&6,147&4.67&5.58 \\
mcr&2AA2&Mineralocorticoid receptor&94&5,146&5.43&5.91 \\
met&3LQ8&Hepatocyte growth factor receptor&166&11,240&7.45&7.48 \\
mk01&2OJG&MAP kinase ERK2&79&4,548&7.25&6.84 \\
mk10&2ZDT&c-Jun N-terminal kinase 3&104&6,599&6.63&6.71 \\
mk14&2QD9&MAP kinase p38 alpha&578&35,810&7.31&7.03 \\
mmp13&830C&Matrix metalloproteinase 13&572&37,126&8.98&8.29 \\
mp2k1&3EQH&Dual specificity mitogen-activated protein kinase kinase 1&121&8,147&6.92&8.02 \\
nos1&1QW6&Nitric-oxide synthase, brain&100&8,050&4.86&5.50 \\
nram&1B9V&Neuraminidase&98&6,199&8.39&7.01 \\
pa2ga&1KVO&Phospholipase A2 group IIA&99&5,146&10.12&9.63 \\
parp1&3L3M&Poly [ADP-ribose] polymerase-1&508&30,035&5.07&5.57 \\
pde5a&1UDT&Phosphodiesterase 5A&398&27,521&6.95&7.63 \\
pgh1&2OYU&Cyclooxygenase-1&195&10,797&4.66&4.72 \\
pgh2&3LN1&Cyclooxygenase-2&435&23,135&5.14&5.28 \\
\hline
	\end{tabular}
\end{table}
\begin{table}[t] \centering
	\caption{DUD-Eの詳細(4)}
	\label{tb:dude_description:4}
	\begin{tabular}{c|c|p{6cm}|rr|rr}
	\hline
	\multirow{2}{*}{ターゲット}	&\multirow{2}{*}{PDBID}	&\multicolumn{1}{c|}{\multirow{2}{*}{タンパク質詳細}}	&\multicolumn{2}{c|}{化合物数}	&\multicolumn{2}{c}{平均フラグメント数}		\\
							&					&											&正例	&負例				&正例	&負例						\\ \hline
plk1&2OWB&Serine / threonine-protein kinase PLK1&107&6,797&7.16&7.95 \\
pnph&3BGS&Purine nucleoside phosphorylase&103&6,950&3.93&6.08 \\
ppara&2P54&Peroxisome proliferator-activated receptor alpha&373&19,356&9.75&8.98 \\
ppard&2ZNP&Peroxisome proliferator-activated receptor delta&240&12,223&8.95&8.69 \\
pparg&2GTK&Peroxisome proliferator-activated receptor gamma&484&25,256&9.19&8.61 \\
prgr&3KBA&Progesterone receptor&293&15,642&3.95&4.65 \\
ptn1&2AZR&Protein-tyrosine phosphatase 1B&130&7,243&8.98&7.62 \\
pur2&1NJS&GAR transformylase&50&2,694&12.18&8.54 \\
pygm&1C8K&Muscle glycogen phosphorylase&77&3,940&6.74&6.19 \\
pyrd&1D3G&Dihydroorotate dehydrogenase&111&6,446&5.66&5.64 \\
reni&3G6Z&Renin&104&6,955&14.59&11.68 \\
rock1&2ETR&Rho-associated protein kinase 1&100&6,297&5.54&6.30 \\
rxra&1MV9&Retinoid X receptor alpha&131&6,935&5.95&5.79 \\
sahh&1LI4&Adenosylhomocysteinase&63&3,450&3.46&5.67 \\
src&3EL8&Tyrosine-protein kinase SRC&524&34,454&7.18&7.61 \\
tgfr1&3HMM&TGF-beta receptor type I&133&8,498&5.02&5.66 \\
thb&1Q4X&Thyroid hormone receptor beta-1&103&7,441&6.52&7.19 \\
thrb&1YPE&Thrombin&461&26,948&11.09&8.75 \\
try1&2AYW&Trypsin I&449&25,914&10.42&8.34 \\
tryb1&2ZEC&Tryptase beta-1&148&7,643&11.55&9.19 \\
tysy&1SYN&Thymidylate synthase&109&6,738&8.57&7.14 \\
urok&1SQT&Urokinase-type plasminogen activator&162&9,841&7.29&6.88 \\
vgfr2&2P2I&Vascular endothelial growth factor receptor 2&409&24,927&7.23&7.37 \\
wee1&3BIZ&Serine / threonine-protein kinase WEE1&102&6,148&5.87&7.57 \\
xiap&3HL5&Inhibitor of apoptosis protein 3&100&5,145&11.89&8.87 \\ \hline
	\end{tabular}
\end{table}

\chapter{各手法を単体で用いた場合のROC曲線}\label{appendix:roc}
\ref{subsec:single_accuracy}節にて示した
総和法(score\_sum),最良値法(score\_max),総和法と最良値法の値の線形和(maxsumBS)の3つの提案手法
およびGlide HTVSモードをそれぞれ単体で用いた場合のROC曲線を示す.
3つの提案手法についてはフラグメントの結合スコア算出をGlide SPモード / Glide HTVSモードで行った場合がそれぞれ示されている.
例えば,「score\_max\_SP」とはフラグメントの結合スコアをGlide SPモードで求め,そのフラグメントスコアの最良値をとった場合の精度が
ROC曲線で示されている.


        \begin{figure}[p]
        
     \begin{minipage}{0.49\hsize}
      \begin{center}
       \fig[width=0.9\hsize]{./fig/appendix/aa2ar_gscore_ROC.pdf}
      \end{center}
     \end{minipage}    
     \begin{minipage}{0.49\hsize}
      \begin{center}
       \fig[width=0.9\hsize]{./fig/appendix/abl1_gscore_ROC.pdf}
      \end{center}
     \end{minipage}    
     \begin{minipage}{0.49\hsize}
      \begin{center}
       \fig[width=0.9\hsize]{./fig/appendix/ace_gscore_ROC.pdf}
      \end{center}
     \end{minipage}    
     \begin{minipage}{0.49\hsize}
      \begin{center}
       \fig[width=0.9\hsize]{./fig/appendix/aces_gscore_ROC.pdf}
      \end{center}
     \end{minipage}    
     \begin{minipage}{0.49\hsize}
      \begin{center}
       \fig[width=0.9\hsize]{./fig/appendix/ada_gscore_ROC.pdf}
      \end{center}
     \end{minipage}    
     \begin{minipage}{0.49\hsize}
      \begin{center}
       \fig[width=0.9\hsize]{./fig/appendix/ada17_gscore_ROC.pdf}
      \end{center}
     \end{minipage}    
         \caption{各手法の単体性能ROC曲線(1)}
         \label{fig:roc:1}
        \end{figure}
        
        \begin{figure}[p]
        
     \begin{minipage}{0.49\hsize}
      \begin{center}
       \fig[width=0.9\hsize]{./fig/appendix/adrb1_gscore_ROC.pdf}
      \end{center}
     \end{minipage}    
     \begin{minipage}{0.49\hsize}
      \begin{center}
       \fig[width=0.9\hsize]{./fig/appendix/adrb2_gscore_ROC.pdf}
      \end{center}
     \end{minipage}    
     \begin{minipage}{0.49\hsize}
      \begin{center}
       \fig[width=0.9\hsize]{./fig/appendix/akt1_gscore_ROC.pdf}
      \end{center}
     \end{minipage}    
     \begin{minipage}{0.49\hsize}
      \begin{center}
       \fig[width=0.9\hsize]{./fig/appendix/akt2_gscore_ROC.pdf}
      \end{center}
     \end{minipage}    
     \begin{minipage}{0.49\hsize}
      \begin{center}
       \fig[width=0.9\hsize]{./fig/appendix/aldr_gscore_ROC.pdf}
      \end{center}
     \end{minipage}    
     \begin{minipage}{0.49\hsize}
      \begin{center}
       \fig[width=0.9\hsize]{./fig/appendix/ampc_gscore_ROC.pdf}
      \end{center}
     \end{minipage}    
         \caption{各手法の単体性能ROC曲線(2)}
         \label{fig:roc:2}
        \end{figure}
        
        \begin{figure}[p]
        
     \begin{minipage}{0.49\hsize}
      \begin{center}
       \fig[width=0.9\hsize]{./fig/appendix/andr_gscore_ROC.pdf}
      \end{center}
     \end{minipage}    
     \begin{minipage}{0.49\hsize}
      \begin{center}
       \fig[width=0.9\hsize]{./fig/appendix/aofb_gscore_ROC.pdf}
      \end{center}
     \end{minipage}    
     \begin{minipage}{0.49\hsize}
      \begin{center}
       \fig[width=0.9\hsize]{./fig/appendix/bace1_gscore_ROC.pdf}
      \end{center}
     \end{minipage}    
     \begin{minipage}{0.49\hsize}
      \begin{center}
       \fig[width=0.9\hsize]{./fig/appendix/braf_gscore_ROC.pdf}
      \end{center}
     \end{minipage}    
     \begin{minipage}{0.49\hsize}
      \begin{center}
       \fig[width=0.9\hsize]{./fig/appendix/cah2_gscore_ROC.pdf}
      \end{center}
     \end{minipage}    
     \begin{minipage}{0.49\hsize}
      \begin{center}
       \fig[width=0.9\hsize]{./fig/appendix/casp3_gscore_ROC.pdf}
      \end{center}
     \end{minipage}    
         \caption{各手法の単体性能ROC曲線(3)}
         \label{fig:roc:3}
        \end{figure}
        
        \begin{figure}[p]
        
     \begin{minipage}{0.49\hsize}
      \begin{center}
       \fig[width=0.9\hsize]{./fig/appendix/cdk2_gscore_ROC.pdf}
      \end{center}
     \end{minipage}    
     \begin{minipage}{0.49\hsize}
      \begin{center}
       \fig[width=0.9\hsize]{./fig/appendix/comt_gscore_ROC.pdf}
      \end{center}
     \end{minipage}    
     \begin{minipage}{0.49\hsize}
      \begin{center}
       \fig[width=0.9\hsize]{./fig/appendix/cp2c9_gscore_ROC.pdf}
      \end{center}
     \end{minipage}    
     \begin{minipage}{0.49\hsize}
      \begin{center}
       \fig[width=0.9\hsize]{./fig/appendix/cp3a4_gscore_ROC.pdf}
      \end{center}
     \end{minipage}    
     \begin{minipage}{0.49\hsize}
      \begin{center}
       \fig[width=0.9\hsize]{./fig/appendix/csf1r_gscore_ROC.pdf}
      \end{center}
     \end{minipage}    
     \begin{minipage}{0.49\hsize}
      \begin{center}
       \fig[width=0.9\hsize]{./fig/appendix/cxcr4_gscore_ROC.pdf}
      \end{center}
     \end{minipage}    
         \caption{各手法の単体性能ROC曲線(4)}
         \label{fig:roc:4}
        \end{figure}
        
        \begin{figure}[p]
        
     \begin{minipage}{0.49\hsize}
      \begin{center}
       \fig[width=0.9\hsize]{./fig/appendix/def_gscore_ROC.pdf}
      \end{center}
     \end{minipage}    
     \begin{minipage}{0.49\hsize}
      \begin{center}
       \fig[width=0.9\hsize]{./fig/appendix/dhi1_gscore_ROC.pdf}
      \end{center}
     \end{minipage}    
     \begin{minipage}{0.49\hsize}
      \begin{center}
       \fig[width=0.9\hsize]{./fig/appendix/dpp4_gscore_ROC.pdf}
      \end{center}
     \end{minipage}    
     \begin{minipage}{0.49\hsize}
      \begin{center}
       \fig[width=0.9\hsize]{./fig/appendix/drd3_gscore_ROC.pdf}
      \end{center}
     \end{minipage}    
     \begin{minipage}{0.49\hsize}
      \begin{center}
       \fig[width=0.9\hsize]{./fig/appendix/dyr_gscore_ROC.pdf}
      \end{center}
     \end{minipage}    
     \begin{minipage}{0.49\hsize}
      \begin{center}
       \fig[width=0.9\hsize]{./fig/appendix/egfr_gscore_ROC.pdf}
      \end{center}
     \end{minipage}    
         \caption{各手法の単体性能ROC曲線(5)}
         \label{fig:roc:5}
        \end{figure}
        
        \begin{figure}[p]
        
     \begin{minipage}{0.49\hsize}
      \begin{center}
       \fig[width=0.9\hsize]{./fig/appendix/esr1_gscore_ROC.pdf}
      \end{center}
     \end{minipage}    
     \begin{minipage}{0.49\hsize}
      \begin{center}
       \fig[width=0.9\hsize]{./fig/appendix/esr2_gscore_ROC.pdf}
      \end{center}
     \end{minipage}    
     \begin{minipage}{0.49\hsize}
      \begin{center}
       \fig[width=0.9\hsize]{./fig/appendix/fa10_gscore_ROC.pdf}
      \end{center}
     \end{minipage}    
     \begin{minipage}{0.49\hsize}
      \begin{center}
       \fig[width=0.9\hsize]{./fig/appendix/fa7_gscore_ROC.pdf}
      \end{center}
     \end{minipage}    
     \begin{minipage}{0.49\hsize}
      \begin{center}
       \fig[width=0.9\hsize]{./fig/appendix/fabp4_gscore_ROC.pdf}
      \end{center}
     \end{minipage}    
     \begin{minipage}{0.49\hsize}
      \begin{center}
       \fig[width=0.9\hsize]{./fig/appendix/fak1_gscore_ROC.pdf}
      \end{center}
     \end{minipage}    
         \caption{各手法の単体性能ROC曲線(6)}
         \label{fig:roc:6}
        \end{figure}
        
        \begin{figure}[p]
        
     \begin{minipage}{0.49\hsize}
      \begin{center}
       \fig[width=0.9\hsize]{./fig/appendix/fgfr1_gscore_ROC.pdf}
      \end{center}
     \end{minipage}    
     \begin{minipage}{0.49\hsize}
      \begin{center}
       \fig[width=0.9\hsize]{./fig/appendix/fkb1a_gscore_ROC.pdf}
      \end{center}
     \end{minipage}    
     \begin{minipage}{0.49\hsize}
      \begin{center}
       \fig[width=0.9\hsize]{./fig/appendix/fnta_gscore_ROC.pdf}
      \end{center}
     \end{minipage}    
     \begin{minipage}{0.49\hsize}
      \begin{center}
       \fig[width=0.9\hsize]{./fig/appendix/fpps_gscore_ROC.pdf}
      \end{center}
     \end{minipage}    
     \begin{minipage}{0.49\hsize}
      \begin{center}
       \fig[width=0.9\hsize]{./fig/appendix/gcr_gscore_ROC.pdf}
      \end{center}
     \end{minipage}    
     \begin{minipage}{0.49\hsize}
      \begin{center}
       \fig[width=0.9\hsize]{./fig/appendix/glcm_gscore_ROC.pdf}
      \end{center}
     \end{minipage}    
         \caption{各手法の単体性能ROC曲線(7)}
         \label{fig:roc:7}
        \end{figure}
        
        \begin{figure}[p]
        
     \begin{minipage}{0.49\hsize}
      \begin{center}
       \fig[width=0.9\hsize]{./fig/appendix/gria2_gscore_ROC.pdf}
      \end{center}
     \end{minipage}    
     \begin{minipage}{0.49\hsize}
      \begin{center}
       \fig[width=0.9\hsize]{./fig/appendix/grik1_gscore_ROC.pdf}
      \end{center}
     \end{minipage}    
     \begin{minipage}{0.49\hsize}
      \begin{center}
       \fig[width=0.9\hsize]{./fig/appendix/hdac2_gscore_ROC.pdf}
      \end{center}
     \end{minipage}    
     \begin{minipage}{0.49\hsize}
      \begin{center}
       \fig[width=0.9\hsize]{./fig/appendix/hdac8_gscore_ROC.pdf}
      \end{center}
     \end{minipage}    
     \begin{minipage}{0.49\hsize}
      \begin{center}
       \fig[width=0.9\hsize]{./fig/appendix/hivint_gscore_ROC.pdf}
      \end{center}
     \end{minipage}    
     \begin{minipage}{0.49\hsize}
      \begin{center}
       \fig[width=0.9\hsize]{./fig/appendix/hivpr_gscore_ROC.pdf}
      \end{center}
     \end{minipage}    
         \caption{各手法の単体性能ROC曲線(8)}
         \label{fig:roc:8}
        \end{figure}
        
        \begin{figure}[p]
        
     \begin{minipage}{0.49\hsize}
      \begin{center}
       \fig[width=0.9\hsize]{./fig/appendix/hivrt_gscore_ROC.pdf}
      \end{center}
     \end{minipage}    
     \begin{minipage}{0.49\hsize}
      \begin{center}
       \fig[width=0.9\hsize]{./fig/appendix/hmdh_gscore_ROC.pdf}
      \end{center}
     \end{minipage}    
     \begin{minipage}{0.49\hsize}
      \begin{center}
       \fig[width=0.9\hsize]{./fig/appendix/hs90a_gscore_ROC.pdf}
      \end{center}
     \end{minipage}    
     \begin{minipage}{0.49\hsize}
      \begin{center}
       \fig[width=0.9\hsize]{./fig/appendix/hxk4_gscore_ROC.pdf}
      \end{center}
     \end{minipage}    
     \begin{minipage}{0.49\hsize}
      \begin{center}
       \fig[width=0.9\hsize]{./fig/appendix/igf1r_gscore_ROC.pdf}
      \end{center}
     \end{minipage}    
     \begin{minipage}{0.49\hsize}
      \begin{center}
       \fig[width=0.9\hsize]{./fig/appendix/inha_gscore_ROC.pdf}
      \end{center}
     \end{minipage}    
         \caption{各手法の単体性能ROC曲線(9)}
         \label{fig:roc:9}
        \end{figure}
        
        \begin{figure}[p]
        
     \begin{minipage}{0.49\hsize}
      \begin{center}
       \fig[width=0.9\hsize]{./fig/appendix/ital_gscore_ROC.pdf}
      \end{center}
     \end{minipage}    
     \begin{minipage}{0.49\hsize}
      \begin{center}
       \fig[width=0.9\hsize]{./fig/appendix/jak2_gscore_ROC.pdf}
      \end{center}
     \end{minipage}    
     \begin{minipage}{0.49\hsize}
      \begin{center}
       \fig[width=0.9\hsize]{./fig/appendix/kif11_gscore_ROC.pdf}
      \end{center}
     \end{minipage}    
     \begin{minipage}{0.49\hsize}
      \begin{center}
       \fig[width=0.9\hsize]{./fig/appendix/kit_gscore_ROC.pdf}
      \end{center}
     \end{minipage}    
     \begin{minipage}{0.49\hsize}
      \begin{center}
       \fig[width=0.9\hsize]{./fig/appendix/kith_gscore_ROC.pdf}
      \end{center}
     \end{minipage}    
     \begin{minipage}{0.49\hsize}
      \begin{center}
       \fig[width=0.9\hsize]{./fig/appendix/kpcb_gscore_ROC.pdf}
      \end{center}
     \end{minipage}    
         \caption{各手法の単体性能ROC曲線(10)}
         \label{fig:roc:10}
        \end{figure}
        
        \begin{figure}[p]
        
     \begin{minipage}{0.49\hsize}
      \begin{center}
       \fig[width=0.9\hsize]{./fig/appendix/lck_gscore_ROC.pdf}
      \end{center}
     \end{minipage}    
     \begin{minipage}{0.49\hsize}
      \begin{center}
       \fig[width=0.9\hsize]{./fig/appendix/lkha4_gscore_ROC.pdf}
      \end{center}
     \end{minipage}    
     \begin{minipage}{0.49\hsize}
      \begin{center}
       \fig[width=0.9\hsize]{./fig/appendix/mapk2_gscore_ROC.pdf}
      \end{center}
     \end{minipage}    
     \begin{minipage}{0.49\hsize}
      \begin{center}
       \fig[width=0.9\hsize]{./fig/appendix/mcr_gscore_ROC.pdf}
      \end{center}
     \end{minipage}    
     \begin{minipage}{0.49\hsize}
      \begin{center}
       \fig[width=0.9\hsize]{./fig/appendix/met_gscore_ROC.pdf}
      \end{center}
     \end{minipage}    
     \begin{minipage}{0.49\hsize}
      \begin{center}
       \fig[width=0.9\hsize]{./fig/appendix/mk01_gscore_ROC.pdf}
      \end{center}
     \end{minipage}    
         \caption{各手法の単体性能ROC曲線(11)}
         \label{fig:roc:11}
        \end{figure}
        
        \begin{figure}[p]
        
     \begin{minipage}{0.49\hsize}
      \begin{center}
       \fig[width=0.9\hsize]{./fig/appendix/mk10_gscore_ROC.pdf}
      \end{center}
     \end{minipage}    
     \begin{minipage}{0.49\hsize}
      \begin{center}
       \fig[width=0.9\hsize]{./fig/appendix/mk14_gscore_ROC.pdf}
      \end{center}
     \end{minipage}    
     \begin{minipage}{0.49\hsize}
      \begin{center}
       \fig[width=0.9\hsize]{./fig/appendix/mmp13_gscore_ROC.pdf}
      \end{center}
     \end{minipage}    
     \begin{minipage}{0.49\hsize}
      \begin{center}
       \fig[width=0.9\hsize]{./fig/appendix/mp2k1_gscore_ROC.pdf}
      \end{center}
     \end{minipage}    
     \begin{minipage}{0.49\hsize}
      \begin{center}
       \fig[width=0.9\hsize]{./fig/appendix/nos1_gscore_ROC.pdf}
      \end{center}
     \end{minipage}    
     \begin{minipage}{0.49\hsize}
      \begin{center}
       \fig[width=0.9\hsize]{./fig/appendix/nram_gscore_ROC.pdf}
      \end{center}
     \end{minipage}    
         \caption{各手法の単体性能ROC曲線(12)}
         \label{fig:roc:12}
        \end{figure}
        
        \begin{figure}[p]
        
     \begin{minipage}{0.49\hsize}
      \begin{center}
       \fig[width=0.9\hsize]{./fig/appendix/pa2ga_gscore_ROC.pdf}
      \end{center}
     \end{minipage}    
     \begin{minipage}{0.49\hsize}
      \begin{center}
       \fig[width=0.9\hsize]{./fig/appendix/parp1_gscore_ROC.pdf}
      \end{center}
     \end{minipage}    
     \begin{minipage}{0.49\hsize}
      \begin{center}
       \fig[width=0.9\hsize]{./fig/appendix/pde5a_gscore_ROC.pdf}
      \end{center}
     \end{minipage}    
     \begin{minipage}{0.49\hsize}
      \begin{center}
       \fig[width=0.9\hsize]{./fig/appendix/pgh1_gscore_ROC.pdf}
      \end{center}
     \end{minipage}    
     \begin{minipage}{0.49\hsize}
      \begin{center}
       \fig[width=0.9\hsize]{./fig/appendix/pgh2_gscore_ROC.pdf}
      \end{center}
     \end{minipage}    
     \begin{minipage}{0.49\hsize}
      \begin{center}
       \fig[width=0.9\hsize]{./fig/appendix/plk1_gscore_ROC.pdf}
      \end{center}
     \end{minipage}    
         \caption{各手法の単体性能ROC曲線(13)}
         \label{fig:roc:13}
        \end{figure}
        
        \begin{figure}[p]
        
     \begin{minipage}{0.49\hsize}
      \begin{center}
       \fig[width=0.9\hsize]{./fig/appendix/pnph_gscore_ROC.pdf}
      \end{center}
     \end{minipage}    
     \begin{minipage}{0.49\hsize}
      \begin{center}
       \fig[width=0.9\hsize]{./fig/appendix/ppara_gscore_ROC.pdf}
      \end{center}
     \end{minipage}    
     \begin{minipage}{0.49\hsize}
      \begin{center}
       \fig[width=0.9\hsize]{./fig/appendix/ppard_gscore_ROC.pdf}
      \end{center}
     \end{minipage}    
     \begin{minipage}{0.49\hsize}
      \begin{center}
       \fig[width=0.9\hsize]{./fig/appendix/pparg_gscore_ROC.pdf}
      \end{center}
     \end{minipage}    
     \begin{minipage}{0.49\hsize}
      \begin{center}
       \fig[width=0.9\hsize]{./fig/appendix/prgr_gscore_ROC.pdf}
      \end{center}
     \end{minipage}    
     \begin{minipage}{0.49\hsize}
      \begin{center}
       \fig[width=0.9\hsize]{./fig/appendix/ptn1_gscore_ROC.pdf}
      \end{center}
     \end{minipage}    
         \caption{各手法の単体性能ROC曲線(14)}
         \label{fig:roc:14}
        \end{figure}
        
        \begin{figure}[p]
        
     \begin{minipage}{0.49\hsize}
      \begin{center}
       \fig[width=0.9\hsize]{./fig/appendix/pur2_gscore_ROC.pdf}
      \end{center}
     \end{minipage}    
     \begin{minipage}{0.49\hsize}
      \begin{center}
       \fig[width=0.9\hsize]{./fig/appendix/pygm_gscore_ROC.pdf}
      \end{center}
     \end{minipage}    
     \begin{minipage}{0.49\hsize}
      \begin{center}
       \fig[width=0.9\hsize]{./fig/appendix/pyrd_gscore_ROC.pdf}
      \end{center}
     \end{minipage}    
     \begin{minipage}{0.49\hsize}
      \begin{center}
       \fig[width=0.9\hsize]{./fig/appendix/reni_gscore_ROC.pdf}
      \end{center}
     \end{minipage}    
     \begin{minipage}{0.49\hsize}
      \begin{center}
       \fig[width=0.9\hsize]{./fig/appendix/rock1_gscore_ROC.pdf}
      \end{center}
     \end{minipage}    
     \begin{minipage}{0.49\hsize}
      \begin{center}
       \fig[width=0.9\hsize]{./fig/appendix/rxra_gscore_ROC.pdf}
      \end{center}
     \end{minipage}    
         \caption{各手法の単体性能ROC曲線(15)}
         \label{fig:roc:15}
        \end{figure}
        
        \begin{figure}[p]
        
     \begin{minipage}{0.49\hsize}
      \begin{center}
       \fig[width=0.9\hsize]{./fig/appendix/sahh_gscore_ROC.pdf}
      \end{center}
     \end{minipage}    
     \begin{minipage}{0.49\hsize}
      \begin{center}
       \fig[width=0.9\hsize]{./fig/appendix/src_gscore_ROC.pdf}
      \end{center}
     \end{minipage}    
     \begin{minipage}{0.49\hsize}
      \begin{center}
       \fig[width=0.9\hsize]{./fig/appendix/tgfr1_gscore_ROC.pdf}
      \end{center}
     \end{minipage}    
     \begin{minipage}{0.49\hsize}
      \begin{center}
       \fig[width=0.9\hsize]{./fig/appendix/thb_gscore_ROC.pdf}
      \end{center}
     \end{minipage}    
     \begin{minipage}{0.49\hsize}
      \begin{center}
       \fig[width=0.9\hsize]{./fig/appendix/thrb_gscore_ROC.pdf}
      \end{center}
     \end{minipage}    
     \begin{minipage}{0.49\hsize}
      \begin{center}
       \fig[width=0.9\hsize]{./fig/appendix/try1_gscore_ROC.pdf}
      \end{center}
     \end{minipage}    
         \caption{各手法の単体性能ROC曲線(16)}
         \label{fig:roc:16}
        \end{figure}
        
        \begin{figure}[p]
        
     \begin{minipage}{0.49\hsize}
      \begin{center}
       \fig[width=0.9\hsize]{./fig/appendix/tryb1_gscore_ROC.pdf}
      \end{center}
     \end{minipage}    
     \begin{minipage}{0.49\hsize}
      \begin{center}
       \fig[width=0.9\hsize]{./fig/appendix/tysy_gscore_ROC.pdf}
      \end{center}
     \end{minipage}    
     \begin{minipage}{0.49\hsize}
      \begin{center}
       \fig[width=0.9\hsize]{./fig/appendix/urok_gscore_ROC.pdf}
      \end{center}
     \end{minipage}    
     \begin{minipage}{0.49\hsize}
      \begin{center}
       \fig[width=0.9\hsize]{./fig/appendix/vgfr2_gscore_ROC.pdf}
      \end{center}
     \end{minipage}    
     \begin{minipage}{0.49\hsize}
      \begin{center}
       \fig[width=0.9\hsize]{./fig/appendix/wee1_gscore_ROC.pdf}
      \end{center}
     \end{minipage}    
     \begin{minipage}{0.49\hsize}
      \begin{center}
       \fig[width=0.9\hsize]{./fig/appendix/xiap_gscore_ROC.pdf}
      \end{center}
     \end{minipage}    
         \caption{各手法の単体性能ROC曲線(17)}
         \label{fig:roc:17}
        \end{figure}
        