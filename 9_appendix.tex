\appendix
\chapter{DUD-Eの詳細}\label{appendix:dude}
\begin{table}[htb] \centering
	\caption{DUD-Eの詳細}
	\label{tb:dude_description}
	\begin{tabular}{c|c|c|rr|rr|}
	\multirow{2}{*}{ターゲット名}	&\multirow{2}{*}{PDBID}	&\multirow{2}{*}{タンパク質詳細}	&\multicolumn{2}{c|}{正例}	&\multicolumn{2}{c|}{負例}	\\
							&					&							&化合物数	&平均分割数	&化合物数	&平均分割数	\\ \hline
	aa2ar					&3EML				&Adenosine A2a receptor		&nnn		&nnn		&nnn		&nnn		\\
	kif11						&					&							&			&			&			&			\\ \hline
	\end{tabular}
\end{table}

\chapter{ROC曲線}\label{appendix:roc}
DUD-Eの102ターゲットそれぞれについて、7通りの手法のROC曲線を記載する。
1つのfigureに7つのROC曲線が描かれるイメージで、それが102個並ぶ。
3×4=12が1ページで、それが8ページ半続く形になることを想像している。