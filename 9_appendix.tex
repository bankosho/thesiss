\appendix
\chapter{DUD-Eの詳細}\label{appendix:dude}
\memo{以下のテーブルが102ターゲットについて続く。}
\begin{table}[htb] \centering
	\caption{DUD-Eの詳細}
	\label{tb:dude_description}
	\begin{tabular}{c|c|p{5cm}|rr|rr}
	\hline
	\multirow{2}{*}{ターゲット名}	&\multirow{2}{*}{PDBID}	&\multicolumn{1}{c|}{\multirow{2}{*}{タンパク質詳細}}	&\multicolumn{2}{c|}{化合物数}	&\multicolumn{2}{c}{平均フラグメント数}		\\
							&					&											&正例	&負例				&正例	&負例						\\ \hline
	aa2ar					&3EML				&Adenosine A2a receptor						&482	&31,498				&6.26	&7.04						\\
	abl1						&2HZI				&Tyrosine-protein kinase ABL					&182	&10,746				&7.08	&7.33						\\ \hline
	\end{tabular}
\end{table}

\chapter{ROC曲線}\label{appendix:roc}
DUD-Eの102ターゲットそれぞれについて、7通りの手法のROC曲線を記載する。
図\ref{fig:roc:1}のようなものがターゲットを変えながら102個並ぶ。
2×3=6が1ページで、それが18ページ続く形になることを想像している。

\begin{figure}[tb]
 \begin{minipage}{0.5\hsize}
  \begin{center}
   \fig[width=0.9\hsize]{./fig/appendix/aa2ar_gscore_ROC.pdf}
  \end{center}
 \end{minipage}
 \begin{minipage}{0.5\hsize}
  \begin{center}
   \fig[width=0.9\hsize]{./fig/appendix/abl1_gscore_ROC.pdf}
  \end{center}
 \end{minipage}
 \begin{minipage}{0.5\hsize}
  \begin{center}
   \fig[width=0.9\hsize]{./fig/appendix/ace_gscore_ROC.pdf}
  \end{center}
 \end{minipage}
 \begin{minipage}{0.5\hsize}
  \begin{center}
   \fig[width=0.9\hsize]{./fig/appendix/aces_gscore_ROC.pdf}
  \end{center}
 \end{minipage}
 \begin{minipage}{0.5\hsize}
  \begin{center}
   \fig[width=0.9\hsize]{./fig/appendix/ada_gscore_ROC.pdf}
  \end{center}
 \end{minipage}
 \begin{minipage}{0.5\hsize}
  \begin{center}
   \fig[width=0.9\hsize]{./fig/appendix/ada17_gscore_ROC.pdf}
  \end{center}
 \end{minipage}
  \caption{ROC曲線例}
  \label{fig:roc:1}
\end{figure}
