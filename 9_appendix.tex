\appendix
\chapter{DUD-Eの詳細}\label{appendix:dude}
DUD-E(A Database of Useful Decoys: Enhanced)はMysingerらによって作成されたドッキングシミュレーションツールを評価するための
データセットである\cite{Mysinger2012}.ターゲットは多様性を考慮して102種類が選択されており,
それぞれに対して
\begin{itemize}
\item 代表タンパク質構造
\item 正例となる既知の薬剤・阻害剤
\item 負例となるデコイ(実験は行われていないが,既知の薬剤・阻害剤と構造が似ていないためターゲットを阻害しないと
	考えられる化合物)および実験によってターゲットを阻害しないことが知られている化合物
\end{itemize}
が用意されている.
\begin{table}[htb] \centering
	\caption{DUD-Eの詳細}
	\label{tb:dude_description}
	\begin{tabular}{c|c|p{5cm}|rr|rr}
	\hline
	\multirow{2}{*}{ターゲット名}	&\multirow{2}{*}{PDBID}	&\multicolumn{1}{c|}{\multirow{2}{*}{タンパク質詳細}}	&\multicolumn{2}{c|}{化合物数}	&\multicolumn{2}{c}{平均フラグメント数}		\\
							&					&											&正例	&負例				&正例	&負例						\\ \hline
	aa2ar					&3EML				&Adenosine A2a receptor						&482	&31,498				&6.26	&7.04						\\
	abl1						&2HZI				&Tyrosine-protein kinase ABL					&182	&10,746				&7.08	&7.33						\\ \hline
	\end{tabular}
\end{table}

\chapter{各手法を単体で用いた場合のROC曲線}\label{appendix:roc}
\ref{subsec:single_accuracy}節にて示した
総和法(score\_sum),最良値法(score\_max),総和法と最良値法の値の線形和(maxsumBS)の3つの提案手法
およびGlide HTVSモードをそれぞれ単体で用いた場合のROC曲線を示す.
3つの提案手法についてはフラグメントの結合スコア算出をGlide SPモード/Glide HTVSモードで行った場合がそれぞれ示されている.
例えば,「score\_max\_SP」とはフラグメントの結合スコアをGlide SPモードで求め,そのフラグメントスコアの最良値をとった場合の精度が
ROC曲線で示されている.


        \begin{figure}[p]
        
     \begin{minipage}{0.49\hsize}
      \begin{center}
       \fig[width=0.9\hsize]{./fig/appendix/aa2ar_gscore_ROC.pdf}
      \end{center}
     \end{minipage}    
     \begin{minipage}{0.49\hsize}
      \begin{center}
       \fig[width=0.9\hsize]{./fig/appendix/abl1_gscore_ROC.pdf}
      \end{center}
     \end{minipage}    
     \begin{minipage}{0.49\hsize}
      \begin{center}
       \fig[width=0.9\hsize]{./fig/appendix/ace_gscore_ROC.pdf}
      \end{center}
     \end{minipage}    
     \begin{minipage}{0.49\hsize}
      \begin{center}
       \fig[width=0.9\hsize]{./fig/appendix/aces_gscore_ROC.pdf}
      \end{center}
     \end{minipage}    
     \begin{minipage}{0.49\hsize}
      \begin{center}
       \fig[width=0.9\hsize]{./fig/appendix/ada_gscore_ROC.pdf}
      \end{center}
     \end{minipage}    
     \begin{minipage}{0.49\hsize}
      \begin{center}
       \fig[width=0.9\hsize]{./fig/appendix/ada17_gscore_ROC.pdf}
      \end{center}
     \end{minipage}    
         \caption{各手法の単体性能ROC曲線(1)}
         \label{fig:roc:1}
        \end{figure}
        
        \begin{figure}[p]
        
     \begin{minipage}{0.49\hsize}
      \begin{center}
       \fig[width=0.9\hsize]{./fig/appendix/adrb1_gscore_ROC.pdf}
      \end{center}
     \end{minipage}    
     \begin{minipage}{0.49\hsize}
      \begin{center}
       \fig[width=0.9\hsize]{./fig/appendix/adrb2_gscore_ROC.pdf}
      \end{center}
     \end{minipage}    
     \begin{minipage}{0.49\hsize}
      \begin{center}
       \fig[width=0.9\hsize]{./fig/appendix/akt1_gscore_ROC.pdf}
      \end{center}
     \end{minipage}    
     \begin{minipage}{0.49\hsize}
      \begin{center}
       \fig[width=0.9\hsize]{./fig/appendix/akt2_gscore_ROC.pdf}
      \end{center}
     \end{minipage}    
     \begin{minipage}{0.49\hsize}
      \begin{center}
       \fig[width=0.9\hsize]{./fig/appendix/aldr_gscore_ROC.pdf}
      \end{center}
     \end{minipage}    
     \begin{minipage}{0.49\hsize}
      \begin{center}
       \fig[width=0.9\hsize]{./fig/appendix/ampc_gscore_ROC.pdf}
      \end{center}
     \end{minipage}    
         \caption{各手法の単体性能ROC曲線(2)}
         \label{fig:roc:2}
        \end{figure}
        
        \begin{figure}[p]
        
     \begin{minipage}{0.49\hsize}
      \begin{center}
       \fig[width=0.9\hsize]{./fig/appendix/andr_gscore_ROC.pdf}
      \end{center}
     \end{minipage}    
     \begin{minipage}{0.49\hsize}
      \begin{center}
       \fig[width=0.9\hsize]{./fig/appendix/aofb_gscore_ROC.pdf}
      \end{center}
     \end{minipage}    
     \begin{minipage}{0.49\hsize}
      \begin{center}
       \fig[width=0.9\hsize]{./fig/appendix/bace1_gscore_ROC.pdf}
      \end{center}
     \end{minipage}    
     \begin{minipage}{0.49\hsize}
      \begin{center}
       \fig[width=0.9\hsize]{./fig/appendix/braf_gscore_ROC.pdf}
      \end{center}
     \end{minipage}    
     \begin{minipage}{0.49\hsize}
      \begin{center}
       \fig[width=0.9\hsize]{./fig/appendix/cah2_gscore_ROC.pdf}
      \end{center}
     \end{minipage}    
     \begin{minipage}{0.49\hsize}
      \begin{center}
       \fig[width=0.9\hsize]{./fig/appendix/casp3_gscore_ROC.pdf}
      \end{center}
     \end{minipage}    
         \caption{各手法の単体性能ROC曲線(3)}
         \label{fig:roc:3}
        \end{figure}
        
        \begin{figure}[p]
        
     \begin{minipage}{0.49\hsize}
      \begin{center}
       \fig[width=0.9\hsize]{./fig/appendix/cdk2_gscore_ROC.pdf}
      \end{center}
     \end{minipage}    
     \begin{minipage}{0.49\hsize}
      \begin{center}
       \fig[width=0.9\hsize]{./fig/appendix/comt_gscore_ROC.pdf}
      \end{center}
     \end{minipage}    
     \begin{minipage}{0.49\hsize}
      \begin{center}
       \fig[width=0.9\hsize]{./fig/appendix/cp2c9_gscore_ROC.pdf}
      \end{center}
     \end{minipage}    
     \begin{minipage}{0.49\hsize}
      \begin{center}
       \fig[width=0.9\hsize]{./fig/appendix/cp3a4_gscore_ROC.pdf}
      \end{center}
     \end{minipage}    
     \begin{minipage}{0.49\hsize}
      \begin{center}
       \fig[width=0.9\hsize]{./fig/appendix/csf1r_gscore_ROC.pdf}
      \end{center}
     \end{minipage}    
     \begin{minipage}{0.49\hsize}
      \begin{center}
       \fig[width=0.9\hsize]{./fig/appendix/cxcr4_gscore_ROC.pdf}
      \end{center}
     \end{minipage}    
         \caption{各手法の単体性能ROC曲線(4)}
         \label{fig:roc:4}
        \end{figure}
        
        \begin{figure}[p]
        
     \begin{minipage}{0.49\hsize}
      \begin{center}
       \fig[width=0.9\hsize]{./fig/appendix/def_gscore_ROC.pdf}
      \end{center}
     \end{minipage}    
     \begin{minipage}{0.49\hsize}
      \begin{center}
       \fig[width=0.9\hsize]{./fig/appendix/dhi1_gscore_ROC.pdf}
      \end{center}
     \end{minipage}    
     \begin{minipage}{0.49\hsize}
      \begin{center}
       \fig[width=0.9\hsize]{./fig/appendix/dpp4_gscore_ROC.pdf}
      \end{center}
     \end{minipage}    
     \begin{minipage}{0.49\hsize}
      \begin{center}
       \fig[width=0.9\hsize]{./fig/appendix/drd3_gscore_ROC.pdf}
      \end{center}
     \end{minipage}    
     \begin{minipage}{0.49\hsize}
      \begin{center}
       \fig[width=0.9\hsize]{./fig/appendix/dyr_gscore_ROC.pdf}
      \end{center}
     \end{minipage}    
     \begin{minipage}{0.49\hsize}
      \begin{center}
       \fig[width=0.9\hsize]{./fig/appendix/egfr_gscore_ROC.pdf}
      \end{center}
     \end{minipage}    
         \caption{各手法の単体性能ROC曲線(5)}
         \label{fig:roc:5}
        \end{figure}
        
        \begin{figure}[p]
        
     \begin{minipage}{0.49\hsize}
      \begin{center}
       \fig[width=0.9\hsize]{./fig/appendix/esr1_gscore_ROC.pdf}
      \end{center}
     \end{minipage}    
     \begin{minipage}{0.49\hsize}
      \begin{center}
       \fig[width=0.9\hsize]{./fig/appendix/esr2_gscore_ROC.pdf}
      \end{center}
     \end{minipage}    
     \begin{minipage}{0.49\hsize}
      \begin{center}
       \fig[width=0.9\hsize]{./fig/appendix/fa10_gscore_ROC.pdf}
      \end{center}
     \end{minipage}    
     \begin{minipage}{0.49\hsize}
      \begin{center}
       \fig[width=0.9\hsize]{./fig/appendix/fa7_gscore_ROC.pdf}
      \end{center}
     \end{minipage}    
     \begin{minipage}{0.49\hsize}
      \begin{center}
       \fig[width=0.9\hsize]{./fig/appendix/fabp4_gscore_ROC.pdf}
      \end{center}
     \end{minipage}    
     \begin{minipage}{0.49\hsize}
      \begin{center}
       \fig[width=0.9\hsize]{./fig/appendix/fak1_gscore_ROC.pdf}
      \end{center}
     \end{minipage}    
         \caption{各手法の単体性能ROC曲線(6)}
         \label{fig:roc:6}
        \end{figure}
        
        \begin{figure}[p]
        
     \begin{minipage}{0.49\hsize}
      \begin{center}
       \fig[width=0.9\hsize]{./fig/appendix/fgfr1_gscore_ROC.pdf}
      \end{center}
     \end{minipage}    
     \begin{minipage}{0.49\hsize}
      \begin{center}
       \fig[width=0.9\hsize]{./fig/appendix/fkb1a_gscore_ROC.pdf}
      \end{center}
     \end{minipage}    
     \begin{minipage}{0.49\hsize}
      \begin{center}
       \fig[width=0.9\hsize]{./fig/appendix/fnta_gscore_ROC.pdf}
      \end{center}
     \end{minipage}    
     \begin{minipage}{0.49\hsize}
      \begin{center}
       \fig[width=0.9\hsize]{./fig/appendix/fpps_gscore_ROC.pdf}
      \end{center}
     \end{minipage}    
     \begin{minipage}{0.49\hsize}
      \begin{center}
       \fig[width=0.9\hsize]{./fig/appendix/gcr_gscore_ROC.pdf}
      \end{center}
     \end{minipage}    
     \begin{minipage}{0.49\hsize}
      \begin{center}
       \fig[width=0.9\hsize]{./fig/appendix/glcm_gscore_ROC.pdf}
      \end{center}
     \end{minipage}    
         \caption{各手法の単体性能ROC曲線(7)}
         \label{fig:roc:7}
        \end{figure}
        
        \begin{figure}[p]
        
     \begin{minipage}{0.49\hsize}
      \begin{center}
       \fig[width=0.9\hsize]{./fig/appendix/gria2_gscore_ROC.pdf}
      \end{center}
     \end{minipage}    
     \begin{minipage}{0.49\hsize}
      \begin{center}
       \fig[width=0.9\hsize]{./fig/appendix/grik1_gscore_ROC.pdf}
      \end{center}
     \end{minipage}    
     \begin{minipage}{0.49\hsize}
      \begin{center}
       \fig[width=0.9\hsize]{./fig/appendix/hdac2_gscore_ROC.pdf}
      \end{center}
     \end{minipage}    
     \begin{minipage}{0.49\hsize}
      \begin{center}
       \fig[width=0.9\hsize]{./fig/appendix/hdac8_gscore_ROC.pdf}
      \end{center}
     \end{minipage}    
     \begin{minipage}{0.49\hsize}
      \begin{center}
       \fig[width=0.9\hsize]{./fig/appendix/hivint_gscore_ROC.pdf}
      \end{center}
     \end{minipage}    
     \begin{minipage}{0.49\hsize}
      \begin{center}
       \fig[width=0.9\hsize]{./fig/appendix/hivpr_gscore_ROC.pdf}
      \end{center}
     \end{minipage}    
         \caption{各手法の単体性能ROC曲線(8)}
         \label{fig:roc:8}
        \end{figure}
        
        \begin{figure}[p]
        
     \begin{minipage}{0.49\hsize}
      \begin{center}
       \fig[width=0.9\hsize]{./fig/appendix/hivrt_gscore_ROC.pdf}
      \end{center}
     \end{minipage}    
     \begin{minipage}{0.49\hsize}
      \begin{center}
       \fig[width=0.9\hsize]{./fig/appendix/hmdh_gscore_ROC.pdf}
      \end{center}
     \end{minipage}    
     \begin{minipage}{0.49\hsize}
      \begin{center}
       \fig[width=0.9\hsize]{./fig/appendix/hs90a_gscore_ROC.pdf}
      \end{center}
     \end{minipage}    
     \begin{minipage}{0.49\hsize}
      \begin{center}
       \fig[width=0.9\hsize]{./fig/appendix/hxk4_gscore_ROC.pdf}
      \end{center}
     \end{minipage}    
     \begin{minipage}{0.49\hsize}
      \begin{center}
       \fig[width=0.9\hsize]{./fig/appendix/igf1r_gscore_ROC.pdf}
      \end{center}
     \end{minipage}    
     \begin{minipage}{0.49\hsize}
      \begin{center}
       \fig[width=0.9\hsize]{./fig/appendix/inha_gscore_ROC.pdf}
      \end{center}
     \end{minipage}    
         \caption{各手法の単体性能ROC曲線(9)}
         \label{fig:roc:9}
        \end{figure}
        
        \begin{figure}[p]
        
     \begin{minipage}{0.49\hsize}
      \begin{center}
       \fig[width=0.9\hsize]{./fig/appendix/ital_gscore_ROC.pdf}
      \end{center}
     \end{minipage}    
     \begin{minipage}{0.49\hsize}
      \begin{center}
       \fig[width=0.9\hsize]{./fig/appendix/jak2_gscore_ROC.pdf}
      \end{center}
     \end{minipage}    
     \begin{minipage}{0.49\hsize}
      \begin{center}
       \fig[width=0.9\hsize]{./fig/appendix/kif11_gscore_ROC.pdf}
      \end{center}
     \end{minipage}    
     \begin{minipage}{0.49\hsize}
      \begin{center}
       \fig[width=0.9\hsize]{./fig/appendix/kit_gscore_ROC.pdf}
      \end{center}
     \end{minipage}    
     \begin{minipage}{0.49\hsize}
      \begin{center}
       \fig[width=0.9\hsize]{./fig/appendix/kith_gscore_ROC.pdf}
      \end{center}
     \end{minipage}    
     \begin{minipage}{0.49\hsize}
      \begin{center}
       \fig[width=0.9\hsize]{./fig/appendix/kpcb_gscore_ROC.pdf}
      \end{center}
     \end{minipage}    
         \caption{各手法の単体性能ROC曲線(10)}
         \label{fig:roc:10}
        \end{figure}
        
        \begin{figure}[p]
        
     \begin{minipage}{0.49\hsize}
      \begin{center}
       \fig[width=0.9\hsize]{./fig/appendix/lck_gscore_ROC.pdf}
      \end{center}
     \end{minipage}    
     \begin{minipage}{0.49\hsize}
      \begin{center}
       \fig[width=0.9\hsize]{./fig/appendix/lkha4_gscore_ROC.pdf}
      \end{center}
     \end{minipage}    
     \begin{minipage}{0.49\hsize}
      \begin{center}
       \fig[width=0.9\hsize]{./fig/appendix/mapk2_gscore_ROC.pdf}
      \end{center}
     \end{minipage}    
     \begin{minipage}{0.49\hsize}
      \begin{center}
       \fig[width=0.9\hsize]{./fig/appendix/mcr_gscore_ROC.pdf}
      \end{center}
     \end{minipage}    
     \begin{minipage}{0.49\hsize}
      \begin{center}
       \fig[width=0.9\hsize]{./fig/appendix/met_gscore_ROC.pdf}
      \end{center}
     \end{minipage}    
     \begin{minipage}{0.49\hsize}
      \begin{center}
       \fig[width=0.9\hsize]{./fig/appendix/mk01_gscore_ROC.pdf}
      \end{center}
     \end{minipage}    
         \caption{各手法の単体性能ROC曲線(11)}
         \label{fig:roc:11}
        \end{figure}
        
        \begin{figure}[p]
        
     \begin{minipage}{0.49\hsize}
      \begin{center}
       \fig[width=0.9\hsize]{./fig/appendix/mk10_gscore_ROC.pdf}
      \end{center}
     \end{minipage}    
     \begin{minipage}{0.49\hsize}
      \begin{center}
       \fig[width=0.9\hsize]{./fig/appendix/mk14_gscore_ROC.pdf}
      \end{center}
     \end{minipage}    
     \begin{minipage}{0.49\hsize}
      \begin{center}
       \fig[width=0.9\hsize]{./fig/appendix/mmp13_gscore_ROC.pdf}
      \end{center}
     \end{minipage}    
     \begin{minipage}{0.49\hsize}
      \begin{center}
       \fig[width=0.9\hsize]{./fig/appendix/mp2k1_gscore_ROC.pdf}
      \end{center}
     \end{minipage}    
     \begin{minipage}{0.49\hsize}
      \begin{center}
       \fig[width=0.9\hsize]{./fig/appendix/nos1_gscore_ROC.pdf}
      \end{center}
     \end{minipage}    
     \begin{minipage}{0.49\hsize}
      \begin{center}
       \fig[width=0.9\hsize]{./fig/appendix/nram_gscore_ROC.pdf}
      \end{center}
     \end{minipage}    
         \caption{各手法の単体性能ROC曲線(12)}
         \label{fig:roc:12}
        \end{figure}
        
        \begin{figure}[p]
        
     \begin{minipage}{0.49\hsize}
      \begin{center}
       \fig[width=0.9\hsize]{./fig/appendix/pa2ga_gscore_ROC.pdf}
      \end{center}
     \end{minipage}    
     \begin{minipage}{0.49\hsize}
      \begin{center}
       \fig[width=0.9\hsize]{./fig/appendix/parp1_gscore_ROC.pdf}
      \end{center}
     \end{minipage}    
     \begin{minipage}{0.49\hsize}
      \begin{center}
       \fig[width=0.9\hsize]{./fig/appendix/pde5a_gscore_ROC.pdf}
      \end{center}
     \end{minipage}    
     \begin{minipage}{0.49\hsize}
      \begin{center}
       \fig[width=0.9\hsize]{./fig/appendix/pgh1_gscore_ROC.pdf}
      \end{center}
     \end{minipage}    
     \begin{minipage}{0.49\hsize}
      \begin{center}
       \fig[width=0.9\hsize]{./fig/appendix/pgh2_gscore_ROC.pdf}
      \end{center}
     \end{minipage}    
     \begin{minipage}{0.49\hsize}
      \begin{center}
       \fig[width=0.9\hsize]{./fig/appendix/plk1_gscore_ROC.pdf}
      \end{center}
     \end{minipage}    
         \caption{各手法の単体性能ROC曲線(13)}
         \label{fig:roc:13}
        \end{figure}
        
        \begin{figure}[p]
        
     \begin{minipage}{0.49\hsize}
      \begin{center}
       \fig[width=0.9\hsize]{./fig/appendix/pnph_gscore_ROC.pdf}
      \end{center}
     \end{minipage}    
     \begin{minipage}{0.49\hsize}
      \begin{center}
       \fig[width=0.9\hsize]{./fig/appendix/ppara_gscore_ROC.pdf}
      \end{center}
     \end{minipage}    
     \begin{minipage}{0.49\hsize}
      \begin{center}
       \fig[width=0.9\hsize]{./fig/appendix/ppard_gscore_ROC.pdf}
      \end{center}
     \end{minipage}    
     \begin{minipage}{0.49\hsize}
      \begin{center}
       \fig[width=0.9\hsize]{./fig/appendix/pparg_gscore_ROC.pdf}
      \end{center}
     \end{minipage}    
     \begin{minipage}{0.49\hsize}
      \begin{center}
       \fig[width=0.9\hsize]{./fig/appendix/prgr_gscore_ROC.pdf}
      \end{center}
     \end{minipage}    
     \begin{minipage}{0.49\hsize}
      \begin{center}
       \fig[width=0.9\hsize]{./fig/appendix/ptn1_gscore_ROC.pdf}
      \end{center}
     \end{minipage}    
         \caption{各手法の単体性能ROC曲線(14)}
         \label{fig:roc:14}
        \end{figure}
        
        \begin{figure}[p]
        
     \begin{minipage}{0.49\hsize}
      \begin{center}
       \fig[width=0.9\hsize]{./fig/appendix/pur2_gscore_ROC.pdf}
      \end{center}
     \end{minipage}    
     \begin{minipage}{0.49\hsize}
      \begin{center}
       \fig[width=0.9\hsize]{./fig/appendix/pygm_gscore_ROC.pdf}
      \end{center}
     \end{minipage}    
     \begin{minipage}{0.49\hsize}
      \begin{center}
       \fig[width=0.9\hsize]{./fig/appendix/pyrd_gscore_ROC.pdf}
      \end{center}
     \end{minipage}    
     \begin{minipage}{0.49\hsize}
      \begin{center}
       \fig[width=0.9\hsize]{./fig/appendix/reni_gscore_ROC.pdf}
      \end{center}
     \end{minipage}    
     \begin{minipage}{0.49\hsize}
      \begin{center}
       \fig[width=0.9\hsize]{./fig/appendix/rock1_gscore_ROC.pdf}
      \end{center}
     \end{minipage}    
     \begin{minipage}{0.49\hsize}
      \begin{center}
       \fig[width=0.9\hsize]{./fig/appendix/rxra_gscore_ROC.pdf}
      \end{center}
     \end{minipage}    
         \caption{各手法の単体性能ROC曲線(15)}
         \label{fig:roc:15}
        \end{figure}
        
        \begin{figure}[p]
        
     \begin{minipage}{0.49\hsize}
      \begin{center}
       \fig[width=0.9\hsize]{./fig/appendix/sahh_gscore_ROC.pdf}
      \end{center}
     \end{minipage}    
     \begin{minipage}{0.49\hsize}
      \begin{center}
       \fig[width=0.9\hsize]{./fig/appendix/src_gscore_ROC.pdf}
      \end{center}
     \end{minipage}    
     \begin{minipage}{0.49\hsize}
      \begin{center}
       \fig[width=0.9\hsize]{./fig/appendix/tgfr1_gscore_ROC.pdf}
      \end{center}
     \end{minipage}    
     \begin{minipage}{0.49\hsize}
      \begin{center}
       \fig[width=0.9\hsize]{./fig/appendix/thb_gscore_ROC.pdf}
      \end{center}
     \end{minipage}    
     \begin{minipage}{0.49\hsize}
      \begin{center}
       \fig[width=0.9\hsize]{./fig/appendix/thrb_gscore_ROC.pdf}
      \end{center}
     \end{minipage}    
     \begin{minipage}{0.49\hsize}
      \begin{center}
       \fig[width=0.9\hsize]{./fig/appendix/try1_gscore_ROC.pdf}
      \end{center}
     \end{minipage}    
         \caption{各手法の単体性能ROC曲線(16)}
         \label{fig:roc:16}
        \end{figure}
        
        \begin{figure}[p]
        
     \begin{minipage}{0.49\hsize}
      \begin{center}
       \fig[width=0.9\hsize]{./fig/appendix/tryb1_gscore_ROC.pdf}
      \end{center}
     \end{minipage}    
     \begin{minipage}{0.49\hsize}
      \begin{center}
       \fig[width=0.9\hsize]{./fig/appendix/tysy_gscore_ROC.pdf}
      \end{center}
     \end{minipage}    
     \begin{minipage}{0.49\hsize}
      \begin{center}
       \fig[width=0.9\hsize]{./fig/appendix/urok_gscore_ROC.pdf}
      \end{center}
     \end{minipage}    
     \begin{minipage}{0.49\hsize}
      \begin{center}
       \fig[width=0.9\hsize]{./fig/appendix/vgfr2_gscore_ROC.pdf}
      \end{center}
     \end{minipage}    
     \begin{minipage}{0.49\hsize}
      \begin{center}
       \fig[width=0.9\hsize]{./fig/appendix/wee1_gscore_ROC.pdf}
      \end{center}
     \end{minipage}    
     \begin{minipage}{0.49\hsize}
      \begin{center}
       \fig[width=0.9\hsize]{./fig/appendix/xiap_gscore_ROC.pdf}
      \end{center}
     \end{minipage}    
         \caption{各手法の単体性能ROC曲線(17)}
         \label{fig:roc:17}
        \end{figure}
        