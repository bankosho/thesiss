\chapter{結論}
\section{本研究の結論}\label{sec:conclusion}
本研究では、ドッキングに基づいた超高速なフィルタリング手法を提案した。
この手法の予測精度はドッキングに基づいたフィルタリングの既存手法であるglide HTVSモードに比べ劣っているが、
計算速度は既存手法では実現不可能なほど高速であり、さらにデータセットが大きくなるほど相対的に速度が向上していく。
\todo{提案手法(maxsumBS)と従来手法(glide HTVS)についての速度と精度に関する簡単な表を作成}

また、フィルタリング後に行う通常のドッキングシミュレーションと組み合わせた場合の速度・精度の評価を行い、
提案手法をフィルタリング手法として用いるべきユースケースを示した。
\todo{提案手法が有利になるケースに関する簡単な表を作成}

\section{今後の課題}
\ref{sec:conclusion}節で述べた通り、提案手法の精度は未だ不十分であるため、
手法の得手不得手の調査の継続や、フラグメント結合スコアの算出方法の改善などを含め、
なるべく速度を維持しつつも精度を高める必要がある。
また、本研究では最大でも数万化合物程度のデータセットを用いて評価を行ったが、
数百万~数千万程度の、より大規模な化合物データセットを用いた速度評価を行う必要がある。