\chapter{結論}
\section{本研究の結論}\label{sec:conclusion}
本研究では、ドッキングに基づいた超高速なフィルタリング手法を提案した。
提案手法をDUD-Eの全ターゲットである102種のデータセットを用いて評価すると、
予測精度はドッキングに基づいたフィルタリングの既存手法であるGlide HTVSモードに比べ劣っているが
計算速度は既存手法では実現不可能なほど高速であることが示された。
\begin{table}[h] \centering
	\caption{提案手法の性能}
	\label{table:conclusion_1}
	\begin{tabular}{l|rrrrrr}
	\hline
	\multicolumn{1}{c|}{\multirow{2}{*}{手法}}	&\multirow{2}{*}{ROC-AUC}	&\multicolumn{4}{c}{Enrichment Factor}	&平均計算時間	\\
										&						&EF(1\%)	&EF(2\%)	&EF(5\%)	&EF(10\%)	&[CPU sec.]		\\ \hline
	\b{提案手法(maxsumBS)}				&0.679					&6.03	&5.03	&3.96	&3.00		&\b{1,673}		\\
	従来手法(Glide HTVSモード)				&0.705				&16.67	&11.18	&6.38	&4.11		&14,813			\\ \hline
	\end{tabular}
\end{table}



また、フィルタリング後に行う通常のドッキングシミュレーションと組み合わせた場合の速度・精度の評価を行い、
提案手法をフィルタリング手法として用いるべきユースケースを示した。
\begin{table}[htbp] \centering
	\caption{通常ドッキング(Glide SP)と組み合わせた速度・精度評価}
	\label{table:conclusion_2}
	\begin{tabular}{l|rrr}
	\hline
												&\multirow{2}{*}{化合物ライブラリサイズ}	&上位1万化合物の		&推定計算時間	\\
												&														&濃縮率(EF)				&[CPU days]		\\ \hline
	提案手法で1\%フィルタリング		&1,000万											&\b{20.26 (EF 0.1\%)}		&\b{26.9}				\\
	従来手法で10\%フィルタリング	&100万												&16.89 (EF 1\%)			&32.0				\\ \hline
	\end{tabular}
\end{table}


\section{今後の課題}
今後に向けて、以下の課題が考えられる。
\begin{itemize}
\item 速度をなるべく維持しつつの精度の向上
	\begin{itemize}
	\item フラグメントをドッキングする際のスコア関数の改善
	\item 通常ドッキングシミュレーションの化合物の結合スコアへの、フィルタリングスコアのフィッティング
	\item 化合物のスコア算出時の非現実的なフラグメント配置に対するペナルティの付与
	\end{itemize}
\item どのようなケースで提案手法を用いるのが好ましいかのさらなる調査
\item 数百万~数千万化合物程度の、より現実のバーチャルスクリーニングに即した化合物データセットを用いた速度評価
\item 提案手法と従来手法の2段階フィルタリングを行った場合の性能・速度評価
\end{itemize}
