\chapter{結論}
\section{本研究の結論}\label{sec:conclusion}
本研究では,ドッキングに基づいた高速なフィルタリング手法を提案した.
提案手法をDUD-Eの全ターゲットである102種のデータセットを用いて評価すると,
予測精度は簡易ドッキングシミュレーションであるGlide HTVSモードに比べ劣っているが
計算速度は既存手法では実現不可能なほど高速であることが示された.
\begin{table}[h] \centering
	\caption{提案手法の性能}
	\label{table:conclusion_1}
	\begin{tabular}{l|rrrrrr}
	\hline
	\multicolumn{1}{c|}{\multirow{2}{*}{手法}}	&\multirow{2}{*}{ROC-AUC}	&\multicolumn{4}{c}{Enrichment Factor}	&平均計算時間	\\
										&						&EF(1\%)	&EF(2\%)	&EF(5\%)	&EF(10\%)	&[CPU sec.]		\\ \hline
	\b{提案手法(maxsumBS, }	&&&&&& \\
	\b{フラグメントドッキング:}				&0.679					&6.03	&5.03	&3.96	&3.00		&\b{1,673}		\\
	\b{Glide SPモード)}			&&&&&& \\ \hline
	従来手法(簡易ドッキング	&&&&&& \\
	シミュレーション	Glide 		&0.705&16.67&11.18&6.38&4.11&14,813 \\ 
	HTVSモード 通常利用)		&&&&&& \\
	\hline
	\end{tabular}
\end{table}

また,フィルタリング後に行う通常のドッキングシミュレーションと組み合わせた場合の速度・精度の評価を行い,
提案手法をフィルタリング手法として用いるべきユースケースを示した.
\begin{table}[b] \centering
	\caption{通常ドッキング(Glide SP)と組み合わせた速度・精度評価}
	\label{table:conclusion_2}
	\begin{tabular}{l|rrr}
	\hline
												&\multirow{2}{*}{化合物数}	&上位1万化合物の		&推定計算時間	\\
												&														&濃縮率(EF)				&[CPU days]		\\ \hline
	提案手法で1\%フィルタリング		&1,000万											&\b{20.26 (EF 0.1\%)}		&\b{26.9}				\\
	従来手法で10\%フィルタリング	&100万												&16.89 (EF 1\%)			&32.0				\\ \hline
	\end{tabular}
	\vspace{-0.5cm}
\end{table}

\newpage

\section{今後の課題}
本研究の今後の課題として,以下の事項が考えられる.
\begin{enumerate}
\item 速度をなるべく維持しつつの精度の向上\\
	提案手法は高速な計算を可能にしている一方,精度は簡易ドッキングシミュレーションに劣っており,改善の余地がある.
	改善の方策として以下が考えられる.
	\begin{itemize}
	\item フラグメントをドッキングする際のスコア関数の改善
	\item 通常ドッキングシミュレーションの化合物の結合スコアへの,フィルタリングスコアのフィッティング
	\item 総和法(score\_sum)のペナルティのターゲットごとの調整
	\item 化合物のスコア算出時の非現実的なフラグメント配置に対するペナルティの付与
	\end{itemize}
\item 数百万~数千万化合物程度の,より現実のバーチャルスクリーニングに即した化合物データセットを用いた速度評価\\
	本研究では一般的なベンチマークデータセットであるDUD-Eを用いた評価を行ったが,これは1つのターゲットに対して最大でも約50,000個
	しか化合物が登録されていない.
	現実のバーチャルスクリーニングで一般的に行われている数百万個の化合物を用いた評価を行うことで,提案手法の有用性をより明示的に
	示すことができると考えられる.
\item 提案手法と従来手法の2段階フィルタリングを行った場合の性能・速度評価\\
	本研究の結果,提案手法は簡易ドッキングシミュレーションのGlide HTVSモードよりも高速であるが精度は劣っていた.
	通常のドッキングシミュレーションに対するフィルタリングのように,簡易ドッキングシミュレーションの前に提案手法を用いることで
	予測精度を保ちつつ計算量を削減することができると考えられる.
\item Glide以外のツールを利用した場合の提案手法の評価\\
	本研究ではフラグメントのドッキングシミュレーションにはGlide SPモードないしはGlide HTVSモードを利用したが,
	ここで用いるソフトウェアは自由に選択することができる.他のソフトウェアを使った場合の精度や速度の評価を行うことで,
	手法の汎用性を確かめることは重要である.
\end{enumerate}
